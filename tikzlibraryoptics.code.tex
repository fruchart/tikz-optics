\iffalse
The tikzoptics library

Copyright (C) 2013-2019 by
  Michel Fruchart <michel.fruchart@ens-lyon.org>

This work may be distributed and/or modified under the conditions of
the LaTeX Project Public License (LPPL), version 1.3c, which can be
found at the address:
https://www.latex-project.org/lppl/lppl-1-3c/

Alternatively, it may be distributed and/or modified under the conditions of
the GNU General Public License (GNU GPL), version 2, which can be found
at the address:
https://www.gnu.org/licenses/gpl-2.0.en.html
or any later version published by the Free Software Foundation.
\fi
\makeatletter


\def\tikzopticsversion{0.2.4}
\def\tikzopticsversiondate{2017-10-07}


%%%%%%%%%%%%%%%%%%%%%%%%%%%%%%%%%%%%%%%%%%%%%%%%%%%%%%%%%%%%%%%%%%%%%%%%%%%%%%%%
% errors from this library
%%%%%%%%%%%%%%%%%%%%%%%%%%%%%%%%%%%%%%%%%%%%%%%%%%%%%%%%%%%%%%%%%%%%%%%%%%%%%%%%
\def\opticserror#1{\pgfutil@packageerror{tikz/optics}{#1}{}}


%%%%%%%%%%%%%%%%%%%%%%%%%%%%%%%%%%%%%%%%%%%%%%%%%%%%%%%%%%%%%%%%%%%%%%%%%%%%%%%%
% handler |collect unknowns|
%%%%%%%%%%%%%%%%%%%%%%%%%%%%%%%%%%%%%%%%%%%%%%%%%%%%%%%%%%%%%%%%%%%%%%%%%%%%%%%%
% from http://tex.stackexchange.com/questions/81821/filtering-options-with-pgfkeys/81985#81985
% used by /tikz/optics/->n-
%
\pgfkeys{
  /handlers/.collect unknowns/.style = {
    unknown options/.initial = {},
    .unknown/.code = {%
      \letcs\reserved{pgfk@\pgfkeyscurrentpath/unknown options}%
      \csedef{pgfk@\pgfkeyscurrentpath/unknown options}{%
        \ifx\reserved\empty\else\expandonce\reserved,\fi
        \expandonce\pgfkeyscurrentname
        \ifx\pgfkeysnovalue##1\else=\expandonce\pgfkeyscurrentvalue\fi
      }%
    }
  }
}

%%%%%%%%%%%%%%%%%%%%%%%%%%%%%%%%%%%%%%%%%%%%%%%%%%%%%%%%%%%%%%%%%%%%%%%%%%%%%%%%
% defining |optics| family
%%%%%%%%%%%%%%%%%%%%%%%%%%%%%%%%%%%%%%%%%%%%%%%%%%%%%%%%%%%%%%%%%%%%%%%%%%%%%%%%

\pgfkeys{
  /tikz/.cd,
  optics/.is family,
  optics/.search also={/tikz},
}

%%%%%%%%%%%%%%%%%%%%%%%%%%%%%%%%%%%%%%%%%%%%%%%%%%%%%%%%%%%%%%%%%%%%%%%%%%%%%%%%
% [use optics] 
%%%%%%%%%%%%%%%%%%%%%%%%%%%%%%%%%%%%%%%%%%%%%%%%%%%%%%%%%%%%%%%%%%%%%%%%%%%%%%%%
\pgfkeys{
  /tikz/use optics/.append code={
  \pgfkeys{
    /tikz/.cd,
      % shapes 
      % FIXME il ne devrait pas y avoir de logique (ni de valeurs par défaut) ici, seulement l'export dans l'espace de nom commun
      lens/.prefix style={shape=lens,optics,draw},
      slit/.prefix style={shape=slit,optics,draw},
      double slit/.prefix style={shape=double slit,optics,draw},
      thin optics element/.prefix style={shape=thin optics element,optics,draw},
      polarizer/.prefix style={shape=polarizer,optics,draw,object aspect ratio=0.1},
      generic optics io/.prefix style={shape=generic optics io,optics,draw},
      sensor/.prefix style={shape=sensor,optics,object aspect ratio=1,draw},
      sensor line/.prefix style={shape=sensor line,optics,draw},
      mirror/.prefix style={shape=mirror,optics,draw},
      spherical mirror/.prefix style={shape=spherical mirror,optics,draw},
      thick optics element/.prefix style={shape=thick optics element,optics,draw,object aspect ratio=0.1},
      heat filter/.prefix style={shape=thick optics element,optics,draw, object aspect ratio=0.05},
      double amici prism/.prefix style={shape=double amici prism,optics,draw, prism height=1cm, prism apex angle=60},
      % styles
      screen/.prefix style={optics,/tikz/optics/screen},
      diffraction grating/.style={optics,/tikz/optics/diffraction grating},
      grid/.style={optics, /tikz/optics/grid},
      semi-transparent mirror/.style={optics, /tikz/optics/semi-transparent mirror},
      diaphragm/.style={optics, /tikz/optics/diaphragm},
      beam splitter/.prefix style={optics,/tikz/optics/beam splitter},
      generic lamp/.prefix style={optics,/tikz/optics/generic lamp},
      generic sensor/.prefix style={optics,/tikz/optics/generic sensor},
      halogen lamp/.prefix style={optics,/tikz/optics/halogen lamp},
      spectral lamp/.prefix style={optics,/tikz/optics/spectral lamp},
      laser/.prefix style={optics,/tikz/optics/laser},
      laser'/.prefix style={optics,/tikz/optics/laser'},
      concave mirror/.prefix style={optics, /tikz/optics/concave mirror},
      convex mirror/.prefix style={optics, /tikz/optics/convex mirror},
      % arrows (styles)
      ->-/.prefix style={optics,/tikz/optics/->-={##1}},
      -<-/.prefix style={optics,/tikz/optics/-<-={##1}},
      ->>-/.prefix style={optics,/tikz/optics/->>-={##1}},
      -<<-/.prefix style={optics,/tikz/optics/-<<-={##1}},
      ->>>-/.prefix style={optics,/tikz/optics/->>>-={##1}},
      -<<<-/.prefix style={optics,/tikz/optics/-<<<-={##1}},
      ->>>>-/.prefix style={optics,/tikz/optics/->>>>-={##1}},
      -<<<<-/.prefix style={optics,/tikz/optics/-<<<<-={##1}},
      ->n-/.prefix style={optics,/tikz/optics/->n-={##1}},
      -<n-/.prefix style={optics,/tikz/optics/-<n-={##1}},
    }
  }
}

%%%%%%%%%%%%%%%%%%%%%%%%%%%%%%%%%%%%%%%%%%%%%%%%%%%%%%%%%%%%%%%%%%%%%%%%%%%%%%%%
% We need some existing tikz libraries and tex packages.
%%%%%%%%%%%%%%%%%%%%%%%%%%%%%%%%%%%%%%%%%%%%%%%%%%%%%%%%%%%%%%%%%%%%%%%%%%%%%%%%
\usetikzlibrary{decorations,decorations.markings, decorations.pathreplacing}
\usetikzlibrary{shapes.misc}
\usepackage{etoolbox}


%%%%%%%%%%%%%%%%%%%%%%%%%%%%%%%%%%%%%%%%%%%%%%%%%%%%%%%%%%%%%%%%%%%%%%%%%%%%%%%%
% Key |object height|
%%%%%%%%%%%%%%%%%%%%%%%%%%%%%%%%%%%%%%%%%%%%%%%%%%%%%%%%%%%%%%%%%%%%%%%%%%%%%%%%

\pgfkeys{/tikz/optics/.cd,
  object height/.initial=2cm,
}

%%%%%%%%%%%%%%%%%%%%%%%%%%%%%%%%%%%%%%%%%%%%%%%%%%%%%%%%%%%%%%%%%%%%%%%%%%%%%%%%
% Shape [thin optics element]
%%%%%%%%%%%%%%%%%%%%%%%%%%%%%%%%%%%%%%%%%%%%%%%%%%%%%%%%%%%%%%%%%%%%%%%%%%%%%%%%
\pgfdeclareshape{thin optics element}
{
  \savedanchor{\center}{
    \pgfpointorigin
  }
  \anchor{center}{\center}

  \savedmacro\objectHeight{%
    \edef\objectHeight{\pgfkeysvalueof{/tikz/optics/object height}}%
  }

  \savedanchor{\north}{
    \pgf@x=0cm%
    \pgf@y=\objectHeight%
    \pgf@y=0.5\pgf@y%
  }
  \anchor{north}{\north}

  \savedanchor{\south}{
    \pgf@x=0cm%
    \pgf@y=\objectHeight%
    \pgf@y=-0.5\pgf@y%
  }
  \anchor{south}{\south}

  \anchor{east}{\center}
  \anchor{west}{\center}

  \anchorborder{%
    \pgf@xb=\pgf@x% xb/yb is target
    \pgf@yb=\pgf@y%
    \south%
    \pgf@xa=\pgf@x% xa/ya is se
    \pgf@ya=\pgf@y%
    \north%
    \advance\pgf@x by-\pgf@xa%
    \advance\pgf@y by-\pgf@ya%
    \pgf@xc=.5\pgf@x% x/y is half width/height
    \pgf@yc=.5\pgf@y%
    \advance\pgf@xa by\pgf@xc% xa/ya becomes center
    \advance\pgf@ya by\pgf@yc%
    \edef\pgf@marshal{%
      \noexpand\pgfpointborderrectangle
      {\noexpand\pgfpoint{\the\pgf@xb}{\the\pgf@yb}}
      {\noexpand\pgfpoint{\the\pgf@xc}{\the\pgf@yc}}%
    }%
    \pgf@process{\pgf@marshal}%
    \advance\pgf@x by\pgf@xa%
    \advance\pgf@y by\pgf@ya%
  }

  \backgroundpath
  {
    \north \pgf@xa=\pgf@x \pgf@ya=\pgf@y
    \south \pgf@xb=\pgf@x \pgf@yb=\pgf@y

    \pgfpathmoveto{\pgfpoint{\pgf@xa}{\pgf@ya}}
    \pgfpathlineto{\pgfpoint{\pgf@xb}{\pgf@yb}}
  }
}


%%%%%%%%%%%%%%%%%%%%%%%%%%%%%%%%%%%%%%%%%%%%%%%%%%%%%%%%%%%%%%%%%%%%%%%%%%%%%%%%
% Shape [lens]
%%%%%%%%%%%%%%%%%%%%%%%%%%%%%%%%%%%%%%%%%%%%%%%%%%%%%%%%%%%%%%%%%%%%%%%%%%%%%%%%
\newif\iftikz@optics@lens@converging
\tikz@optics@lens@convergingtrue

% keys
% - focal lens is used to define anchors |east focus| and |west focus|
% - lens height is used to define anchors |lens north| and |lens south|
% - lens type (converging or diverging)
\pgfkeys{/tikz/optics/.cd,
  focal length/.initial=1cm,
  lens height/.initial=0.8,
  lens type/.is choice,
  lens type/converging/.code={\tikz@optics@lens@convergingtrue},
  lens type/diverging/.code={\tikz@optics@lens@convergingfalse}
}
\pgfdeclareshape{lens}
{
  \savedanchor{\center}{
    \pgfpointorigin
  }
  \anchor{center}{\center}

  \savedmacro\focalLength{%
    \edef\focalLength{\pgfkeysvalueof{/tikz/optics/focal length}}%
  }

  \savedmacro\objectHeight{%
    \edef\objectHeight{\pgfkeysvalueof{/tikz/optics/object height}}%
  }

  \savedmacro\lensHeight{%
    \pgfmathparse{\pgfkeysvalueof{/tikz/optics/lens height}}%
    \ifpgfmathunitsdeclared%
      \pgfmathsetlengthmacro{\lensHeight}{\pgfkeysvalueof{/tikz/optics/lens height}}%
    \else%
      \pgfmathsetlengthmacro{\lensHeight}{\pgfkeysvalueof{/tikz/optics/lens height}*\pgfkeysvalueof{/tikz/optics/object height}}%
    \fi%
  }

  \savedanchor{\lensnorth}{
    \pgfpointorigin
    \pgf@y=\lensHeight%
    \pgf@y=0.5\pgf@y%
  }
  \anchor{lens north}{\lensnorth}

  \savedanchor{\lenssouth}{
    \pgfpointorigin
    \pgf@y=\lensHeight%
    \pgf@y=-0.5\pgf@y%
  }
  \anchor{lens south}{\lenssouth}

  \savedanchor{\north}{
    \pgf@x=0cm%
    \pgf@y=\objectHeight%
    \pgf@y=0.5\pgf@y%
  }
  \anchor{north}{\north}

  \savedanchor{\south}{
    \pgf@x=0cm%
    \pgf@y=\objectHeight%
    \pgf@y=-0.5\pgf@y%
  }
  \anchor{south}{\south}

  \savedanchor{\eastfocal}{
    \pgf@x=\focalLength%
    \pgf@y=0cm%
  }
  \anchor{east focal point}{\eastfocal}
  \anchor{east focus}{\eastfocal}
  

  \savedanchor{\westfocal}{
    \pgf@x=-\focalLength%
    \pgf@y=0cm%
  }
  \anchor{west focal point}{\westfocal}
  \anchor{west focus}{\westfocal}

  \anchor{east}{\center}
  \anchor{west}{\center}

  \inheritanchorborder[from=thin optics element]

  \backgroundpath
  {
    \north \pgf@xb=\pgf@x \pgf@yb=\pgf@y
    \south \pgf@xa=\pgf@x \pgf@ya=\pgf@y
    
    \pgfpathmoveto{\pgfpoint{\pgf@xa}{\pgf@ya}}
    \pgfpathlineto{\pgfpoint{\pgf@xb}{\pgf@yb}}

    \iftikz@optics@lens@converging
      \pgfsetarrowsstart{lens arrow}
      \pgfsetarrowsend{lens arrow}
    \else
      \pgfsetarrowsstart{lens arrow reversed}
      \pgfsetarrowsend{lens arrow reversed}
    \fi
  }
}


%%%%%%%%%%%%%%%%%%%%%%%%%%%%%%%%%%%%%%%%%%%%%%%%%%%%%%%%%%%%%%%%%%%%%%%%%%%%%%%%
% Shape [slit]
%%%%%%%%%%%%%%%%%%%%%%%%%%%%%%%%%%%%%%%%%%%%%%%%%%%%%%%%%%%%%%%%%%%%%%%%%%%%%%%%
% keys : slit height
\pgfkeys{/tikz/optics/.cd,
  slit height/.initial=0.1
}

\pgfdeclareshape{slit}
{
  \savedanchor{\center}{
    \pgfpointorigin
  }
  \anchor{center}{\center}
  \anchor{slit center}{\center} 
  % TODO par cohérence ?

  \savedmacro\objectHeight{%
    \edef\objectHeight{\pgfkeysvalueof{/tikz/optics/object height}}%
  }

  \savedmacro\slitHeight{%
    \pgfmathparse{\pgfkeysvalueof{/tikz/optics/slit height}}%
    \ifpgfmathunitsdeclared%
      \pgfmathsetlengthmacro{\slitHeight}{\pgfkeysvalueof{/tikz/optics/slit height}}%
    \else%
      \pgfmathsetlengthmacro{\slitHeight}{\pgfkeysvalueof{/tikz/optics/slit height}*\pgfkeysvalueof{/tikz/optics/object height}}%
    \fi%
  }

  \savedanchor{\north}{
    \pgf@x=0cm%
    \pgf@y=\objectHeight%
    \pgf@y=0.5\pgf@y%
  }
  \anchor{north}{\north}

  \savedanchor{\south}{
    \pgf@x=0cm%
    \pgf@y=\objectHeight%
    \pgf@y=-0.5\pgf@y%
  }
  \anchor{south}{\south}

  \savedanchor{\slitnorth}{
    \pgf@x=0cm%
    \pgf@y=\slitHeight%
    \pgf@y=0.5\pgf@y%
  }
  \anchor{slit north}{\slitnorth}

  \savedanchor{\slitsouth}{
    \pgf@x=0cm%
    \pgf@y=\slitHeight%
    \pgf@y=-0.5\pgf@y%
  }
  \anchor{slit south}{\slitsouth}

  \anchor{east}{\center}
  \anchor{west}{\center}

  \inheritanchorborder[from=thin optics element]

  \backgroundpath
  {
    % erreurs possibles dans la spécification du dessin
    \pgfmathparse{notless(\slitHeight, \objectHeight)}
    \ifnum\pgfmathresult=1
      \opticserror{<slit height> should be strictly lower than <object height> (in slit)}
    \fi
    \north \pgf@xa=\pgf@x \pgf@ya=\pgf@y
    \slitnorth \pgf@xb=\pgf@x \pgf@yb=\pgf@y
    \pgfpathmoveto{\pgfpoint{\pgf@xa}{\pgf@ya}}
    \pgfpathlineto{\pgfpoint{\pgf@xb}{\pgf@yb}}

    \south \pgf@xa=\pgf@x \pgf@ya=\pgf@y
    \slitsouth \pgf@xb=\pgf@x \pgf@yb=\pgf@y
    \pgfpathmoveto{\pgfpoint{\pgf@xa}{\pgf@ya}}
    \pgfpathlineto{\pgfpoint{\pgf@xb}{\pgf@yb}}
  }
}




%%%%%%%%%%%%%%%%%%%%%%%%%%%%%%%%%%%%%%%%%%%%%%%%%%%%%%%%%%%%%%%%%%%%%%%%%%%%%%%%
% Shape [double slit]
%%%%%%%%%%%%%%%%%%%%%%%%%%%%%%%%%%%%%%%%%%%%%%%%%%%%%%%%%%%%%%%%%%%%%%%%%%%%%%%%
% keys :
% - slit height : (relative) height of the holes (each)
% - slit separation : (relative) distance between the centers of the holes
\pgfkeys{/tikz/optics/.cd,
  object height/.initial=2cm,
  slit height/.initial=0.075,
  slit separation/.initial=0.2
}

\pgfdeclareshape{double slit}
{
  \savedanchor{\center}{
    \pgfpointorigin
  }
  \anchor{center}{\center}
  
  \savedmacro\objectHeight{%
    \edef\objectHeight{\pgfkeysvalueof{/tikz/optics/object height}}%
  }

  \savedmacro\slitHeight{%
    \pgfmathparse{\pgfkeysvalueof{/tikz/optics/slit height}}%
    \ifpgfmathunitsdeclared%
      \pgfmathsetlengthmacro{\slitHeight}{\pgfkeysvalueof{/tikz/optics/slit height}}%
    \else%
      \pgfmathsetlengthmacro{\slitHeight}{\pgfkeysvalueof{/tikz/optics/slit height}*\pgfkeysvalueof{/tikz/optics/object height}}%
    \fi%
  }

  \savedmacro\slitSeparation{%
    \pgfmathparse{\pgfkeysvalueof{/tikz/optics/slit separation}}%
    \ifpgfmathunitsdeclared%
      \pgfmathsetlengthmacro{\slitSeparation}{\pgfkeysvalueof{/tikz/optics/slit separation}}%
    \else%
      \pgfmathsetlengthmacro{\slitSeparation}{\pgfkeysvalueof{/tikz/optics/slit separation}*\pgfkeysvalueof{/tikz/optics/object height}}%
    \fi%
  }

  \savedmacro\macro@slitOneCenter{
    \def\macro@slitOneCenter{
      \pgfpointorigin
      \pgf@ya=\slitSeparation
      \pgf@ya=0.5\pgf@ya
      \advance \pgf@y by \pgf@ya
    }
  }

  \savedmacro\macro@slitTwoCenter{
    \def\macro@slitTwoCenter{
      \pgfpointorigin
      \pgf@ya=\slitSeparation
      \pgf@ya=-0.5\pgf@ya
      \advance \pgf@y by \pgf@ya
    }
  }

  \savedanchor{\north}{
    \pgf@x=0cm%
    \pgf@y=\objectHeight%
    \pgf@y=0.5\pgf@y%
  }
  \anchor{north}{\north}

  \savedanchor{\south}{
    \pgf@x=0cm%
    \pgf@y=\objectHeight%
    \pgf@y=-0.5\pgf@y%
  }
  \anchor{south}{\south}

  % slit 1
  \savedanchor{\slitOneCenter}{
    \macro@slitOneCenter
  }
  \anchor{slit 1 center}{\slitOneCenter}

  \savedanchor{\slitOneNorth}{
    \macro@slitOneCenter
    \pgf@ya = \slitHeight
    \pgf@ya = 0.5\pgf@ya
    \advance \pgf@y by \pgf@ya
  }
  \anchor{slit 1 north}{\slitOneNorth}

  \savedanchor{\slitOneSouth}{
    \macro@slitOneCenter
    \pgf@ya = \slitHeight
    \pgf@ya = -0.5\pgf@ya
    \advance \pgf@y by \pgf@ya
  }
  \anchor{slit 1 south}{\slitOneSouth}

  % slit 2
  \savedanchor{\slitTwoCenter}{
    \macro@slitTwoCenter
  }
  \anchor{slit 2 center}{\slitTwoCenter}

  \savedanchor{\slitTwoNorth}{
    \macro@slitTwoCenter
    \pgf@ya = \slitHeight
    \pgf@ya = 0.5\pgf@ya
    \advance \pgf@y by \pgf@ya
  }
  \anchor{slit 2 north}{\slitTwoNorth}

  \savedanchor{\slitTwoSouth}{
    \macro@slitTwoCenter
    \pgf@ya = \slitHeight
    \pgf@ya = -0.5\pgf@ya
    \advance \pgf@y by \pgf@ya
  }
  \anchor{slit 2 south}{\slitTwoSouth}

  \anchor{east}{\center}
  \anchor{west}{\center}

  \inheritanchorborder[from=thin optics element]

  \backgroundpath
  {
    % erreurs possibles dans les spécifications du dessin
    \pgfmathparse{notgreater(\slitSeparation, \slitHeight)}
    \ifnum\pgfmathresult=1
      \opticserror{<slit separation> should be strictly lower than <slit height> (in double slit)}
    \fi
    \pgfmathparse{notless(\slitSeparation+\slitHeight, \objectHeight)}
    \ifnum\pgfmathresult=1
      \opticserror{<slit height> plus <slit separation> should be strictly lower than <object height> (in double slit)}
    \fi
    \north \pgf@xa=\pgf@x \pgf@ya=\pgf@y
    \slitOneNorth \pgf@xb=\pgf@x \pgf@yb=\pgf@y
    \pgfpathmoveto{\pgfpoint{\pgf@xa}{\pgf@ya}}
    \pgfpathlineto{\pgfpoint{\pgf@xb}{\pgf@yb}}

    \slitOneSouth \pgf@xa=\pgf@x \pgf@ya=\pgf@y
    \slitTwoNorth \pgf@xb=\pgf@x \pgf@yb=\pgf@y
    \pgfpathmoveto{\pgfpoint{\pgf@xa}{\pgf@ya}}
    \pgfpathlineto{\pgfpoint{\pgf@xb}{\pgf@yb}}

    \slitTwoSouth \pgf@xa=\pgf@x \pgf@ya=\pgf@y
    \south \pgf@xb=\pgf@x \pgf@yb=\pgf@y
    \pgfpathmoveto{\pgfpoint{\pgf@xa}{\pgf@ya}}
    \pgfpathlineto{\pgfpoint{\pgf@xb}{\pgf@yb}}
  }
}



%%%%%%%%%%%%%%%%%%%%%%%%%%%%%%%%%%%%%%%%%%%%%%%%%%%%%%%%%%%%%%%%%%%%%%%%%%%%%%%%
% Shape [mirror]
%%%%%%%%%%%%%%%%%%%%%%%%%%%%%%%%%%%%%%%%%%%%%%%%%%%%%%%%%%%%%%%%%%%%%%%%%%%%%%%%
% keys
% - mirror decoration separation
% - mirror decoration amplitude
\pgfkeys{/tikz/optics/.cd,
  object height/.initial=2cm,
  mirror decoration separation/.initial=0.15cm,
  mirror decoration amplitude/.initial=0.125cm,
}

\pgfdeclareshape{mirror}
{
  \savedanchor{\center}{
    \pgfpointorigin
  }
  \anchor{center}{\center}

  \savedmacro\objectHeight{%
    \edef\objectHeight{\pgfkeysvalueof{/tikz/optics/object height}}%
  }

  \savedanchor{\north}{
    \pgf@x=0cm%
    \pgf@y=\objectHeight%
    \pgf@y=0.5\pgf@y%
  }
  \anchor{north}{\north}

  \savedanchor{\south}{
    \pgf@x=0cm%
    \pgf@y=\objectHeight%
    \pgf@y=-0.5\pgf@y%
  }
  \anchor{south}{\south}

  \anchor{east}{\center}
  \anchor{west}{\center}

  % bof
  \inheritanchorborder[from=thin optics element]

  \backgroundpath
  {
    \north \pgf@xa=\pgf@x \pgf@ya=\pgf@y
    \south \pgf@xb=\pgf@x \pgf@yb=\pgf@y
    % We use a border decoration to make a mirror (hachures).
    % First set the decoration parameters :
    % - decoration angle
    % l'angle est 45-180 et on multiplie l'amplitude par -1 pour que la décoration commence et finisse par un trait (c'est un peu louche mais bon)
    \def\pgfdecorationsegmentangle{45-180}%
    % - decoration step length
    \pgfmathparse{\pgfkeysvalueof{/tikz/optics/mirror decoration separation}}
    \ifpgfmathunitsdeclared%
      \pgfmathsetlengthmacro{\pgfdecorationsegmentlength}{\pgfkeysvalueof{/tikz/optics/mirror decoration separation}}%
    \else%
      \pgfmathsetlengthmacro{\pgfdecorationsegmentlength}{\pgfkeysvalueof{/tikz/optics/mirror decoration separation}*\pgfkeysvalueof{/tikz/optics/object height}}%
    \fi%
    % magouillons pour que \pgfdecorationsegmentlength soit un multiple de la longueur ...
    \pgfmathsetmacro\initialstep{\pgfdecorationsegmentlength} 
    \pgfmathsetmacro\totallength{\pgfkeysvalueof{/tikz/optics/object height}} 
    \pgfmathsetmacro\newstep{\totallength/floor(\totallength/\initialstep)} 
    \pgfmathsetlengthmacro{\pgfdecorationsegmentlength}{\newstep}
    % fin magouille
    % - decoration amplitude
    % on multiplie par -1 pour que ça aille dans le bon sens (pour éviter des problèmes louches)
    \pgfmathparse{\pgfkeysvalueof{/tikz/optics/mirror decoration amplitude}}
    \ifpgfmathunitsdeclared%
      \pgfmathsetlengthmacro{\pgfdecorationsegmentamplitude}{-1*\pgfkeysvalueof{/tikz/optics/mirror decoration amplitude}}%
    \else%
      \pgfmathsetlengthmacro{\pgfdecorationsegmentamplitude}{-1*\pgfkeysvalueof{/tikz/optics/mirror decoration amplitude}*\pgfkeysvalueof{/tikz/optics/object height}}%
    \fi%
    % Use decoration.
    \pgfpathmoveto{\pgfpoint{\pgf@xa}{\pgf@ya}}
    \pgfpathlineto{\pgfpoint{\pgf@xb}{\pgf@yb}}
    \pgfdecoratecurrentpath{border} %
    % dessin du miroir (trait)
    \pgfpathmoveto{\pgfpoint{\pgf@xa}{\pgf@ya}}
    \pgfpathlineto{\pgfpoint{\pgf@xb}{\pgf@yb}}
  }
}

%%%%%%%%%%%%%%%%%%%%%%%%%%%%%%%%%%%%%%%%%%%%%%%%%%%%%%%%%%%%%%%%%%%%%%%%%%%%%%%%
% Shape [spherical mirror]
%%%%%%%%%%%%%%%%%%%%%%%%%%%%%%%%%%%%%%%%%%%%%%%%%%%%%%%%%%%%%%%%%%%%%%%%%%%%%%%%
% keys :
% - spherical mirror angle : the aperture angle of the mirror
% - spherical mirror type (concave or convex)
% - spherical mirror orientation (ltr or rtl)
% The properties of the decoration are controled by the same parameters as in [mirror].
% Height is controled by |object height|, as usual.
\pgfkeys{/tikz/optics/spherical mirror angle/.initial=150}

\newif\iftikz@optics@sphericalmirror@concave
\tikz@optics@sphericalmirror@concavetrue
\pgfkeys{/tikz/optics/.cd,
  spherical mirror type/.is choice,
  spherical mirror type/concave/.code={\tikz@optics@sphericalmirror@concavetrue},
  spherical mirror type/convex/.code={\tikz@optics@sphericalmirror@concavefalse}
}

\newif\iftikz@optics@sphericalmirror@ltr
\tikz@optics@sphericalmirror@ltrtrue
\pgfkeys{/tikz/optics/.cd,
  spherical mirror orientation/.is choice,
  spherical mirror orientation/ltr/.code={\tikz@optics@sphericalmirror@ltrtrue},
  spherical mirror orientation/rtl/.code={\tikz@optics@sphericalmirror@ltrfalse}
}

% shortcuts (styles) : [concave mirror] and [convex mirror]
\pgfkeys{/tikz/optics/.cd,
      concave mirror/.style={optics, spherical mirror, /tikz/optics/spherical mirror type=concave},
      convex mirror/.style={optics, spherical mirror, /tikz/optics/spherical mirror type=convex},
}

\pgfdeclareshape{spherical mirror}{%
  \savedmacro\installsphericalmirrorparameters{%
    %
    % Define a \centerpoint
    %
    \pgfextract@process\centerpoint{%
      \pgfpointorigin
    }%
    %
    % Define height
    %
    \pgfmathsetlengthmacro\height{\pgfkeysvalueof{/tikz/optics/object height}}%
    %
    % Function to define angle from radius
    % use e.g. [spherical mirror angle=from_radius(2cm)] instead of e.g. [spherical mirror angle=90]
    % This can seem somewhat ridiculous as we will undo this when calculating \radius, however it helps providing a simpler API.
    % This function has to be defined before parsing \pgfkeysvalueof{/tikz/optics/spherical mirror angle}.
    %
    \pgfmathdeclarefunction{from_radius}{1}{
      \begingroup
      \pgfmathparse{notless(2*#1,\height)}
      \ifnum\pgfmathresult=0
        \opticserror{(in /tikz/optics/spherical mirror angle=from_radius(R)) : for a spherical mirror, the radius R cannot be smaller than half the height </tikz/optics/object height>. Set a bigger radius of a smaller height.}
      \fi
      \newdimen\angle
      \pgfmathsetlength\angle{2*asin(\height/(2*#1))}
      \pgf@x=\angle
      \pgfmathreturn\pgf@x
      \endgroup
    }
    %
    % Define angle
    %
    \pgfmathsetmacro\angle{\pgfkeysvalueof{/tikz/optics/spherical mirror angle}}
    %
    % Compute radius from height and angle
    %
    \pgfmathsetlengthmacro\radius{\height/(2*sin(\angle/2))}
    %
    % Half of the sector angle is more useful.
    %
    \pgfmathmod{\angle}{360}%
    \ifdim\pgfmathresult pt<0pt\relax%
      \pgfmathadd@{\pgfmathresult}{360}%
    \fi%
    \let\angle\pgfmathresult%
    \pgfmathdivide@{\pgfmathresult}{2}%
    \let\halfangle\pgfmathresult%
    %
    % Get the start and end angles of the arc.
    %
    \iftikz@optics@sphericalmirror@concave
      \iftikz@optics@sphericalmirror@ltr
        \pgfmathsetmacro\startangle{-\halfangle}
        \pgfmathsetmacro\endangle{+\halfangle}
      \else
        \pgfmathsetmacro\startangle{180-\halfangle}
        \pgfmathsetmacro\endangle{180+\halfangle}
      \fi
      \else
      \iftikz@optics@sphericalmirror@ltr
        \pgfmathsetmacro\startangle{180-\halfangle}
        \pgfmathsetmacro\endangle{180+\halfangle}
      \else
        \pgfmathsetmacro\startangle{-\halfangle}
        \pgfmathsetmacro\endangle{+\halfangle}
      \fi
    \fi
    % 
    % Calculate R cos(angle/2) and R sin(angle/2)
    % 
    \pgfmathabs@{\halfangle}%
    \pgfmathcos@{\pgfmathresult}%
    \let\coshalfangle\pgfmathresult%
    \pgfmathabs@{\halfangle}%
    \pgfmathsin@{\pgfmathresult}%
    \let\sinhalfangle\pgfmathresult%
    \pgfmathsetlength\pgf@xa{\radius*\coshalfangle}
    \edef\rcoshalfangle{\the\pgf@xa}%
    \pgfmathsetlength\pgf@xa{\radius*\sinhalfangle}
    \edef\rsinhalfangle{\the\pgf@xa}%
    %
    % Calculate the arc coordinates
    %
    \pgfextract@process\arcstart{%
      \pgfqpointpolar{\startangle}{\radius}%
      \pgf@xa\pgf@x%
      \pgf@ya\pgf@y%
      \centerpoint%
      \advance\pgf@x\pgf@xa%
      \advance\pgf@y\pgf@ya%
    }%
    \pgfextract@process\arcend{%
      \pgfqpointpolar{\endangle}{\radius}%
      \pgf@xa\pgf@x%
      \pgf@ya\pgf@y%
      \centerpoint%
      \advance\pgf@x\pgf@xa%
      \advance\pgf@y\pgf@ya%
    }%
    \def\convexrtlsetx#1#2{
      \iftikz@optics@sphericalmirror@concave%
        \iftikz@optics@sphericalmirror@ltr%
          \advance\pgf@x by #1
        \else% % => rtl
          \advance\pgf@x by #2
        \fi%
      \else% % => convex
        \iftikz@optics@sphericalmirror@ltr%
          \advance\pgf@x by #2
        \else% % => rtl
          \advance\pgf@x by #1
        \fi%
      \fi%
    }
    \def\convexrtlinvert{%
      \iftikz@optics@sphericalmirror@concave%
        \iftikz@optics@sphericalmirror@ltr%
          %nothing
        \else% % => rtl
          \pgf@x=-\pgf@x%
        \fi%
      \else% % => convex
        \iftikz@optics@sphericalmirror@ltr%
          \pgf@x=-\pgf@x%
        \else% % => rtl
          %nothing
        \fi%
      \fi%
    }%
    %
    % Save everything. 
    % "NB \addtosavedmacro is currently experimental. May get changed." d'après le code où je l'ai piqué
    %
    \addtosavedmacro{\radius}%
    %
    \addtosavedmacro{\rcoshalfangle}%
    \addtosavedmacro{\rsinhalfangle}%
    %
    \addtosavedmacro{\endangle}%
    \addtosavedmacro{\startangle}%
    %
    \addtosavedmacro{\centerpoint}%
    \addtosavedmacro{\arcstart}%
    \addtosavedmacro{\arcend}%
     \addtosavedmacro{\convexrtlinvert}%
    %
  }%
  %
  % Define anchors
  %
  \savedanchor\mirrorcenterpoint{%
    \pgfpointorigin
  }%
  \savedanchor\centerpoint{%
    \pgfpointorigin%
    \advance\pgf@x by \radius%
    \advance\pgf@x by \rcoshalfangle%
    \iftikz@optics@sphericalmirror@concave%
      \iftikz@optics@sphericalmirror@ltr%
        %nothing
      \else% % => rtl
        \pgf@x=-\pgf@x%
      \fi%
    \else% % => convex
      \iftikz@optics@sphericalmirror@ltr%
        \pgf@x=-\pgf@x%
      \else% % => rtl
        %nothing
      \fi%
    \fi%
    \divide\pgf@x by 2%
  }%
  \savedanchor\focalpoint{%
    \pgfpointorigin%
    \pgf@xa=\radius%
    \advance\pgf@x by .5\pgf@xa%
    \iftikz@optics@sphericalmirror@concave%
      \iftikz@optics@sphericalmirror@ltr%
        %nothing
      \else% % => rtl
        \pgf@x=-\pgf@x%
      \fi%
    \else% % => convex
      \iftikz@optics@sphericalmirror@ltr%
        \pgf@x=-\pgf@x%
      \else% % => rtl
        %nothing
      \fi%
    \fi%
  }%
  \savedanchor\north{%
    \pgfpointorigin
    \pgf@xa=0pt
    \advance\pgf@xa by \rcoshalfangle
    \advance\pgf@xa by \radius
    \divide\pgf@xa by 2
    \advance\pgf@x by \pgf@xa
    \advance\pgf@y by \rsinhalfangle
    \convexrtlinvert
  }%
  \savedanchor\arccenter{%
    \centerpoint
    \advance\pgf@x by \radius
    \convexrtlinvert
  }
  \savedanchor\south{%
    \pgfpointorigin
    \pgf@xa=0pt
    \advance\pgf@xa by \rcoshalfangle
    \advance\pgf@xa by \radius
    \divide\pgf@xa by 2
    \advance\pgf@x by \pgf@xa
    \advance\pgf@y by -\rsinhalfangle
    \convexrtlinvert
  }
  \savedanchor\east{%
    \pgfpointorigin
    \convexrtlsetx{\radius}{\rcoshalfangle}
    \convexrtlinvert
  }
  \savedanchor\west{%
    \pgfpointorigin
    \convexrtlsetx{\rcoshalfangle}{\radius}
    \convexrtlinvert
  }
  \savedanchor\northwest{%
    \pgfpointorigin
    \convexrtlsetx{\rcoshalfangle}{\radius}
    \advance\pgf@y by \rsinhalfangle
    \convexrtlinvert
  }
  \savedanchor\southwest{%
    \pgfpointorigin
    \convexrtlsetx{\rcoshalfangle}{\radius}
    \advance\pgf@y by -\rsinhalfangle
    \convexrtlinvert
  }
  \savedanchor\northeast{%
    \pgfpointorigin
    \convexrtlsetx{\radius}{\rcoshalfangle}
    \advance\pgf@y by \rsinhalfangle
    \convexrtlinvert
  }
  \savedanchor\southeast{%
    \pgfpointorigin
    \convexrtlsetx{\radius}{\rcoshalfangle}
    \advance\pgf@y by -\rsinhalfangle
    \convexrtlinvert
  }
  \anchor{arc start}{%
    \installsphericalmirrorparameters%
    \arcstart%
  }
  \anchor{arc end}{%
    \installsphericalmirrorparameters%
    \arcend%
  }
  \anchor{focal point}{%
    \installsphericalmirrorparameters%
    \focalpoint
  }
  \anchor{focus}{%
    \installsphericalmirrorparameters%
    \focalpoint
  }
  \anchor{mirror center}{\mirrorcenterpoint}  
  \anchor{center}{\centerpoint} 
  \anchor{arc center}{\arccenter}
  \anchor{north}{\north}%
  \anchor{south}{\south}%
  \anchor{east}{\east}%
  \anchor{west}{\west}%
  \anchor{north west}{\northwest}%
  \anchor{south west}{\southwest}%
  \anchor{north east}{\northeast}%
  \anchor{south east}{\southeast}%
  %
  % Draw backgroundpath
  %
  \backgroundpath{%
    \installsphericalmirrorparameters%
    % We use a border decoration to make a mirror.
    % First set the decoration parameters :
    % - decoration angle
    \iftikz@optics@sphericalmirror@concave
      \def\pgfdecorationsegmentangle{45}%
    \else
      \def\pgfdecorationsegmentangle{-90-45}%
    \fi
    % - decoration step length
    \pgfmathparse{\pgfkeysvalueof{/tikz/optics/mirror decoration separation}}
    \ifpgfmathunitsdeclared%
      \pgfmathsetlengthmacro{\pgfdecorationsegmentlength}{\pgfkeysvalueof{/tikz/optics/mirror decoration separation}}%
    \else%
      \pgfmathsetlengthmacro{\pgfdecorationsegmentlength}{\pgfkeysvalueof{/tikz/optics/mirror decoration separation}*\pgfkeysvalueof{/tikz/optics/object height}}%
    \fi%
    % magouillons pour que \pgfdecorationsegmentlength soit un multiple de la longueur ...
    \pgfmathsetmacro\initialstep{\pgfdecorationsegmentlength} 
    \pgfmathsetmacro\totallength{(2*pi/360)*\angle*\radius} 
    \pgfmathsetmacro\newstep{\totallength/floor(\totallength/\initialstep)} 
    \pgfmathsetlengthmacro{\pgfdecorationsegmentlength}{\newstep}
    % fin magouille
    % - decoration amplitude
    % on multiplie par -1 pour que ça aille dans le bon sens sans devoir modifier l'angle (pour éviter des problèmes louches)
    \pgfmathparse{\pgfkeysvalueof{/tikz/optics/mirror decoration amplitude}}
    \ifpgfmathunitsdeclared%
      \pgfmathsetlengthmacro{\pgfdecorationsegmentamplitude}{-1*\pgfkeysvalueof{/tikz/optics/mirror decoration amplitude}}%
    \else%
      \pgfmathsetlengthmacro{\pgfdecorationsegmentamplitude}{-1*\pgfkeysvalueof{/tikz/optics/mirror decoration amplitude}*\pgfkeysvalueof{/tikz/optics/object height}}%
    \fi%
    % Now use decoration.
    % Draw decoration of path : an arc of radius \radius from \arcstart to \arcend
    \pgfpathmoveto{\arcstart}%
    \pgfpatharc{\startangle}{\endangle}{\radius}%
    \pgfdecoratecurrentpath{border} %
    % Now draw the path.
    \pgfpathmoveto{\arcstart}%
    \pgfpatharc{\startangle}{\endangle}{\radius}%
  }
  %
  % Anchor border
  % This is needed for anchors .<angle> (like mirror.0, mirror.90, etc.) to work.
  %
  \anchorborder{%
    % Save x and y
    \edef\externalx{\the\pgf@x}%
    \edef\externaly{\the\pgf@y}%
    \installsphericalmirrorparameters%
    % Use circular border
    \pgfpointborderellipse{ \pgfpoint{\externalx}{\externaly} }{ \pgfpoint{\radius}{\radius} }%
  }%
}



%%%%%%%%%%%%%%%%%%%%%%%%%%%%%%%%%%%%%%%%%%%%%%%%%%%%%%%%%%%%%%%%%%%%%%%%%%%%%%%%
% Shape [thick optics element]
%%%%%%%%%%%%%%%%%%%%%%%%%%%%%%%%%%%%%%%%%%%%%%%%%%%%%%%%%%%%%%%%%%%%%%%%%%%%%%%%
% keys :
% - object aspect ratio (alias : object width)
% ar = largeur/hauteur => largeur = ar * hauteur
\pgfkeys{/tikz/optics/.cd,
  object aspect ratio/.initial=0.2,
  object width/.style={/tikz/optics/object aspect ratio=#1}
}

\pgfdeclareshape{thick optics element}
{
  \savedanchor{\center}{
    \pgfpointorigin
  }
  \anchor{center}{\center}

  \savedmacro\objectHeight{%
    \edef\objectHeight{\pgfkeysvalueof{/tikz/optics/object height}}%
  }

  \savedmacro\objectWidth{%
    \pgfmathparse{\pgfkeysvalueof{/tikz/optics/object aspect ratio}}
    \ifpgfmathunitsdeclared%
      \pgfmathsetlengthmacro{\objectWidth}{\pgfkeysvalueof{/tikz/optics/object aspect ratio}}%
    \else%
      \pgfmathsetlengthmacro{\objectWidth}{\pgfkeysvalueof{/tikz/optics/object aspect ratio}*\pgfkeysvalueof{/tikz/optics/object height}}%
    \fi%
  }

  \savedanchor{\northeast}{
    \pgf@x=\objectWidth%
    \pgf@y=\objectHeight%
    \pgf@y=0.5\pgf@y%
    \pgf@x=0.5\pgf@x%
  }
  \anchor{north east}{\northeast}

  \savedanchor{\southwest}{
    \pgf@x=\objectWidth%
    \pgf@y=\objectHeight%
    \pgf@y=-0.5\pgf@y%
    \pgf@x=-0.5\pgf@x%
  }
  \anchor{south west}{\southwest}

  \anchor{north}{
    \center \pgf@xa=\pgf@x \pgf@ya=\pgf@y
    \northeast \pgf@xb=\pgf@x \pgf@yb=\pgf@y
    \pgf@x=\pgf@xa
    \pgf@y=\pgf@yb
  }

  \anchor{south}{
    \center \pgf@xa=\pgf@x \pgf@ya=\pgf@y
    \southwest \pgf@xb=\pgf@x \pgf@yb=\pgf@y
    \pgf@x=\pgf@xa
    \pgf@y=\pgf@yb
  }

  \anchor{east}{
    \center \pgf@xa=\pgf@x \pgf@ya=\pgf@y
    \northeast \pgf@xb=\pgf@x \pgf@yb=\pgf@y
    \pgf@x=\pgf@xb
    \pgf@y=\pgf@ya
  }

  \anchor{west}{
    \center \pgf@xa=\pgf@x \pgf@ya=\pgf@y
    \southwest \pgf@xb=\pgf@x \pgf@yb=\pgf@y
    \pgf@x=\pgf@xb
    \pgf@y=\pgf@ya
  }

  \anchor{north west}{
    \northeast \pgf@xa=\pgf@x \pgf@ya=\pgf@y
    \southwest \pgf@xb=\pgf@x \pgf@yb=\pgf@y
    \pgf@x=\pgf@xb
    \pgf@y=\pgf@ya
  }

  \anchor{south east}{
    \northeast \pgf@xa=\pgf@x \pgf@ya=\pgf@y
    \southwest \pgf@xb=\pgf@x \pgf@yb=\pgf@y
    \pgf@x=\pgf@xa
    \pgf@y=\pgf@yb
  }

  \inheritanchorborder[from=rectangle]

  \backgroundpath
  {
    % rectangle
    \pgfpathrectanglecorners{\northeast}{\southwest}
  }
}



%%%%%%%%%%%%%%%%%%%%%%%%%%%%%%%%%%%%%%%%%%%%%%%%%%%%%%%%%%%%%%%%%%%%%%%%%%%%%%%%
% Shape [polarizer]
%%%%%%%%%%%%%%%%%%%%%%%%%%%%%%%%%%%%%%%%%%%%%%%%%%%%%%%%%%%%%%%%%%%%%%%%%%%%%%%%
\pgfdeclareshape{polarizer}
{
  \savedanchor{\center}{
    \pgfpointorigin
  }
  \anchor{center}{\center}

  \savedmacro\objectHeight{%
    \edef\objectHeight{\pgfkeysvalueof{/tikz/optics/object height}}%
  }

  \savedmacro\objectWidth{%
    \pgfmathparse{\pgfkeysvalueof{/tikz/optics/object aspect ratio}}
    \ifpgfmathunitsdeclared%
      \pgfmathsetlengthmacro{\objectWidth}{\pgfkeysvalueof{/tikz/optics/object aspect ratio}}%
    \else%
      \pgfmathsetlengthmacro{\objectWidth}{\pgfkeysvalueof{/tikz/optics/object aspect ratio}*\pgfkeysvalueof{/tikz/optics/object height}}%
    \fi%
  }

  \savedanchor{\northeast}{
    \pgf@x=\objectWidth%
    \pgf@y=\objectHeight%
    \pgf@y=0.5\pgf@y%
    \pgf@x=0.5\pgf@x%
  }
  \anchor{north east}{\northeast}

  \savedanchor{\southwest}{
    \pgf@x=\objectWidth%
    \pgf@y=\objectHeight%
    \pgf@y=-0.5\pgf@y%
    \pgf@x=-0.5\pgf@x%
  }
  \anchor{south west}{\southwest}

  \anchor{north}{
    \center \pgf@xa=\pgf@x \pgf@ya=\pgf@y
    \northeast \pgf@xb=\pgf@x \pgf@yb=\pgf@y
    \pgf@x=\pgf@xa
    \pgf@y=\pgf@yb
  }

  \anchor{south}{
    \center \pgf@xa=\pgf@x \pgf@ya=\pgf@y
    \southwest \pgf@xb=\pgf@x \pgf@yb=\pgf@y
    \pgf@x=\pgf@xa
    \pgf@y=\pgf@yb
  }

  \anchor{east}{
    \center \pgf@xa=\pgf@x \pgf@ya=\pgf@y
    \northeast \pgf@xb=\pgf@x \pgf@yb=\pgf@y
    \pgf@x=\pgf@xb
    \pgf@y=\pgf@ya
  }

  \anchor{west}{
    \center \pgf@xa=\pgf@x \pgf@ya=\pgf@y
    \southwest \pgf@xb=\pgf@x \pgf@yb=\pgf@y
    \pgf@x=\pgf@xb
    \pgf@y=\pgf@ya
  }

  \anchor{north west}{
    \northeast \pgf@xa=\pgf@x \pgf@ya=\pgf@y
    \southwest \pgf@xb=\pgf@x \pgf@yb=\pgf@y
    \pgf@x=\pgf@xb
    \pgf@y=\pgf@ya
  }

  \anchor{south east}{
    \northeast \pgf@xa=\pgf@x \pgf@ya=\pgf@y
    \southwest \pgf@xb=\pgf@x \pgf@yb=\pgf@y
    \pgf@x=\pgf@xa
    \pgf@y=\pgf@yb
  }

  \inheritanchorborder[from=rectangle]

  \backgroundpath
  {
    % rectangle
    \pgfpathrectanglecorners{\northeast}{\southwest}

    % diagonale du polariseur
    \northeast \pgf@xa=\pgf@x \pgf@ya=\pgf@y
    \southwest \pgf@xb=\pgf@x \pgf@yb=\pgf@y
    \pgfpathmoveto{\pgfpoint{\pgf@xa}{\pgf@ya}}
    \pgfpathlineto{\pgfpoint{\pgf@xb}{\pgf@yb}}
  }
}

%%%%%%%%%%%%%%%%%%%%%%%%%%%%%%%%%%%%%%%%%%%%%%%%%%%%%%%%%%%%%%%%%%%%%%%%%%%%%%%%
% Shape [double amici prism]
%%%%%%%%%%%%%%%%%%%%%%%%%%%%%%%%%%%%%%%%%%%%%%%%%%%%%%%%%%%%%%%%%%%%%%%%%%%%%%%%
% keys :
% - prism height
% - prism apex angle
\pgfkeys{/tikz/optics/.cd,
  prism height/.initial=1.5cm,
  prism apex angle/.initial=60,
}
% Idea : we get 
% apexAngle
% prismHeight
% demiPrismWidth = tan(apexAngle/2)*prismHeight
%
% so we compute
%
% C = (0,0)
% NE = (demiPrismWidth,0.5*prismHeight)
% SW = (-2*demiPrismWidth,-0.5*prismHeight)
% NW = (-demiPrismWidth,0.5*prismHeight)
% SE = (2*demiPrismWidth,-0.5*prismHeight)
%
% N = (0,0.5*prismHeight)
% S = (0,-0.5*prismHeight)
% E = (3*0.5*demiPrismWidth,0)
% W = (-3.0.5*demiPrismWidth,0)
\pgfdeclareshape{double amici prism}
{
  \savedanchor{\center}{
    \pgfpointorigin
  }
  \anchor{center}{\center}

  \savedmacro\prismHeight{%
    \edef\prismHeight{\pgfkeysvalueof{/tikz/optics/prism height}}%
  }

  \savedmacro\apexAngle{%
      \pgfmathsetlengthmacro{\apexAngle}{\pgfkeysvalueof{/tikz/optics/prism apex angle}}
  }

  \savedmacro\demiPrismWidth{%
      \pgfmathsetlengthmacro{\demiPrismWidth}{tan(0.5*\pgfkeysvalueof{/tikz/optics/prism apex angle})*\pgfkeysvalueof{/tikz/optics/prism height}}
  }

  \savedanchor{\northeast}{
    \pgf@x=\demiPrismWidth%
    \pgf@y=\prismHeight%
    \pgf@y=0.5\pgf@y%
  }
  \anchor{north east}{\northeast}

  \savedanchor{\southwest}{
    \pgf@x=\demiPrismWidth%
    \pgf@y=\prismHeight%
    \pgf@x=-2\pgf@x%
    \pgf@y=-0.5\pgf@y%
  }
  \anchor{south west}{\southwest}

  \savedanchor{\northwest}{
    \pgf@x=\demiPrismWidth%
    \pgf@y=\prismHeight%
    \pgf@x=-\pgf@x%
    \pgf@y=0.5\pgf@y%
  }
  \anchor{north west}{\northwest}

  \savedanchor{\southeast}{
    \pgf@x=\demiPrismWidth%
    \pgf@y=\prismHeight%
    \pgf@x=2\pgf@x%
    \pgf@y=-0.5\pgf@y%
  }
  \anchor{south east}{\southeast}

  \anchor{north}{
    \center \pgf@xa=\pgf@x \pgf@ya=\pgf@y
    \northeast \pgf@xb=\pgf@x \pgf@yb=\pgf@y
    \pgf@x=\pgf@xa
    \pgf@y=\pgf@yb
  }

  \savedanchor{\south}{
    \pgfpointorigin
    \pgf@y=\prismHeight%
    \pgf@y=-0.5\pgf@y%
  }
  \anchor{south}{\south}

  \anchor{east}{
    \southeast \pgf@xa=\pgf@x \pgf@ya=\pgf@y
    \northeast \pgf@xb=\pgf@x \pgf@yb=\pgf@y
    \pgf@x=\pgf@xa
    \advance\pgf@x by\pgf@xb
    \pgf@x=0.5\pgf@x
    \pgf@y=\pgf@ya
    \advance\pgf@y by\pgf@yb
    \pgf@y=0.5\pgf@y
  }

  \anchor{west}{
    \southwest \pgf@xa=\pgf@x \pgf@ya=\pgf@y
    \northwest \pgf@xb=\pgf@x \pgf@yb=\pgf@y
    \pgf@x=\pgf@xa
    \advance\pgf@x by\pgf@xb
    \pgf@x=0.5\pgf@x
    \pgf@y=\pgf@ya
    \advance\pgf@y by\pgf@yb
    \pgf@y=0.5\pgf@y
  }

   \inheritanchorborder[from=rectangle]

  \backgroundpath
  {
    \northwest
    \pgfpathmoveto{\pgfpoint{\pgf@x}{\pgf@y}}
    \southwest
    \pgfpathlineto{\pgfpoint{\pgf@x}{\pgf@y}}
    \southeast
    \pgfpathlineto{\pgfpoint{\pgf@x}{\pgf@y}}
    \northeast
    \pgfpathlineto{\pgfpoint{\pgf@x}{\pgf@y}}
    \pgfpathclose
    %
    % FIXME : ceci devrait s'appliquer seulement au triangle intérieur -> un fgpath ou assimilé
    \pgfsetbeveljoin
    \northwest
    \pgfpathmoveto{\pgfpoint{\pgf@x}{\pgf@y}}
    \south
    \pgfpathlineto{\pgfpoint{\pgf@x}{\pgf@y}}
    \northeast
    \pgfpathlineto{\pgfpoint{\pgf@x}{\pgf@y}}
  }
}






%%%%%%%%%%%%%%%%%%%%%%%%%%%%%%%%%%%%%%%%%%%%%%%%%%%%%%%%%%%%%%%%%%%%%%%%%%%%%%%%
% Shape [generic optics io]
%%%%%%%%%%%%%%%%%%%%%%%%%%%%%%%%%%%%%%%%%%%%%%%%%%%%%%%%%%%%%%%%%%%%%%%%%%%%%%%%
% keys :
% - io body height : height of the body of the io body
% - io body aspect ratio (alias io body width) : width/height of the body of the io body
% - io aperture width : width of the output device (condenser, etc.) [in units of io body height]
% - io aperture height : height of the output device (condenser, etc.) [in units of io body height]
% - io aperture shift : vertical shift of the output device [in units of io body height]
% - io orientation (ltr or rtl)
\pgfkeys{/tikz/optics/.cd,
  io body height/.initial=0.75cm,
  io body aspect ratio/.initial=2,
  io aperture width/.initial=0.33,
  io aperture height/.initial=0.66,
  io aperture shift/.initial=0,
}
\pgfkeys{/tikz/optics/io body width/.style={/tikz/optics/io body aspect ratio=#1}}
%
\newif\if@tikz@optics@io@ltr
\@tikz@optics@io@ltrtrue
%
\pgfkeys{/tikz/optics/io orientation/.is choice}
\pgfkeys{/tikz/optics/io orientation/ltr/.code={\@tikz@optics@io@ltrtrue}}
\pgfkeys{/tikz/optics/io orientation/rtl/.code={\@tikz@optics@io@ltrfalse}}
%
\pgfdeclareshape{generic optics io}
{
  \savedanchor{\center}{
    \pgfpointorigin
  }
  \anchor{center}{\center}

  \savedmacro\objectHeight{%
    \edef\objectHeight{\pgfkeysvalueof{/tikz/optics/io body height}}%
  }

  \savedmacro\objectWidth{%
    \pgfmathparse{\pgfkeysvalueof{/tikz/optics/io body aspect ratio}}
    \ifpgfmathunitsdeclared%
      \pgfmathsetlengthmacro{\objectWidth}{\pgfkeysvalueof{/tikz/optics/io body aspect ratio}}%
    \else%
      \pgfmathsetlengthmacro{\objectWidth}{\pgfkeysvalueof{/tikz/optics/io body aspect ratio}*\pgfkeysvalueof{/tikz/optics/io body height}}%
    \fi%
  }

  \savedmacro\outHeight{%
    \pgfmathparse{\pgfkeysvalueof{/tikz/optics/io aperture height}}
    \ifpgfmathunitsdeclared%
      \pgfmathsetlengthmacro{\outHeight}{\pgfkeysvalueof{/tikz/optics/io aperture height}}%
    \else%
      \pgfmathsetlengthmacro{\outHeight}{\pgfkeysvalueof{/tikz/optics/io aperture height}*\pgfkeysvalueof{/tikz/optics/io body height}}%
    \fi%
  }

  \savedmacro\outWidth{%
    \pgfmathparse{\pgfkeysvalueof{/tikz/optics/io aperture width}}
    \ifpgfmathunitsdeclared%
      \pgfmathsetlengthmacro{\outWidth}{\pgfkeysvalueof{/tikz/optics/io aperture width}}%
    \else%
      \pgfmathsetlengthmacro{\outWidth}{\pgfkeysvalueof{/tikz/optics/io aperture width}*\pgfkeysvalueof{/tikz/optics/io body height}}%
    \fi%
  }

  \savedmacro\outShift{%
    \pgfmathparse{\pgfkeysvalueof{/tikz/optics/io aperture shift}}
    \ifpgfmathunitsdeclared%
      \pgfmathsetlengthmacro{\outShift}{\pgfkeysvalueof{/tikz/optics/io aperture shift}}%
    \else%
      \pgfmathsetlengthmacro{\outShift}{\pgfkeysvalueof{/tikz/optics/io aperture shift}*\pgfkeysvalueof{/tikz/optics/io body height}}%
    \fi%
  }

  \savedanchor{\bodynortheast}{
    \pgf@x=\objectWidth%
    \pgf@y=\objectHeight%
    \pgf@y=0.5\pgf@y%
    \pgf@x=0.5\pgf@x%
  }
  \anchor{body north east}{\bodynortheast}

  \savedanchor{\bodysouthwest}{
    \pgf@x=\objectWidth%
    \pgf@y=\objectHeight%
    \pgf@y=-0.5\pgf@y%
    \pgf@x=-0.5\pgf@x%
  }
  \anchor{body south west}{\bodysouthwest}

  \savedanchor{\outnortheast}{   
    \if@tikz@optics@io@ltr
      % Left To Right (LTR)
      \pgf@x=\objectWidth%
      \pgf@x=0.5\pgf@x%
      \advance\pgf@x by\outWidth
    \else
      % Right To Left (RTL)
      \pgf@x=\objectWidth%
      \pgf@x=0.5\pgf@x%
      \pgf@x=-\pgf@x%
    \fi
    \pgf@y=0pt%
    \advance\pgf@y by\outHeight
    \pgf@y=0.5\pgf@y%
    \advance\pgf@y by\outShift
  }
  \anchor{aperture north east}{\outnortheast}

  \savedanchor{\outsouthwest}{
    \if@tikz@optics@io@ltr
      % Left To Right (LTR)
      \pgf@x=\objectWidth%
      \pgf@x=0.5\pgf@x%
    \else
      % Right To Left (RTL)
      \pgf@x=\objectWidth%
      \pgf@x=0.5\pgf@x%
      \advance\pgf@x by\outWidth
      \pgf@x=-\pgf@x%
    \fi
    \pgf@y=0pt%
    \advance\pgf@y by\outHeight
    \pgf@y=-0.5\pgf@y%
    \advance\pgf@y by\outShift
  }
  \anchor{aperture south west}{\outsouthwest}

  \savedanchor{\outcenter}{
    \if@tikz@optics@io@ltr
      % Left To Right (LTR)
      \pgf@x=\objectWidth%
      \advance\pgf@x by\outWidth
    \else
      % Right To Left (RTL)
      \pgf@x=\objectWidth%
      \advance\pgf@x by\outWidth
      \pgf@x=-\pgf@x%
    \fi
    \pgf@x=0.5\pgf@x%
    \pgf@y=0pt%
    \advance\pgf@y by\outShift
  }
  \anchor{aperture center}{\outcenter}
  

  \anchor{body center}{\center}

  \anchor{text}{
    \pgfpointorigin
    \advance\pgf@x by -.5\wd\pgfnodeparttextbox%
    \advance\pgf@y by -.5\ht\pgfnodeparttextbox%
    \advance\pgf@y by +.5\dp\pgfnodeparttextbox%
  }

  \anchor{body north}{
    \center \pgf@xa=\pgf@x \pgf@ya=\pgf@y
    \bodynortheast \pgf@xb=\pgf@x \pgf@yb=\pgf@y
    \pgf@x=\pgf@xa
    \pgf@y=\pgf@yb
  }
  % north = body north
  \anchor{north}{
    \center \pgf@xa=\pgf@x \pgf@ya=\pgf@y
    \bodynortheast \pgf@xb=\pgf@x \pgf@yb=\pgf@y
    \pgf@x=\pgf@xa
    \pgf@y=\pgf@yb
  }

  \anchor{body south}{
    \center \pgf@xa=\pgf@x \pgf@ya=\pgf@y
    \bodysouthwest \pgf@xb=\pgf@x \pgf@yb=\pgf@y
    \pgf@x=\pgf@xa
    \pgf@y=\pgf@yb
  }
  % south = body south
  \anchor{south}{
    \center \pgf@xa=\pgf@x \pgf@ya=\pgf@y
    \bodysouthwest \pgf@xb=\pgf@x \pgf@yb=\pgf@y
    \pgf@x=\pgf@xa
    \pgf@y=\pgf@yb
  }

  \anchor{body east}{
    \center \pgf@xa=\pgf@x \pgf@ya=\pgf@y
    \bodynortheast \pgf@xb=\pgf@x \pgf@yb=\pgf@y
    \pgf@x=\pgf@xb
    \pgf@y=\pgf@ya
  }

  \anchor{body west}{
    \center \pgf@xa=\pgf@x \pgf@ya=\pgf@y
    \bodysouthwest \pgf@xb=\pgf@x \pgf@yb=\pgf@y
    \pgf@x=\pgf@xb
    \pgf@y=\pgf@ya
  }

  \anchor{body north west}{
    \bodynortheast \pgf@xa=\pgf@x \pgf@ya=\pgf@y
    \bodysouthwest \pgf@xb=\pgf@x \pgf@yb=\pgf@y
    \pgf@x=\pgf@xb
    \pgf@y=\pgf@ya
  }

  \anchor{body south east}{
    \bodynortheast \pgf@xa=\pgf@x \pgf@ya=\pgf@y
    \bodysouthwest \pgf@xb=\pgf@x \pgf@yb=\pgf@y
    \pgf@x=\pgf@xa
    \pgf@y=\pgf@yb
  }


  \anchor{aperture north}{
    \outcenter \pgf@xa=\pgf@x \pgf@ya=\pgf@y
    \outnortheast \pgf@xb=\pgf@x \pgf@yb=\pgf@y
    \pgf@x=\pgf@xa
    \pgf@y=\pgf@yb
  }

  \anchor{aperture south}{
    \outcenter \pgf@xa=\pgf@x \pgf@ya=\pgf@y
    \outsouthwest \pgf@xb=\pgf@x \pgf@yb=\pgf@y
    \pgf@x=\pgf@xa
    \pgf@y=\pgf@yb
  }

  \anchor{aperture east}{
    \outcenter \pgf@xa=\pgf@x \pgf@ya=\pgf@y
    \outnortheast \pgf@xb=\pgf@x \pgf@yb=\pgf@y
    \pgf@x=\pgf@xb
    \pgf@y=\pgf@ya
  }

  \anchor{aperture west}{
    \outcenter \pgf@xa=\pgf@x \pgf@ya=\pgf@y
    \outsouthwest \pgf@xb=\pgf@x \pgf@yb=\pgf@y
    \pgf@x=\pgf@xb
    \pgf@y=\pgf@ya
  }

  \anchor{aperture north west}{
    \outnortheast \pgf@xa=\pgf@x \pgf@ya=\pgf@y
    \outsouthwest \pgf@xb=\pgf@x \pgf@yb=\pgf@y
    \pgf@x=\pgf@xb
    \pgf@y=\pgf@ya
  }

  \anchor{aperture south east}{
    \outnortheast \pgf@xa=\pgf@x \pgf@ya=\pgf@y
    \outsouthwest \pgf@xb=\pgf@x \pgf@yb=\pgf@y
    \pgf@x=\pgf@xa
    \pgf@y=\pgf@yb
  }

  \anchor{center}{\center}

  \savedanchor{\realeast}{
    \pgfpointorigin
    \if@tikz@optics@io@ltr
      % Left To Right (LTR)
      %ltr : aperture east (<- out north east)
      \pgf@x=\objectWidth%
      \pgf@x=0.5\pgf@x%
      \advance\pgf@x by\outWidth
      %\pgf@y=0pt%
    \else
      % Right To Left (rtl)
      \pgf@x=\objectWidth%
      \pgf@x=0.5\pgf@x%
    \fi
  }
  \anchor{east}{\realeast}

  \savedanchor{\realwest}{
    \pgfpointorigin
    \if@tikz@optics@io@ltr
      % Left To Right (LTR)
      \pgf@x=\objectWidth%
      \pgf@x=-0.5\pgf@x%
    \else
      % Right To Left (rtl)
      \pgf@x=\objectWidth%
      \pgf@x=0.5\pgf@x%
      \advance\pgf@x by\outWidth
      \pgf@x=-\pgf@x%
    \fi
  }
  \anchor{west}{\realwest}

  % this is used only for the anchorborder
  \savedanchor{\anchorbordersouthwest}{
    \if@tikz@optics@io@ltr
      % Left To Right (LTR)
      \pgf@x=\objectWidth%
      \pgf@x=-0.5\pgf@x%
    \else
      % Right To Left (rtl)
      \pgf@x=\objectWidth%
      \pgf@x=0.5\pgf@x%
      \advance\pgf@x by\outWidth
      \pgf@x=-\pgf@x%
    \fi
    \pgf@y=\objectHeight%
    \pgf@y=-0.5\pgf@y%
  }

  % this is used only for the anchorborder
  \savedanchor{\anchorbordernortheast}{
    \if@tikz@optics@io@ltr
      % Left To Right (LTR)
      \pgf@x=\objectWidth%
      \pgf@x=0.5\pgf@x%
      \advance\pgf@x by\outWidth
    \else
      % Right To Left (RTL)
      \pgf@x=\objectWidth%
      \pgf@x=0.5\pgf@x%
      %\pgf@x=-\pgf@x%
    \fi
    \pgf@y=\objectHeight%
    \pgf@y=0.5\pgf@y%
  }

  % anchorborder
  % c'est celui de rectangle, mais un peu ajusté
  \anchorborder{%
    \pgf@xb=\pgf@x% xb/yb is target
    \pgf@yb=\pgf@y%
    \anchorbordersouthwest%
    \pgf@xa=\pgf@x% xa/ya is se
    \pgf@ya=\pgf@y%
    \anchorbordernortheast%
    \advance\pgf@x by-\pgf@xa%
    \advance\pgf@y by-\pgf@ya%
    \pgf@xc=.5\pgf@x% x/y is half width/height
    \pgf@yc=.5\pgf@y%
    \advance\pgf@xa by\pgf@xc% xa/ya becomes center
    \advance\pgf@ya by\pgf@yc%
    \edef\pgf@marshal{%
      \noexpand\pgfpointborderrectangle
      {\noexpand\pgfqpoint{\the\pgf@xb}{\the\pgf@yb}}
      {\noexpand\pgfqpoint{\the\pgf@xc}{\the\pgf@yc}}%
    }%
    \pgf@process{\pgf@marshal}%
    % \advance\pgf@x by\pgf@xa%
    % \advance\pgf@y by\pgf@ya%
  }

  \backgroundpath
  {
    % corps
    \pgfpathrectanglecorners{\bodynortheast}{\bodysouthwest}
    % out
    \pgfpathrectanglecorners{\outnortheast}{\outsouthwest}

    %\pgfusepath{draw}
    % il ne faut PAS mettre ça pour pouvoir utiliser des styles correctement après (genre double)
  }
}


%%%%%%%%%%%%%%%%%%%%%%%%%%%%%%%%%%%%%%%%%%%%%%%%%%%%%%%%%%%%%%%%%%%%%%%%%%%%%%%%
%%%%%%%%%%%%%%%%%%%%%%%%%%%%%%%%%%%%%%%%%%%%%%%%%%%%%%%%%%%%%%%%%%%%%%%%%%%%%%%%
% Sensors
%%%%%%%%%%%%%%%%%%%%%%%%%%%%%%%%%%%%%%%%%%%%%%%%%%%%%%%%%%%%%%%%%%%%%%%%%%%%%%%%
%%%%%%%%%%%%%%%%%%%%%%%%%%%%%%%%%%%%%%%%%%%%%%%%%%%%%%%%%%%%%%%%%%%%%%%%%%%%%%%%


%%%%%%%%%%%%%%%%%%%%%%%%%%%%%%%%%%%%%%%%%%%%%%%%%%%%%%%%%%%%%%%%%%%%%%%%%%%%%%%%
% Shape [sensor line]
%%%%%%%%%%%%%%%%%%%%%%%%%%%%%%%%%%%%%%%%%%%%%%%%%%%%%%%%%%%%%%%%%%%%%%%%%%%%%%%%
% Sensor line.
%
% This defines anchors center, north, south, east, west, north east
% north west, south east, south west, as well as
% pixel <i> <subanchor> with <subanchor> is any of the former.
% ça va être horrible à faire
%
% keys :
% - sensor line height
% - sensor line aspect ratio (largeur en unités de la hauteur du capteur)
% - sensor line pixel number
% - sensor line pixel width (en unités de la largeur du capteur)
% - sensor line inner ysep (en unités de la hauteur du capteur)
\pgfkeys{/tikz/optics/.cd,
  sensor line height/.initial=2cm,
  sensor line aspect ratio/.initial=0.2,
  sensor line pixel number/.initial=5,
  sensor line pixel width/.initial=0.4,
  sensor line inner ysep/.initial=0.05, % entre le bord et les capteurs
}

\pgfdeclareshape{sensor line}
{
  \savedanchor{\center}{
    \pgfpointorigin
  }
  \anchor{center}{\center}

  \savedmacro\objectHeight{%
    \edef\objectHeight{\pgfkeysvalueof{/tikz/optics/sensor line height}}%
  }

  \savedmacro\objectWidth{%
    \pgfmathsetlengthmacro{\objectWidth}{\pgfkeysvalueof{/tikz/optics/sensor line aspect ratio}*\pgfkeysvalueof{/tikz/optics/sensor line height}}
  }

  \savedmacro\pixelNumber{%
    \pgfmathtruncatemacro\pixelNumber{\pgfkeysvalueof{/tikz/optics/sensor line pixel number}}%
  }

  \savedmacro\innerysep{%
    \pgfmathparse{\pgfkeysvalueof{/tikz/optics/sensor line inner ysep}}
    \ifpgfmathunitsdeclared%
      \pgfmathsetlengthmacro{\innerysep}{\pgfkeysvalueof{/tikz/optics/sensor line inner ysep}}%
    \else%
      \pgfmathsetlengthmacro{\innerysep}{\pgfkeysvalueof{/tikz/optics/sensor line inner ysep}*\pgfkeysvalueof{/tikz/optics/sensor line height}}%
    \fi%
  }
  
  \savedmacro\pixelWidth{%
    \pgfmathparse{\pgfkeysvalueof{/tikz/optics/sensor line pixel width}}
    \ifpgfmathunitsdeclared%
      \pgfmathsetlengthmacro{\pixelWidth}{\pgfkeysvalueof{/tikz/optics/sensor line pixel width}}%
    \else%
      \pgfmathsetlengthmacro{\pixelWidth}{\pgfkeysvalueof{/tikz/optics/sensor line pixel width}*\objectWidth}%
    \fi%
  }

  \savedmacro\pixelHeight{%
    \pgfmathparse{(\objectHeight-2*\innerysep)/\pixelNumber}
    \edef\pixelHeight{\pgfmathresult pt}%
  }

  \savedanchor{\northeast}{
    \pgf@x=\objectWidth%
    \pgf@y=\objectHeight%
    \pgf@y=0.5\pgf@y%
    \pgf@x=0.5\pgf@x%
  }
  \anchor{north east}{\northeast}

  \savedanchor{\southwest}{
    \pgf@x=\objectWidth%
    \pgf@y=\objectHeight%
    \pgf@y=-0.5\pgf@y%
    \pgf@x=-0.5\pgf@x%
  }
  \anchor{south west}{\southwest}

  \anchor{north}{
    \center \pgf@xa=\pgf@x \pgf@ya=\pgf@y
    \northeast \pgf@xb=\pgf@x \pgf@yb=\pgf@y
    \pgf@x=\pgf@xa
    \pgf@y=\pgf@yb
  }

  \anchor{south}{
    \center \pgf@xa=\pgf@x \pgf@ya=\pgf@y
    \southwest \pgf@xb=\pgf@x \pgf@yb=\pgf@y
    \pgf@x=\pgf@xa
    \pgf@y=\pgf@yb
  }

  \anchor{east}{
    \center \pgf@xa=\pgf@x \pgf@ya=\pgf@y
    \northeast \pgf@xb=\pgf@x \pgf@yb=\pgf@y
    \pgf@x=\pgf@xb
    \pgf@y=\pgf@ya
  }

  \anchor{west}{
    \center \pgf@xa=\pgf@x \pgf@ya=\pgf@y
    \southwest \pgf@xb=\pgf@x \pgf@yb=\pgf@y
    \pgf@x=\pgf@xb
    \pgf@y=\pgf@ya
  }

  \anchor{north west}{
    \northeast \pgf@xa=\pgf@x \pgf@ya=\pgf@y
    \southwest \pgf@xb=\pgf@x \pgf@yb=\pgf@y
    \pgf@x=\pgf@xb
    \pgf@y=\pgf@ya
  }

  \anchor{south east}{
    \northeast \pgf@xa=\pgf@x \pgf@ya=\pgf@y
    \southwest \pgf@xb=\pgf@x \pgf@yb=\pgf@y
    \pgf@x=\pgf@xa
    \pgf@y=\pgf@yb
  }

  % ok, idée :
  % HAUT (north) du pixel n (allant de 1 à N)
  % \objectHeight/2 - \innerysep - (n-1) * \pixelHeight
  % BAS (south) du pixel n
  % \objectHeight/2 - \innerysep - n * \pixelHeight
  % west = le même que la boite, donc utiliser \southwest
  % east = west + \pixelWidth   

  % Le but de ce code est de définir des ancres
  % pixel <i> <pos>
  % pour <i> = 4, 2, ..., \pixelNumber
  % et <pos> = north, south, east, west, north west, south west, north east south east, center
  % Comme \pixelNumber est défini dynamiquement, il faut passer par cette horreur.
  % Le code est inspiré de pfglibraryshapes.geometric.code.tex (shape regular polygon).
  \expandafter\pgfutil@g@addto@macro\csname pgf@sh@s@sensor line\endcsname{%
    \c@pgf@counta\pixelNumber\relax%
    \pgfmathloop%
      \ifnum\c@pgf@counta>0\relax%
        \pgfutil@ifundefined{pgf@anchor@sensor line@pixel\space\the\c@pgf@counta\space north west}{%
        %
        % ...(manually \xdef as \gdef is normally used by \anchor)...
        %
        %
        % pixel surface north
        \expandafter\xdef\csname pgf@anchor@sensor line@pixel\space\the\c@pgf@counta\space north\endcsname{%
          \noexpand\northeast \noexpand\pgf@xa=\noexpand\pgf@x \noexpand\pgf@ya=\noexpand\pgf@y
          \noexpand\southwest \noexpand\pgf@xb=\noexpand\pgf@x \noexpand\pgf@yb=\noexpand\pgf@y
          \noexpand\pgf@x=\noexpand\pixelWidth
          \noexpand\pgf@x=.5\noexpand\pgf@x
          \noexpand\advance\noexpand\pgf@x by\noexpand\pgf@xb
          \noexpand\pgf@y=\noexpand\pgf@ya
          \noexpand\newdimen\noexpand\temp@y
          \noexpand\pgfmathsetlength\noexpand\temp@y{(\the\c@pgf@counta-\noexpand\pixelNumber)*\noexpand\pixelHeight-\noexpand\innerysep}
          \noexpand\advance\noexpand\pgf@y by\noexpand\temp@y
        }%
        %
        %
        % pixel surface north east
        \expandafter\xdef\csname pgf@anchor@sensor line@pixel\space\the\c@pgf@counta\space north east\endcsname{%
          \noexpand\northeast \noexpand\pgf@xa=\noexpand\pgf@x \noexpand\pgf@ya=\noexpand\pgf@y
          \noexpand\southwest \noexpand\pgf@xb=\noexpand\pgf@x \noexpand\pgf@yb=\noexpand\pgf@y
          \noexpand\pgf@x=\noexpand\pixelWidth
          \noexpand\advance\noexpand\pgf@x by\noexpand\pgf@xb
          \noexpand\pgf@y=\noexpand\pgf@ya
          \noexpand\newdimen\noexpand\temp@y
          \noexpand\pgfmathsetlength\noexpand\temp@y{(\the\c@pgf@counta-\noexpand\pixelNumber)*\noexpand\pixelHeight-\noexpand\innerysep}
          \noexpand\advance\noexpand\pgf@y by\noexpand\temp@y
        }%
        %
        %
        % pixel surface north west
        \expandafter\xdef\csname pgf@anchor@sensor line@pixel\space\the\c@pgf@counta\space north west\endcsname{%
          \noexpand\northeast \noexpand\pgf@xa=\noexpand\pgf@x \noexpand\pgf@ya=\noexpand\pgf@y
          \noexpand\southwest \noexpand\pgf@xb=\noexpand\pgf@x \noexpand\pgf@yb=\noexpand\pgf@y
          \noexpand\pgf@x=\noexpand\pgf@xb
          \noexpand\pgf@y=\noexpand\pgf@ya
          \noexpand\newdimen\noexpand\temp@y
          \noexpand\pgfmathsetlength\noexpand\temp@y{(\the\c@pgf@counta-\noexpand\pixelNumber)*\noexpand\pixelHeight-\noexpand\innerysep}
          \noexpand\advance\noexpand\pgf@y by\noexpand\temp@y
        }%
        %
        %
        % pixel surface south
        \expandafter\xdef\csname pgf@anchor@sensor line@pixel\space\the\c@pgf@counta\space south\endcsname{%
          \noexpand\northeast \noexpand\pgf@xa=\noexpand\pgf@x \noexpand\pgf@ya=\noexpand\pgf@y
          \noexpand\southwest \noexpand\pgf@xb=\noexpand\pgf@x \noexpand\pgf@yb=\noexpand\pgf@y
          \noexpand\pgf@x=\noexpand\pixelWidth
          \noexpand\pgf@x=.5\noexpand\pgf@x
          \noexpand\advance\noexpand\pgf@x by\noexpand\pgf@xb
          \noexpand\pgf@y=\noexpand\pgf@ya
          \noexpand\newdimen\noexpand\temp@y
          \noexpand\pgfmathsetlength\noexpand\temp@y{(\the\c@pgf@counta-\noexpand\pixelNumber-1)*\noexpand\pixelHeight-\noexpand\innerysep}
          \noexpand\advance\noexpand\pgf@y by\noexpand\temp@y
        }%
        %
        %
        % pixel surface south east
        \expandafter\xdef\csname pgf@anchor@sensor line@pixel\space\the\c@pgf@counta\space south east\endcsname{%
          \noexpand\northeast \noexpand\pgf@xa=\noexpand\pgf@x \noexpand\pgf@ya=\noexpand\pgf@y
          \noexpand\southwest \noexpand\pgf@xb=\noexpand\pgf@x \noexpand\pgf@yb=\noexpand\pgf@y
          \noexpand\pgf@x=\noexpand\pixelWidth
          \noexpand\advance\noexpand\pgf@x by\noexpand\pgf@xb
          \noexpand\pgf@y=\noexpand\pgf@ya
          \noexpand\newdimen\noexpand\temp@y
          \noexpand\pgfmathsetlength\noexpand\temp@y{(\the\c@pgf@counta-\noexpand\pixelNumber-1)*\noexpand\pixelHeight-\noexpand\innerysep}
          \noexpand\advance\noexpand\pgf@y by\noexpand\temp@y
        }%
        %
        %
        % pixel surface south west
        \expandafter\xdef\csname pgf@anchor@sensor line@pixel\space\the\c@pgf@counta\space south west\endcsname{%
          \noexpand\northeast \noexpand\pgf@xa=\noexpand\pgf@x \noexpand\pgf@ya=\noexpand\pgf@y
          \noexpand\southwest \noexpand\pgf@xb=\noexpand\pgf@x \noexpand\pgf@yb=\noexpand\pgf@y
          \noexpand\pgf@x=\noexpand\pgf@xb
          \noexpand\pgf@y=\noexpand\pgf@ya
          \noexpand\newdimen\noexpand\temp@y
          \noexpand\pgfmathsetlength\noexpand\temp@y{(\the\c@pgf@counta-\noexpand\pixelNumber-1)*\noexpand\pixelHeight-\noexpand\innerysep}
          \noexpand\advance\noexpand\pgf@y by\noexpand\temp@y
        }%
        %
        %
        % pixel surface east
        \expandafter\xdef\csname pgf@anchor@sensor line@pixel\space\the\c@pgf@counta\space east\endcsname{%
          \noexpand\northeast \noexpand\pgf@xa=\noexpand\pgf@x \noexpand\pgf@ya=\noexpand\pgf@y
          \noexpand\southwest \noexpand\pgf@xb=\noexpand\pgf@x \noexpand\pgf@yb=\noexpand\pgf@y
          \noexpand\pgf@x=\noexpand\pixelWidth
          \noexpand\advance\noexpand\pgf@x by\noexpand\pgf@xb
          \noexpand\pgf@y=\noexpand\pgf@ya
          \noexpand\newdimen\noexpand\temp@y
          \noexpand\pgfmathsetlength\noexpand\temp@y{(\the\c@pgf@counta-\noexpand\pixelNumber-0.5)*\noexpand\pixelHeight-\noexpand\innerysep}
          \noexpand\advance\noexpand\pgf@y by\noexpand\temp@y
        }%
        %
        %
        % pixel surface center
        \expandafter\xdef\csname pgf@anchor@sensor line@pixel\space\the\c@pgf@counta\space center\endcsname{%
          \noexpand\northeast \noexpand\pgf@xa=\noexpand\pgf@x \noexpand\pgf@ya=\noexpand\pgf@y
          \noexpand\southwest \noexpand\pgf@xb=\noexpand\pgf@x \noexpand\pgf@yb=\noexpand\pgf@y
          \noexpand\pgf@x=\noexpand\pixelWidth
          \noexpand\pgf@x=.5\noexpand\pgf@x
          \noexpand\advance\noexpand\pgf@x by\noexpand\pgf@xb
          \noexpand\pgf@y=\noexpand\pgf@ya
          \noexpand\newdimen\noexpand\temp@y
          \noexpand\pgfmathsetlength\noexpand\temp@y{(\the\c@pgf@counta-\noexpand\pixelNumber-0.5)*\noexpand\pixelHeight-\noexpand\innerysep}
          \noexpand\advance\noexpand\pgf@y by\noexpand\temp@y
        }%
        %
        %
        % pixel surface west
        \expandafter\xdef\csname pgf@anchor@sensor line@pixel\space\the\c@pgf@counta\space west\endcsname{%
          \noexpand\northeast \noexpand\pgf@xa=\noexpand\pgf@x \noexpand\pgf@ya=\noexpand\pgf@y
          \noexpand\southwest \noexpand\pgf@xb=\noexpand\pgf@x \noexpand\pgf@yb=\noexpand\pgf@y
          \noexpand\pgf@x=\noexpand\pgf@xb
          \noexpand\pgf@y=\noexpand\pgf@ya
          \noexpand\newdimen\noexpand\temp@y
          \noexpand\pgfmathsetlength\noexpand\temp@y{(\the\c@pgf@counta-\noexpand\pixelNumber-0.5)*\noexpand\pixelHeight-\noexpand\innerysep}
          \noexpand\advance\noexpand\pgf@y by\noexpand\temp@y
        }%
      }{\c@pgf@counta0\relax}% 
      \advance\c@pgf@counta-1\relax%
    \repeatpgfmathloop% 
  }%

  \inheritanchorborder[from=rectangle]

  \backgroundpath
  {
    % rectangle contour
    \pgfpathrectanglecorners{\northeast}{\southwest}

    % dessin des pixels
    \c@pgf@counta\pixelNumber\relax%
    \pgfmathloop%
      \ifnum\c@pgf@counta>0\relax%
        \newdimen\pixel@northeast@x
        \newdimen\pixel@northeast@y
        \newdimen\pixel@southwest@x
        \newdimen\pixel@southwest@y
        \northeast \pgf@xa=\pgf@x \pgf@ya=\pgf@y
        \southwest \pgf@xb=\pgf@x \pgf@yb=\pgf@y
        % calcul pixel@northeast
        %% calcul x
        \pixel@northeast@x=\pgf@xb
        \advance\pixel@northeast@x by\pixelWidth
        %% calcul y
        \pixel@northeast@y=\pgf@ya
        \newdimen\temp@y
        \pgfmathsetlength\temp@y{(\the\c@pgf@counta-\pixelNumber)*\pixelHeight-\innerysep}
        \advance\pixel@northeast@y by\noexpand\temp@y
        % calcul pixel@southwest
        \pixel@southwest@x=\pgf@xb
        \pixel@southwest@y=\pgf@ya
        \newdimen\temp@y
        \pgfmathsetlength\temp@y{(\the\c@pgf@counta-\pixelNumber-1)*\pixelHeight-\innerysep}
        \advance\pixel@southwest@y by\temp@y
        % dessin
        \pgfpathrectanglecorners{\pgfpoint{\pixel@northeast@x}{\pixel@northeast@y}}{\pgfpoint{\pixel@southwest@x}{\pixel@southwest@y}}
      \advance\c@pgf@counta-1\relax%
    \repeatpgfmathloop% 
  }
}


%%%%%%%%%%%%%%%%%%%%%%%%%%%%%%%%%%%%%%%%%%%%%%%%%%%%%%%%%%%%%%%%%%%%%%%%%%%%%%%%
%%%%%%%%%%%%%%%%%%%%%%%%%%%%%%%%%%%%%%%%%%%%%%%%%%%%%%%%%%%%%%%%%%%%%%%%%%%%%%%%
% Styles defining optics elements in terms of the existing shapes.
%%%%%%%%%%%%%%%%%%%%%%%%%%%%%%%%%%%%%%%%%%%%%%%%%%%%%%%%%%%%%%%%%%%%%%%%%%%%%%%%
%%%%%%%%%%%%%%%%%%%%%%%%%%%%%%%%%%%%%%%%%%%%%%%%%%%%%%%%%%%%%%%%%%%%%%%%%%%%%%%%

% Style [screen]
\pgfkeys{/tikz/optics/screen/.style={thin optics element, very thick}}
% Style [diffraction grating]
\pgfkeys{/tikz/optics/diffraction grating/.style={thin optics element, /tikz/optics/cheating dash={on 4pt off 2pt}}}
% Style [grid]
\pgfkeys{/tikz/optics/grid/.style={thin optics element, /tikz/optics/cheating dash={on 4pt off 2pt}, ultra thick}}
% Style [semi-transparent mirror]
\pgfkeys{/tikz/optics/semi-transparent mirror/.style={thin optics element, densely dotted, thick}}
% Style [diaphragm]
\pgfkeys{/tikz/optics/diaphragm/.style={slit,/tikz/optics/slit height=0.4}}


% Style [generic lamp]
\pgfkeys{/tikz/optics/generic lamp/.style={shape=generic optics io,optics,draw}}
% Style [generic sensor]
\pgfkeys{/tikz/optics/generic sensor/.style={shape=generic optics io,optics,io orientation=rtl,draw, io body height=1cm, io body aspect ratio=0.5, io aperture width=0.15}}
% Style [halogen lamp]
\pgfkeys{/tikz/optics/halogen lamp/.style={/tikz/optics/generic lamp, io body height=0.75cm, io body aspect ratio=2,io aperture width=0.33,io aperture height=0.66,io aperture shift=0}}
% Style [spectral lamp]
\pgfkeys{/tikz/optics/spectral lamp/.style={/tikz/optics/generic lamp, io body height=2.25cm, io body aspect ratio=2/3,io aperture width=0.11,io aperture height=0.22,io aperture shift=0.25,/tikz/optics/io multiline}}
% Style [laser]
\pgfkeys{/tikz/optics/laser/.style={/tikz/optics/generic lamp, io body height=0.5cm, io body aspect ratio=3,io aperture width=0.22,io aperture height=0.5,io aperture shift=0}}
% Style [laser']
\pgfkeys{/tikz/optics/laser'/.style={/tikz/optics/generic lamp, io body height=0.5cm, io body aspect ratio=3,io aperture width=0,io aperture height=0.5,io aperture shift=0}}
% Style [beam splitter]
\pgfkeys{/tikz/optics/beam splitter/.style={shape=polarizer, optics, draw, object height=1cm, object aspect ratio=1}}



%%%%%%%%%%%%%%%%%%%%%%%%%%%%%%%%%%%%%%%%%%%%%%%%%%%%%%%%%%%%%%%%%%%%%%%%%%%%%%%%
% Helper style to draw correctly dashed paths.
%%%%%%%%%%%%%%%%%%%%%%%%%%%%%%%%%%%%%%%%%%%%%%%%%%%%%%%%%%%%%%%%%%%%%%%%%%%%%%%%
% source : http://tex.stackexchange.com/questions/133271/can-tikz-dashed-lines-emulate-pstricks-dashed-lines
% le but est d'avoir des dash symétriques par rapport au milieu et surtout avec un trait entier de chaque côté
\tikzset{
    /tikz/optics/cheating dash/.code args={on #1 off #2}{
        % Use csname so catcode of @ doesn't have do be changed.
        \csname tikz@addoption\endcsname{%
            \pgfgetpath\currentpath%
            \pgfprocessround{\currentpath}{\currentpath}%
            \csname pgf@decorate@parsesoftpath\endcsname{\currentpath}{\currentpath}%
            \pgfmathparse{\csname pgf@decorate@totalpathlength\endcsname-#1}\let\rest=\pgfmathresult%
            \pgfmathparse{#1+#2}\let\onoff=\pgfmathresult%
            \pgfmathparse{max(floor(\rest/\onoff), 1)}\let\nfullonoff=\pgfmathresult%
            \pgfmathparse{max((\rest-\onoff*\nfullonoff)/\nfullonoff+#2, #2)}\let\offexpand=\pgfmathresult%
            \pgfsetdash{{#1}{\offexpand}}{0pt}}%
    }
}


%%%%%%%%%%%%%%%%%%%%%%%%%%%%%%%%%%%%%%%%%%%%%%%%%%%%%%%%%%%%%%%%%%%%%%%%%%%%%%%%
% Helper style for multiline io elements.
%%%%%%%%%%%%%%%%%%%%%%%%%%%%%%%%%%%%%%%%%%%%%%%%%%%%%%%%%%%%%%%%%%%%%%%%%%%%%%%%
\pgfkeys{/tikz/optics/io multiline/.code={\newdimen\tmplen\pgfmathsetlength{\tmplen}{\pgfkeysvalueof{/tikz/optics/io body aspect ratio}*\pgfkeysvalueof{/tikz/optics/io body height}}\tikzset{text width=\the\tmplen,align=center}}}


%%%%%%%%%%%%%%%%%%%%%%%%%%%%%%%%%%%%%%%%%%%%%%%%%%%%%%%%%%%%%%%%%%%%%%%%%%%%%%%%
%%%%%%%%%%%%%%%%%%%%%%%%%%%%%%%%%%%%%%%%%%%%%%%%%%%%%%%%%%%%%%%%%%%%%%%%%%%%%%%%
% Helper styles to mark interesing points
%%%%%%%%%%%%%%%%%%%%%%%%%%%%%%%%%%%%%%%%%%%%%%%%%%%%%%%%%%%%%%%%%%%%%%%%%%%%%%%%
%%%%%%%%%%%%%%%%%%%%%%%%%%%%%%%%%%%%%%%%%%%%%%%%%%%%%%%%%%%%%%%%%%%%%%%%%%%%%%%%


% Style [mark point]
% Describes how interesting points should be drawn (e.g. by [draw focal points], [draw mirror focus], [draw mirror center])
\pgfkeys{/tikz/optics/mark point/.style={optics/mark a cross}}

% Style [draw focal points]
% Should be applied to a [shape=lens] node to draw its focal points according to the style [mark point].
\pgfkeys{/tikz/optics/draw focal points/.style={append after command={
    \pgfextra{
      \begin{pgfinterruptpath}
          \node[/tikz/optics/mark point,#1] at (\tikzlastnode.west focal point) {};
          \node[/tikz/optics/mark point,#1] at (\tikzlastnode.east focal point) {};
      \end{pgfinterruptpath}
    }
  }
}}

% Style [draw mirror focus]
% Should be applied to a [shape=spherical mirror] node to draw its focal point according to the style [mark point].
\pgfkeys{/tikz/optics/draw mirror focus/.style={append after command={
    \pgfextra{
      \begin{pgfinterruptpath}
          \node[/tikz/optics/mark point,#1] at (\tikzlastnode.focus) {};
      \end{pgfinterruptpath}
    }
  }
}}

% Style [draw mirror center]
% Should be applied to a [shape=spherical mirror] node to draw its center according to the style [mark point].
\pgfkeys{/tikz/optics/draw mirror center/.style={append after command={
    \pgfextra{
      \begin{pgfinterruptpath}
          \node[/tikz/optics/mark point,#1] at (\tikzlastnode.mirror center) {};
      \end{pgfinterruptpath}
    }
  }
}}

% Style [mark a cross]
% Draws a cross at the node
\pgfkeys{/tikz/optics/mark a cross/.style={cross out,draw,inner sep=0pt,minimum width=2pt,minimum height=2pt}}

% idée : utiliser pgfextra pour les flèches, si on peut avoir les deux dernières nodes ?

%%%%%%%%%%%%%%%%%%%%%%%%%%%%%%%%%%%%%%%%%%%%%%%%%%%%%%%%%%%%%%%%%%%%%%%%%%%%%%%%
%%%%%%%%%%%%%%%%%%%%%%%%%%%%%%%%%%%%%%%%%%%%%%%%%%%%%%%%%%%%%%%%%%%%%%%%%%%%%%%%
% Helper styles to easily put things on paths.
% It is often needed in optics to have e.g. arrows in the middle of a path.
% It is often needed everywhere to be able to put a coordinate somewhere on 
% a path. 
% This is the aim of these helpers.
%%%%%%%%%%%%%%%%%%%%%%%%%%%%%%%%%%%%%%%%%%%%%%%%%%%%%%%%%%%%%%%%%%%%%%%%%%%%%%%%
%%%%%%%%%%%%%%%%%%%%%%%%%%%%%%%%%%%%%%%%%%%%%%%%%%%%%%%%%%%%%%%%%%%%%%%%%%%%%%%%


%%%%%%%%%%%%%%%%%%%%%%%%%%%%%%%%%%%%%%%%%%%%%%%%%%%%%%%%%%%%%%%%%%%%%%%%%%%%%%%%
% Style [put coordinate]
%%%%%%%%%%%%%%%%%%%%%%%%%%%%%%%%%%%%%%%%%%%%%%%%%%%%%%%%%%%%%%%%%%%%%%%%%%%%%%%%
% Le style [put coordinate=<coord> at <pos>] crée une node[coordinate]
% à l'abscisse curviligne <pos> sur le chemin auquel est appliqué le style.
%
% Exemple : 
% \draw[put coordinate=P at 0.3] (0,0) to[bend left] (2cm,0);
% \draw[red] (P) -- (0,0);
%
\tikzset{
  put coordinate/.style args={#1 at #2}{decoration={markings, mark=at position #2 with {\node[coordinate] (#1) {};}},postaction={decorate}}
}

%%%%%%%%%%%%%%%%%%%%%%%%%%%%%%%%%%%%%%%%%%%%%%%%%%%%%%%%%%%%%%%%%%%%%%%%%%%%%%%%
% Style [put arrow]
%%%%%%%%%%%%%%%%%%%%%%%%%%%%%%%%%%%%%%%%%%%%%%%%%%%%%%%%%%%%%%%%%%%%%%%%%%%%%%%%
%
% The aim of this horrible mess is that I want expansion to take place at the right time, so I can ask for something like
% \draw[/tikz/optics/->-={at=0.25}, /tikz/optics/->-={at=0.75}] (0,0) -- (2cm,2cm) -- (4cm,0);
% and have it work. Obviously, I want more complicated things (several arrows with different directions, colors, number of >, etc. on the same path).
% With a naive implementation, this does not work and the last setting always wins. 
% A possible solution would be positional arguments, however keywords arguments are Better (tm).
% Hence the need to take care of expansion order, so the specifications apply only to the currently drawn arrow tip.
% The main idea behing this code is to use Magic (tm) so that Things Work (tm) and never touch it again. 
% However, it is obvious from a trivial application of Murphy Law that this will backfire sometime.
% Indeed, a key with positional arguments are used internally (|ordered draw key|), which is called from the parsed arguments.
\tikzset{/tikz/put arrow/.cd,
  pos/.initial=0.5,
  at/.style={pos=#1},
  pos var/.initial={},
  style/.initial={},
  style var/.initial={},
  postaction style/.initial={}, % more expansion magic needed for ->n- and friends (\arrow[thing=stuff] does not seem to work, probably because of the =)
  arrow macro/.initial={arrow},
  arrow macro var/.initial={},
  arrow type/.initial={>},
  arrow type var/.initial={},
  reversed/.style={arrow macro=arrowreversed},
  arrow/.style={arrow type=#1},
  arrow'/.style={arrow type=#1, reversed},
  every arrow/.style={},
}

\tikzset{put arrow/.code={
  % parse arguments correctly
  \tikzset{/tikz/put arrow/.cd, #1}
  \tikzset{/tikz/put arrow/pos var/.expand once={\pgfkeysvalueof{/tikz/put arrow/pos}}}
  \tikzset{/tikz/put arrow/style var/.expanded={\pgfkeysvalueof{/tikz/put arrow/style}}}
  \tikzset{/tikz/put arrow/arrow macro var/.expand once={\pgfkeysvalueof{/tikz/put arrow/arrow macro}}}
  \tikzset{/tikz/put arrow/arrow type var/.expand once={\pgfkeysvalueof{/tikz/put arrow/arrow type}}}
  % call |ordered draw key| which does the real job
  \tikzset{/tikz/put arrow/ordered draw key/.expanded=%
      {\pgfkeysvalueof{/tikz/put arrow/pos var}}%
      {\pgfkeysvalueof{/tikz/put arrow/style var}}%
      {\pgfkeysvalueof{/tikz/put arrow/arrow macro var}}%
      {\pgfkeysvalueof{/tikz/put arrow/arrow type var}}%
      {\pgfkeysvalueof{/tikz/put arrow/postaction style}}%
  }
  % restore initial values
  \tikzset{/tikz/put arrow/pos=0.5} 
  \tikzset{/tikz/put arrow/style={}}
  \tikzset{/tikz/put arrow/arrow macro={arrow}}
  \tikzset{/tikz/put arrow/arrow type={>}}
}}

\tikzset{/tikz/put arrow/ordered draw key/.code n args={5}{
  \tikzset{postaction={#5,decorate, decoration={markings, mark=at position #1 with {\csname #3\endcsname[/tikz/put arrow/every arrow,#2]{#4}};}}}
}}

%%%%%%%%%%%%%%%%%%%%%%%%%%%%%%%%%%%%%%%%%%%%%%%%%%%%%%%%%%%%%%%%%%%%%%%%%%%%%%%%
% Styles to mark light rays
%%%%%%%%%%%%%%%%%%%%%%%%%%%%%%%%%%%%%%%%%%%%%%%%%%%%%%%%%%%%%%%%%%%%%%%%%%%%%%%%
\tikzset{/tikz/optics/multiple ray arrow/.cd,
  n/.initial=1,
  n var/.initial=1,
  set n/.code={\pgfsetarrowoptions{multiple ray arrow}{#1}},
}


\pgfkeys{
  /tikz/optics/.cd, 
  use ray arrow >/.code 2 args={
    \pgfsetarrowoptions{ray arrow@length}{4pt}
    \pgfsetarrowoptions{ray arrow@angle}{50}
    \tikzset{/tikz/put arrow/postaction style/.expanded={/tikz/optics/multiple ray arrow/set n=#1}}
    \tikzset{/tikz/put arrow/.expanded={arrow={multiple ray arrow}, #2}}
  },
  use ray arrow </.code 2 args={
    \pgfsetarrowoptions{ray arrow@length}{4pt}
    \pgfsetarrowoptions{ray arrow@angle}{50}
    \tikzset{/tikz/put arrow/postaction style/.expanded={/tikz/optics/multiple ray arrow/set n=#1}}
    \tikzset{/tikz/put arrow/.expanded={arrow'={multiple ray arrow}, #2}}
  },
  ->n-/.code={
    \tikzset{/tikz/optics/multiple ray arrow/.cd, .collect unknowns,%
      #1,
      unknown options/.get = \arrowkeys}
    \tikzset{/tikz/optics/multiple ray arrow/n var/.expand once={\pgfkeysvalueof{/tikz/optics/multiple ray arrow/n}}}
    \tikzset{/tikz/optics/use ray arrow >={\pgfkeysvalueof{/tikz/optics/multiple ray arrow/n var}}{\arrowkeys}}
  },
  -<n-/.code={
    \tikzset{/tikz/optics/multiple ray arrow/.cd, .collect unknowns,%
      #1,
      unknown options/.get = \arrowkeys}
    \tikzset{/tikz/optics/multiple ray arrow/n var/.expand once={\pgfkeysvalueof{/tikz/optics/multiple ray arrow/n}}}
    \tikzset{/tikz/optics/use ray arrow <={\pgfkeysvalueof{/tikz/optics/multiple ray arrow/n var}}{\arrowkeys}}
  },
  %
  ->-/.style={/tikz/optics/->n-={n=1, #1}},
  -<-/.style={/tikz/optics/-<n-={n=1, #1}},
  ->>-/.style={/tikz/optics/->n-={n=2, #1}},
  -<<-/.style={/tikz/optics/-<n-={n=2, #1}},
  ->>>-/.style={/tikz/optics/->n-={n=3, #1}},
  -<<<-/.style={/tikz/optics/-<n-={n=3, #1}},
  ->>>>-/.style={/tikz/optics/->n-={n=4, #1}},
  -<<<<-/.style={/tikz/optics/-<n-={n=4, #1}},
}




%%%%%%%%%%%%%%%%%%%%%%%%%%%%%%%%%%%%%%%%%%%%%%%%%%%%%%%%%%%%%%%%%%%%%%%%%%%%%%%%
%%%%%%%%%%%%%%%%%%%%%%%%%%%%%%%%%%%%%%%%%%%%%%%%%%%%%%%%%%%%%%%%%%%%%%%%%%%%%%%%
% Styles [dim arrow] and [short dim arrow]
%%%%%%%%%%%%%%%%%%%%%%%%%%%%%%%%%%%%%%%%%%%%%%%%%%%%%%%%%%%%%%%%%%%%%%%%%%%%%%%%
%%%%%%%%%%%%%%%%%%%%%%%%%%%%%%%%%%%%%%%%%%%%%%%%%%%%%%%%%%%%%%%%%%%%%%%%%%%%%%%%
\def\dimarrow@short@position{0}
\newif\ifdimarrow@nearstart
\dimarrow@nearstarttrue
\tikzset{%
  /tikz/dim arrow/.code={\tikzset{draw,/tikz/dim arrow/draw dim arrow}\pgfkeys{/tikz/dim arrow/.cd,#1}},
  /tikz/dim arrow'/.code={\pgfkeysgetvalue{/tikz/dim arrow/raise}{\tmp@tdar}\tikzset{draw,/tikz/dim arrow/draw dim arrow,/tikz/dim arrow/raise=-\tmp@tdar}\pgfkeys{/tikz/dim arrow/.cd,#1}},
  /tikz/short dim arrow/.code={\tikzset{draw,/tikz/dim arrow/draw short dim arrow}\pgfkeys{/tikz/dim arrow/.cd,#1}},
  /tikz/short dim arrow'/.code={\pgfkeysgetvalue{/tikz/dim arrow/raise}{\tmp@tdar}\tikzset{draw,/tikz/dim arrow/draw short dim arrow,/tikz/dim arrow/raise=-\tmp@tdar}\pgfkeys{/tikz/dim arrow/.cd,#1}},
  /tikz/dim arrow/.cd,
  raise/.initial={0.5cm},
  no raise/.style={raise=0},
  label/.code={\pgfkeys{/tikz/dim arrow/label text=#1}},
  label'/.code={\pgfkeys{/tikz/dim arrow/label text=#1,/tikz/dim arrow/label style/.append style={swap},}},
  label text/.initial={},
  label style/.style={},
  % /tikz/dim arrow/label pos is only for the (non short) dim arrows ; \dimarrow@short@position is only for the short version
  label pos/.initial={0.5}, 
  label near start/.code={\pgfkeys{/tikz/dim arrow/label pos=0} \def\dimarrow@short@position{0}}, 
  label near middle/.code={\pgfkeys{/tikz/dim arrow/label pos=0.5} \def\dimarrow@short@position{2}}, 
  label near end/.code={\pgfkeys{/tikz/dim arrow/label pos=1} \def\dimarrow@short@position{1}}, 
  arrow length/.initial={5mm}, % only for short
  arrow style/.style={|<->|}, % this is modified by [left/right dim arrow]
  ->/.code={\tikzset{/tikz/dim arrow/arrow style/.style={|->|}}},
  <-/.code={\tikzset{/tikz/dim arrow/arrow style/.style={|<-|}}},
  <->/.code={\tikzset{/tikz/dim arrow/arrow style/.style={|<->|}}},
  draw short dim arrow/.style={to path={\pgfextra{%
    \let\tikz@mode@save=\tikz@mode%
        \let\tikz@options@save=\tikz@options%
    \newdimen\labelTotalRaise
    \pgfmathsetlength\labelTotalRaise{\pgfkeysvalueof{/tikz/dim arrow/raise}}
        \pgfinterruptpath
        \draw[>=technical,->|] \pgfextra{\let\tikz@mode=\tikz@mode@save\let\tikz@options=\tikz@options@save}
    let
        \p1=($(\tikztostart)!\pgfkeysvalueof{/tikz/dim arrow/raise}!90:(\tikztotarget)$),
        \p2=($(\tikztotarget)!\pgfkeysvalueof{/tikz/dim arrow/raise}!-90:(\tikztostart)$)
        in ($(\p1)!-\pgfkeysvalueof{/tikz/dim arrow/arrow length}!(\p2)$) -- ($(\p1)!0!(\p2)$);
    \draw[>=technical,->|] \pgfextra{\let\tikz@mode=\tikz@mode@save\let\tikz@options=\tikz@options@save}
    let
        \p1=($(\tikztostart)!\pgfkeysvalueof{/tikz/dim arrow/raise}!90:(\tikztotarget)$),
        \p2=($(\tikztotarget)!\pgfkeysvalueof{/tikz/dim arrow/raise}!-90:(\tikztostart)$)
        in ($(\p2)!-\pgfkeysvalueof{/tikz/dim arrow/arrow length}!(\p1)$) -- ($(\p2)!0!(\p1)$);
    \ifnum\dimarrow@short@position=0
    \path let 
        \p1=($(\tikztostart)!\labelTotalRaise!90:(\tikztotarget)$),
        \p2=($(\tikztotarget)!\labelTotalRaise!-90:(\tikztostart)$)
        in let
    \p3=($(\p1)!-1*\pgfkeysvalueof{/tikz/dim arrow/arrow length}!(\p2)$),
    \p4=($(\p1)!-0*\pgfkeysvalueof{/tikz/dim arrow/arrow length}!(\p2)$)
    in
    (\p3) -- (\p4) node[pos=0.5,auto=left,/tikz/dim arrow/label style] {\pgfkeysvalueof{/tikz/dim arrow/label text}};
  \fi
    \ifnum\dimarrow@short@position=1
      \path let 
        \p1=($(\tikztostart)!\labelTotalRaise!90:(\tikztotarget)$),
        \p2=($(\tikztotarget)!\labelTotalRaise!-90:(\tikztostart)$)
        in let
    \p3=($(\p2)!-1*\pgfkeysvalueof{/tikz/dim arrow/arrow length}!(\p1)$),
    \p4=($(\p2)!-0*\pgfkeysvalueof{/tikz/dim arrow/arrow length}!(\p1)$)
    in
    (\p4) -- (\p3) node[pos=0.5,auto=left,/tikz/dim arrow/label style] {\pgfkeysvalueof{/tikz/dim arrow/label text}};
    \fi
  \ifnum\dimarrow@short@position=2
    \path let 
        \p1=($(\tikztostart)!\labelTotalRaise!90:(\tikztotarget)$),
        \p2=($(\tikztotarget)!\labelTotalRaise!-90:(\tikztostart)$)
    in
    (\p1) -- (\p2) node[pos=0.5,/tikz/dim arrow/label style] {\pgfkeysvalueof{/tikz/dim arrow/label text}};
  \fi
        \endpgfinterruptpath
      }(\tikztostart) (\tikztotarget) \tikztonodes
    }
  },
  draw dim arrow/.style={to path={\pgfextra{%
    \let\tikz@mode@save=\tikz@mode%
        \let\tikz@options@save=\tikz@options%
    \newdimen\labelTotalRaise
    \pgfmathsetlength\labelTotalRaise{\pgfkeysvalueof{/tikz/dim arrow/raise}}
        \pgfinterruptpath
        \draw[>=technical,/tikz/dim arrow/arrow style] \pgfextra{\let\tikz@mode=\tikz@mode@save\let\tikz@options=\tikz@options@save}
    let 
        \p1=($(\tikztostart)!\pgfkeysvalueof{/tikz/dim arrow/raise}!90:(\tikztotarget)$),
        \p2=($(\tikztotarget)!\pgfkeysvalueof{/tikz/dim arrow/raise}!-90:(\tikztostart)$)
        in (\p1) -- (\p2);
    \path let 
        \p1=($(\tikztostart)!\labelTotalRaise!90:(\tikztotarget)$),
        \p2=($(\tikztotarget)!\labelTotalRaise!-90:(\tikztostart)$)
        in (\p1) -- (\p2) node[pos=\pgfkeysvalueof{/tikz/dim arrow/label pos},auto=left,/tikz/dim arrow/label style] {\pgfkeysvalueof{/tikz/dim arrow/label text}};
    % rq : inner sep controle la distance chemin-node
        \endpgfinterruptpath
      }(\tikztostart) (\tikztotarget) \tikztonodes
    }
  },
}

%%%%%%%%%%%%%%%%%%%%%%%%%%%%%%%%%%%%%%%%%%%%%%%%%%%%%%%%%%%%%%%%%%%%%%%%%%%%%%%%
%%%%%%%%%%%%%%%%%%%%%%%%%%%%%%%%%%%%%%%%%%%%%%%%%%%%%%%%%%%%%%%%%%%%%%%%%%%%%%%%
% Arrows
%%%%%%%%%%%%%%%%%%%%%%%%%%%%%%%%%%%%%%%%%%%%%%%%%%%%%%%%%%%%%%%%%%%%%%%%%%%%%%%%
%%%%%%%%%%%%%%%%%%%%%%%%%%%%%%%%%%%%%%%%%%%%%%%%%%%%%%%%%%%%%%%%%%%%%%%%%%%%%%%%

%%%%%%%%%%%%%%%%%%%%%%%%%%%%%%%%%%%%%%%%%%%%%%%%%%%%%%%%%%%%%%%%%%%%%%%%%%%%%%%%
% Arrow lens arrow
% used to draw lenses (perhaps not the best idea).
%%%%%%%%%%%%%%%%%%%%%%%%%%%%%%%%%%%%%%%%%%%%%%%%%%%%%%%%%%%%%%%%%%%%%%%%%%%%%%%%
\pgfsetarrowoptions{lens arrow@length}{6pt}
\pgfsetarrowoptions{lens arrow@angle}{50}
\pgfarrowsdeclare{lens arrow}{lens arrow}
{
   \pgfarrowsleftextend{0pt}
   \pgfarrowsrightextend{0pt}
}
{
  \pgfsetroundcap
  \pgfsetmiterjoin
  \pgfmathsetlength{\pgfutil@tempdimb}{\pgfgetarrowoptions{lens arrow@length}*sin(\pgfgetarrowoptions{lens arrow@angle}/2)}    
  \def\arrow@origin{\pgfpoint{0pt}{0pt}}
  \pgfutil@tempdima=\pgfgetarrowoptions{lens arrow@length}%
  \pgfmathsetmacro{\tmp@lens@angle}{90+\pgfgetarrowoptions{lens arrow@angle}}
  \pgfmathsetmacro{\tmp@lens@anglediv}{\pgfgetarrowoptions{lens arrow@angle}/2}
  \advance\pgfutil@tempdima by -1.5\pgflinewidth%
  \pgfmathsetlength{\pgfutil@tempdima}{\pgfutil@tempdima/cos(\pgfgetarrowoptions{lens arrow@angle}/2)}    
  \pgfpathmoveto{\pgfpointadd{\arrow@origin}{\pgfqpointpolar{\tmp@lens@angle}{\pgfutil@tempdima}}}
  \pgfpathlineto{\arrow@origin}
  \pgfpathlineto{\pgfpointadd{\arrow@origin}{\pgfqpointpolar{-\tmp@lens@angle}{\pgfutil@tempdima}}}
  \pgfusepathqstroke
}
\pgfarrowsdeclarereversed{lens arrow reversed}{lens arrow reversed}{lens arrow}{lens arrow}


%%%%%%%%%%%%%%%%%%%%%%%%%%%%%%%%%%%%%%%%%%%%%%%%%%%%%%%%%%%%%%%%%%%%%%%%%%%%%%%%
% Ray arrow
% It is useful to have an arrow which goes on the exact middle of a path.
% This is used on ->-, etc.
%%%%%%%%%%%%%%%%%%%%%%%%%%%%%%%%%%%%%%%%%%%%%%%%%%%%%%%%%%%%%%%%%%%%%%%%%%%%%%%%
% flèche utilisée pour marquer les rayons lumineux (avec les styles ->-, etc.)
\pgfsetarrowoptions{ray arrow@length}{4pt}
\pgfsetarrowoptions{ray arrow@angle}{45}

\makeatletter

\pgfsetarrowoptions{multiple ray arrow}{0}
\pgfarrowsdeclare{multiple ray arrow}{multiple ray arrow}
{
    \pgfarrowsleftextend{0pt}
    \pgfarrowsrightextend{0pt}
}
{
  \pgfsetroundcap
  \pgfsetmiterjoin
  \pgfutil@tempdima=\pgfgetarrowoptions{ray arrow@length}%
  \pgfmathsetmacro{\tmp@ray@angle}{90+\pgfgetarrowoptions{ray arrow@angle}}
  \pgfmathsetmacro{\tmp@ray@anglediv}{\pgfgetarrowoptions{ray arrow@angle}/2}
  \advance\pgfutil@tempdima by -1.5\pgflinewidth%
  \pgfmathsetlength{\pgfutil@tempdima}{\pgfutil@tempdima/cos(\pgfgetarrowoptions{ray arrow@angle}/2)}
  %
  \foreach \i in {1,...,\pgfgetarrowoptions{multiple ray arrow}}
  {
    \pgfmathsetlength{\pgfutil@tempdimb}{(2*\i-\pgfgetarrowoptions{multiple ray arrow})*\pgfgetarrowoptions{ray arrow@length}*sin(\pgfgetarrowoptions{ray arrow@angle}/2)}
    %
    \def\arrow@origin{\pgfpoint{\pgfutil@tempdimb}{0pt}}
    %
    \pgfpathmoveto{\pgfpointadd{\arrow@origin}{\pgfqpointpolar{\tmp@ray@angle}{\pgfutil@tempdima}}}
    \pgfpathlineto{\arrow@origin}
    \pgfpathlineto{\pgfpointadd{\arrow@origin}{\pgfqpointpolar{-\tmp@ray@angle}{\pgfutil@tempdima}}}
    \pgfusepathqstroke
  }
}


%%%%%%%%%%%%%%%%%%%%%%%%%%%%%%%%%%%%%%%%%%%%%%%%%%%%%%%%%%%%%%%%%%%%%%%%%%%%%%%%
% Arrow technical
%%%%%%%%%%%%%%%%%%%%%%%%%%%%%%%%%%%%%%%%%%%%%%%%%%%%%%%%%%%%%%%%%%%%%%%%%%%%%%%%
\makeatletter
\pgfarrowsdeclare{technical}{technical}
{%
  \pgfutil@tempdima=0.48pt%
  \pgfutil@tempdimb=\pgflinewidth%
  \ifdim\pgfinnerlinewidth>0pt%
    \pgfmathsetlength\pgfutil@tempdimb{.6\pgflinewidth-.4*\pgfinnerlinewidth}%
  \fi%
  \advance\pgfutil@tempdima by.3\pgfutil@tempdimb%
  \pgfarrowsleftextend{+-3\pgfutil@tempdima}%
  \pgfarrowsrightextend{+8\pgfutil@tempdima}%
}
{%
  \pgfutil@tempdima=0.48pt%
  \pgfutil@tempdimb=\pgflinewidth%
  \ifdim\pgfinnerlinewidth>0pt%
    \pgfmathsetlength\pgfutil@tempdimb{.6\pgflinewidth-.4*\pgfinnerlinewidth}%
  \fi%
  \advance\pgfutil@tempdima by.3\pgfutil@tempdimb%
  \pgfpathmoveto{\pgfqpoint{8\pgfutil@tempdima}{0pt}}%
  \pgfpathlineto{\pgfqpoint{-3\pgfutil@tempdima}{3\pgfutil@tempdima}}%
  \pgfpathlineto{\pgfpointorigin}%
  \pgfpathlineto{\pgfqpoint{-3\pgfutil@tempdima}{-3\pgfutil@tempdima}}%
  \pgfusepathqfill%
}

\pgfarrowsdeclare{technical reversed}{technical reversed}
{%
  \pgfutil@tempdima=0.48pt%
  \pgfutil@tempdimb=\pgflinewidth%
  \ifdim\pgfinnerlinewidth>0pt%
    \pgfmathsetlength\pgfutil@tempdimb{.6\pgflinewidth-.4*\pgfinnerlinewidth}%
  \fi%
  \advance\pgfutil@tempdima by.3\pgfutil@tempdimb%
  \pgfarrowsleftextend{-8\pgfutil@tempdima}
  \pgfarrowsrightextend{-8\pgfutil@tempdima}
}
{%
  \pgfutil@tempdima=0.48pt%
  \pgfutil@tempdimb=\pgflinewidth%
  \ifdim\pgfinnerlinewidth>0pt%
    \pgfmathsetlength\pgfutil@tempdimb{.6\pgflinewidth-.4*\pgfinnerlinewidth}%
  \fi%
  \advance\pgfutil@tempdima by.3\pgfutil@tempdimb%
  \pgfpathmoveto{\pgfqpoint{-8\pgfutil@tempdima}{0pt}}%
  \pgfpathlineto{\pgfqpoint{3\pgfutil@tempdima}{3\pgfutil@tempdima}}%
  \pgfpathlineto{\pgfpointorigin}%
  \pgfpathlineto{\pgfqpoint{3\pgfutil@tempdima}{-3\pgfutil@tempdima}}%
  \pgfusepathqfill%
}


% Changelog:
% 2013-10-21 : ajout du style |distance arrow| et de la décoration |line| correspondante.
% 2013-11-19 : suppression de |distance arrow| et ajout à la place de |dim arrow| et assimilés
% 2013-11-22 : choix entre distances relatives et absolues (http://www.texample.net/tikz/examples/supersonic-nozzle/)
% 2013-11-24 : styles de flèches |->-|, |-<-|, |->>-|, |-<<-| (et flèches pgf |ray arrow|, etc. correspondantes)
% 2013-11-24 : flèches pgf |lens arrow| et |lens arrow reversed|
% 2013-11-24 : |generic optics element| -> |thin optics element| et |thick optics element| ; conséquences. |beam splitter|
% 2014-01-01 : ajout de |double amici prism|,  |optics| -> |use optics| et |one arrow| -> |put arrow| ; |mark a *| supprimés
% 2014-03-19 : anchorborder pour |generic optics io| (les labels devraient donc être placés correctement)
%              |io body aspect ratio| accepte désormais aussi des longueurs absolues, ajout d'un alias |io body width| pour |io body aspect ratio|
% 2014-06-26 : modification du code des flèches |->-|, |->>-|, etc. et ajout de |->>>-|, |->>>>-|,|->n-=<nombre>| (idem dans l'autre sens)
% 2014-07-08 : ajout de |arrow style| à |put arrow|
% 2014-09-20 : ajout de |spherical mirror| et quelques modifications à |mirror| (ajustement de la décoration et de ses réglages par défaut)
% 2014-09-22 : ajustements de |spherical mirror| (concave et convexe), et ajout des styles correspondants |convex mirror| et |concave mirror|
% 2014-09-24 : ajustements de |spherical mirror| (ltr/rtl) ; correction des ancres de generic optics io (aperture north, aperture center, aperture south étaient incorrectes)
% 2014-09-25 : corrections à |spherical mirror| (ltr/rtl vs concave/convex) ; 
% 2014-10-02 : ajout d'une fonction |from_radius| pour définir l'angle d'ouverture de |spherical mirror|, encore des corrections à |spherical mirror| (ltr/rtl vs concave/convex) ; 
%               macro pour les messages d'erreur
% 2014-10-03 : vérifications de cohérence des grandeurs pour |slit| et |double slit| ; messages d'erreur au besoin
% 2014-12-07 : modifications substantielles à |put arrow| et |optics/->n-|, etc. pour pouvoir avoir plusieurs flèches sur le même chemin ; la compatibilité arrière est brisée.
% 2014-12-11 : nettoyage
% 2015-03-10 : ajout d'un alias |object width| pour |object aspect ratio|, qui accepte désormais aussi des longueurs absolues
% 2015-06-13 : mise en cohérence des noms des points focaux pour le miroir et la lentille (désormais, "focus" et "focal point")
% 2016-11-21 : appel aux biblothèques tikz |decorations| et |decorations.pathreplacing| qui sont nécessaires
\makeatother