\iffalse
The tikzoptics library documentation

Copyright (C) 2013-2019 by
  Michel Fruchart <michel.fruchart@ens-lyon.org>

This work may be distributed and/or modified under the conditions of
the LaTeX Project Public License (LPPL), version 1.3c, which can be
found at the address:
https://www.latex-project.org/lppl/lppl-1-3c/

Alternatively, it may be distributed and/or modified under the conditions of
the GNU Free Documentation License (GNU FDL), version 1.3, which can be found
at the address:
https://www.gnu.org/licenses/fdl-1.3.en.html
or any later version published by the Free Software Foundation.
\fi
% allow compression of cross references.
\pdfminorversion=5 
\pdfobjcompresslevel=2
\documentclass[a4paper]{ltxdoc}
\usepackage{etex}

\usepackage[utf8]{inputenc}
\usepackage[T1]{fontenc}
\usepackage[french]{babel}

\usepackage[a4paper,left=2.25cm,right=2.25cm,top=2.5cm,bottom=2.5cm,nohead]{geometry}
\usepackage{amsmath}

%\usepackage{makeidx}
\usepackage[version=3]{mhchem}

\usepackage{csquotes}

\usepackage{xkeyval,calc,listings}
\usepackage[svgnames,x11names]{xcolor}
\usepackage{xxcolor} % tikz

\usepackage{tikz}
\usetikzlibrary{shapes,shapes.misc,shapes.geometric} 
\usetikzlibrary{arrows}
\usetikzlibrary{calc}
\usetikzlibrary{positioning}
\usetikzlibrary{decorations,decorations.markings,decorations.pathreplacing,decorations.pathmorphing}
\usetikzlibrary{patterns}
\usetikzlibrary{intersections}
\usetikzlibrary{matrix}
\usetikzlibrary{fit}
\usetikzlibrary{ocgx}


\usepackage{fancyvrb}


\usepackage[framemethod=tikz,usetwoside=false]{mdframed}

\usepackage{fourier-orns}
\usepackage{lmodern} 
\usepackage[charter]{mathdesign}
\usepackage{charter}
\def\rmdefault{bch} % not scaled
\def\sfdefault{SourceSansPro-TLF}

\usepackage{textcomp}
\usepackage[detect-all=true]{siunitx}
\sisetup{
        math-micro=\muup,
        math-ohm  =\Omegaup,
        text-micro={\fontfamily{mdbch}\textmu},
        text-ohm  ={\fontfamily{mdbch}\textohm}
}

\usepackage{attachfile2}
\attachfilesetup{color=blue}


\usepackage[protrusion=true,expansion,kerning=true,final,verbose=false,babel=false]{microtype}
\DisableLigatures{encoding=T1,family=tt*}


% corrige l'interaction foireuse entre tikz et babel[french]
\AtBeginEnvironment{tikzpicture}{\shorthandoff{:;!}}
\AtBeginEnvironment{pgfpicture}{\shorthandoff{:;!}}


% This is not the standard way to load a tikzlibrary
% but I want to make sure that the documentation loads
% the corresponding library (and not e.g. the production version).
\iffalse
The tikzoptics library

Copyright (C) 2013-2016 by
  Michel Fruchart <michel.fruchart@ens-lyon.org>

This work may be distributed and/or modified under the conditions of
the LaTeX Project Public License (LPPL), version 1.3c, which can be
found at the address:
https://www.latex-project.org/lppl/lppl-1-3c/

Alternatively, it may be distributed and/or modified under the conditions of
the GNU General Public License (GNU GPL), version 2, which can be found
at the address:
https://www.gnu.org/licenses/gpl-2.0.en.html
or any later version published by the Free Software Foundation.
\fi
\makeatletter


\def\tikzopticsversion{0.2.3}
\def\tikzopticsversiondate{2017-03-11}


%%%%%%%%%%%%%%%%%%%%%%%%%%%%%%%%%%%%%%%%%%%%%%%%%%%%%%%%%%%%%%%%%%%%%%%%%%%%%%%%
% errors from this library
%%%%%%%%%%%%%%%%%%%%%%%%%%%%%%%%%%%%%%%%%%%%%%%%%%%%%%%%%%%%%%%%%%%%%%%%%%%%%%%%
\def\opticserror#1{\pgfutil@packageerror{tikz/optics}{#1}{}}


%%%%%%%%%%%%%%%%%%%%%%%%%%%%%%%%%%%%%%%%%%%%%%%%%%%%%%%%%%%%%%%%%%%%%%%%%%%%%%%%
% handler |collect unknowns|
%%%%%%%%%%%%%%%%%%%%%%%%%%%%%%%%%%%%%%%%%%%%%%%%%%%%%%%%%%%%%%%%%%%%%%%%%%%%%%%%
% from http://tex.stackexchange.com/questions/81821/filtering-options-with-pgfkeys/81985#81985
% used by /tikz/optics/->n-
%
\pgfkeys{
  /handlers/.collect unknowns/.style = {
    unknown options/.initial = {},
    .unknown/.code = {%
      \letcs\reserved{pgfk@\pgfkeyscurrentpath/unknown options}%
      \csedef{pgfk@\pgfkeyscurrentpath/unknown options}{%
        \ifx\reserved\empty\else\expandonce\reserved,\fi
        \expandonce\pgfkeyscurrentname
        \ifx\pgfkeysnovalue##1\else=\expandonce\pgfkeyscurrentvalue\fi
      }%
    }
  }
}

%%%%%%%%%%%%%%%%%%%%%%%%%%%%%%%%%%%%%%%%%%%%%%%%%%%%%%%%%%%%%%%%%%%%%%%%%%%%%%%%
% defining |optics| family
%%%%%%%%%%%%%%%%%%%%%%%%%%%%%%%%%%%%%%%%%%%%%%%%%%%%%%%%%%%%%%%%%%%%%%%%%%%%%%%%

\pgfkeys{
  /tikz/.cd,
  optics/.is family,
  optics/.search also={/tikz},
}

%%%%%%%%%%%%%%%%%%%%%%%%%%%%%%%%%%%%%%%%%%%%%%%%%%%%%%%%%%%%%%%%%%%%%%%%%%%%%%%%
% [use optics] 
%%%%%%%%%%%%%%%%%%%%%%%%%%%%%%%%%%%%%%%%%%%%%%%%%%%%%%%%%%%%%%%%%%%%%%%%%%%%%%%%
\pgfkeys{
  /tikz/use optics/.append code={
  \pgfkeys{
    /tikz/.cd,
      % shapes 
      % FIXME il ne devrait pas y avoir de logique (ni de valeurs par défaut) ici, seulement l'export dans l'espace de nom commun
      lens/.prefix style={shape=lens,optics,draw},
      slit/.prefix style={shape=slit,optics,draw},
      double slit/.prefix style={shape=double slit,optics,draw},
      thin optics element/.prefix style={shape=thin optics element,optics,draw},
      polarizer/.prefix style={shape=polarizer,optics,draw,object aspect ratio=0.1},
      generic optics io/.prefix style={shape=generic optics io,optics,draw},
      sensor/.prefix style={shape=sensor,optics,object aspect ratio=1,draw},
      sensor line/.prefix style={shape=sensor line,optics,draw},
      mirror/.prefix style={shape=mirror,optics,draw},
      spherical mirror/.prefix style={shape=spherical mirror,optics,draw},
      thick optics element/.prefix style={shape=thick optics element,optics,draw,object aspect ratio=0.1},
      heat filter/.prefix style={shape=thick optics element,optics,draw, object aspect ratio=0.05},
      double amici prism/.prefix style={shape=double amici prism,optics,draw, prism height=1cm, prism apex angle=60},
      % styles
      screen/.prefix style={optics,/tikz/optics/screen},
      diffraction grating/.style={optics,/tikz/optics/diffraction grating},
      grid/.style={optics, /tikz/optics/grid},
      semi-transparent mirror/.style={optics, /tikz/optics/semi-transparent mirror},
      diaphragm/.style={optics, /tikz/optics/diaphragm},
      beam splitter/.prefix style={optics,/tikz/optics/beam splitter},
      generic lamp/.prefix style={optics,/tikz/optics/generic lamp},
      generic sensor/.prefix style={optics,/tikz/optics/generic sensor},
      halogen lamp/.prefix style={optics,/tikz/optics/halogen lamp},
      spectral lamp/.prefix style={optics,/tikz/optics/spectral lamp},
      laser/.prefix style={optics,/tikz/optics/laser},
      laser'/.prefix style={optics,/tikz/optics/laser'},
      concave mirror/.prefix style={optics, /tikz/optics/concave mirror},
      convex mirror/.prefix style={optics, /tikz/optics/convex mirror},
      % arrows (styles)
      ->-/.prefix style={optics,/tikz/optics/->-={##1}},
      -<-/.prefix style={optics,/tikz/optics/-<-={##1}},
      ->>-/.prefix style={optics,/tikz/optics/->>-={##1}},
      -<<-/.prefix style={optics,/tikz/optics/-<<-={##1}},
      ->>>-/.prefix style={optics,/tikz/optics/->>>-={##1}},
      -<<<-/.prefix style={optics,/tikz/optics/-<<<-={##1}},
      ->>>>-/.prefix style={optics,/tikz/optics/->>>>-={##1}},
      -<<<<-/.prefix style={optics,/tikz/optics/-<<<<-={##1}},
      ->n-/.prefix style={optics,/tikz/optics/->n-={##1}},
      -<n-/.prefix style={optics,/tikz/optics/-<n-={##1}},
    }
  }
}

%%%%%%%%%%%%%%%%%%%%%%%%%%%%%%%%%%%%%%%%%%%%%%%%%%%%%%%%%%%%%%%%%%%%%%%%%%%%%%%%
% We need some existing tikz libraries and tex packages.
%%%%%%%%%%%%%%%%%%%%%%%%%%%%%%%%%%%%%%%%%%%%%%%%%%%%%%%%%%%%%%%%%%%%%%%%%%%%%%%%
\usetikzlibrary{decorations,decorations.markings, decorations.pathreplacing}
\usepackage{etoolbox}


%%%%%%%%%%%%%%%%%%%%%%%%%%%%%%%%%%%%%%%%%%%%%%%%%%%%%%%%%%%%%%%%%%%%%%%%%%%%%%%%
% Key |object height|
%%%%%%%%%%%%%%%%%%%%%%%%%%%%%%%%%%%%%%%%%%%%%%%%%%%%%%%%%%%%%%%%%%%%%%%%%%%%%%%%

\pgfkeys{/tikz/optics/.cd,
  object height/.initial=2cm,
}

%%%%%%%%%%%%%%%%%%%%%%%%%%%%%%%%%%%%%%%%%%%%%%%%%%%%%%%%%%%%%%%%%%%%%%%%%%%%%%%%
% Shape [thin optics element]
%%%%%%%%%%%%%%%%%%%%%%%%%%%%%%%%%%%%%%%%%%%%%%%%%%%%%%%%%%%%%%%%%%%%%%%%%%%%%%%%
\pgfdeclareshape{thin optics element}
{
  \savedanchor{\center}{
    \pgfpointorigin
  }
  \anchor{center}{\center}

  \savedmacro\objectHeight{%
    \edef\objectHeight{\pgfkeysvalueof{/tikz/optics/object height}}%
  }

  \savedanchor{\north}{
    \pgf@x=0cm%
    \pgf@y=\objectHeight%
    \pgf@y=0.5\pgf@y%
  }
  \anchor{north}{\north}

  \savedanchor{\south}{
    \pgf@x=0cm%
    \pgf@y=\objectHeight%
    \pgf@y=-0.5\pgf@y%
  }
  \anchor{south}{\south}

  \anchor{east}{\center}
  \anchor{west}{\center}

  \anchorborder{%
    \pgf@xb=\pgf@x% xb/yb is target
    \pgf@yb=\pgf@y%
    \south%
    \pgf@xa=\pgf@x% xa/ya is se
    \pgf@ya=\pgf@y%
    \north%
    \advance\pgf@x by-\pgf@xa%
    \advance\pgf@y by-\pgf@ya%
    \pgf@xc=.5\pgf@x% x/y is half width/height
    \pgf@yc=.5\pgf@y%
    \advance\pgf@xa by\pgf@xc% xa/ya becomes center
    \advance\pgf@ya by\pgf@yc%
    \edef\pgf@marshal{%
      \noexpand\pgfpointborderrectangle
      {\noexpand\pgfpoint{\the\pgf@xb}{\the\pgf@yb}}
      {\noexpand\pgfpoint{\the\pgf@xc}{\the\pgf@yc}}%
    }%
    \pgf@process{\pgf@marshal}%
    \advance\pgf@x by\pgf@xa%
    \advance\pgf@y by\pgf@ya%
  }

  \backgroundpath
  {
    \north \pgf@xa=\pgf@x \pgf@ya=\pgf@y
    \south \pgf@xb=\pgf@x \pgf@yb=\pgf@y

    \pgfpathmoveto{\pgfpoint{\pgf@xa}{\pgf@ya}}
    \pgfpathlineto{\pgfpoint{\pgf@xb}{\pgf@yb}}
  }
}


%%%%%%%%%%%%%%%%%%%%%%%%%%%%%%%%%%%%%%%%%%%%%%%%%%%%%%%%%%%%%%%%%%%%%%%%%%%%%%%%
% Shape [lens]
%%%%%%%%%%%%%%%%%%%%%%%%%%%%%%%%%%%%%%%%%%%%%%%%%%%%%%%%%%%%%%%%%%%%%%%%%%%%%%%%
\newif\iftikz@optics@lens@converging
\tikz@optics@lens@convergingtrue

% keys
% - focal lens is used to define anchors |east focus| and |west focus|
% - lens height is used to define anchors |lens north| and |lens south|
% - lens type (converging or diverging)
\pgfkeys{/tikz/optics/.cd,
  focal length/.initial=1cm,
  lens height/.initial=0.8,
  lens type/.is choice,
  lens type/converging/.code={\tikz@optics@lens@convergingtrue},
  lens type/diverging/.code={\tikz@optics@lens@convergingfalse}
}
\pgfdeclareshape{lens}
{
  \savedanchor{\center}{
    \pgfpointorigin
  }
  \anchor{center}{\center}

  \savedmacro\focalLength{%
    \edef\focalLength{\pgfkeysvalueof{/tikz/optics/focal length}}%
  }

  \savedmacro\objectHeight{%
    \edef\objectHeight{\pgfkeysvalueof{/tikz/optics/object height}}%
  }

  \savedmacro\lensHeight{%
    \pgfmathparse{\pgfkeysvalueof{/tikz/optics/lens height}}%
    \ifpgfmathunitsdeclared%
      \pgfmathsetlengthmacro{\lensHeight}{\pgfkeysvalueof{/tikz/optics/lens height}}%
    \else%
      \pgfmathsetlengthmacro{\lensHeight}{\pgfkeysvalueof{/tikz/optics/lens height}*\pgfkeysvalueof{/tikz/optics/object height}}%
    \fi%
  }

  \savedanchor{\lensnorth}{
    \pgfpointorigin
    \pgf@y=\lensHeight%
    \pgf@y=0.5\pgf@y%
  }
  \anchor{lens north}{\lensnorth}

  \savedanchor{\lenssouth}{
    \pgfpointorigin
    \pgf@y=\lensHeight%
    \pgf@y=-0.5\pgf@y%
  }
  \anchor{lens south}{\lenssouth}

  \savedanchor{\north}{
    \pgf@x=0cm%
    \pgf@y=\objectHeight%
    \pgf@y=0.5\pgf@y%
  }
  \anchor{north}{\north}

  \savedanchor{\south}{
    \pgf@x=0cm%
    \pgf@y=\objectHeight%
    \pgf@y=-0.5\pgf@y%
  }
  \anchor{south}{\south}

  \savedanchor{\eastfocal}{
    \pgf@x=\focalLength%
    \pgf@y=0cm%
  }
  \anchor{east focal point}{\eastfocal}
  \anchor{east focus}{\eastfocal}
  

  \savedanchor{\westfocal}{
    \pgf@x=-\focalLength%
    \pgf@y=0cm%
  }
  \anchor{west focal point}{\westfocal}
  \anchor{west focus}{\westfocal}

  \anchor{east}{\center}
  \anchor{west}{\center}

  \inheritanchorborder[from=thin optics element]

  \backgroundpath
  {
    \north \pgf@xb=\pgf@x \pgf@yb=\pgf@y
    \south \pgf@xa=\pgf@x \pgf@ya=\pgf@y
    
    \pgfpathmoveto{\pgfpoint{\pgf@xa}{\pgf@ya}}
    \pgfpathlineto{\pgfpoint{\pgf@xb}{\pgf@yb}}

    \iftikz@optics@lens@converging
      \pgfsetarrowsstart{lens arrow}
      \pgfsetarrowsend{lens arrow}
    \else
      \pgfsetarrowsstart{lens arrow reversed}
      \pgfsetarrowsend{lens arrow reversed}
    \fi
  }
}


%%%%%%%%%%%%%%%%%%%%%%%%%%%%%%%%%%%%%%%%%%%%%%%%%%%%%%%%%%%%%%%%%%%%%%%%%%%%%%%%
% Shape [slit]
%%%%%%%%%%%%%%%%%%%%%%%%%%%%%%%%%%%%%%%%%%%%%%%%%%%%%%%%%%%%%%%%%%%%%%%%%%%%%%%%
% keys : slit height
\pgfkeys{/tikz/optics/.cd,
  slit height/.initial=0.1
}

\pgfdeclareshape{slit}
{
  \savedanchor{\center}{
    \pgfpointorigin
  }
  \anchor{center}{\center}
  \anchor{slit center}{\center} 
  % TODO par cohérence ?

  \savedmacro\objectHeight{%
    \edef\objectHeight{\pgfkeysvalueof{/tikz/optics/object height}}%
  }

  \savedmacro\slitHeight{%
    \pgfmathparse{\pgfkeysvalueof{/tikz/optics/slit height}}%
    \ifpgfmathunitsdeclared%
      \pgfmathsetlengthmacro{\slitHeight}{\pgfkeysvalueof{/tikz/optics/slit height}}%
    \else%
      \pgfmathsetlengthmacro{\slitHeight}{\pgfkeysvalueof{/tikz/optics/slit height}*\pgfkeysvalueof{/tikz/optics/object height}}%
    \fi%
  }

  \savedanchor{\north}{
    \pgf@x=0cm%
    \pgf@y=\objectHeight%
    \pgf@y=0.5\pgf@y%
  }
  \anchor{north}{\north}

  \savedanchor{\south}{
    \pgf@x=0cm%
    \pgf@y=\objectHeight%
    \pgf@y=-0.5\pgf@y%
  }
  \anchor{south}{\south}

  \savedanchor{\slitnorth}{
    \pgf@x=0cm%
    \pgf@y=\slitHeight%
    \pgf@y=0.5\pgf@y%
  }
  \anchor{slit north}{\slitnorth}

  \savedanchor{\slitsouth}{
    \pgf@x=0cm%
    \pgf@y=\slitHeight%
    \pgf@y=-0.5\pgf@y%
  }
  \anchor{slit south}{\slitsouth}

  \anchor{east}{\center}
  \anchor{west}{\center}

  \inheritanchorborder[from=thin optics element]

  \backgroundpath
  {
    % erreurs possibles dans la spécification du dessin
    \pgfmathparse{notless(\slitHeight, \objectHeight)}
    \ifnum\pgfmathresult=1
      \opticserror{<slit height> should be strictly lower than <object height> (in slit)}
    \fi
    \north \pgf@xa=\pgf@x \pgf@ya=\pgf@y
    \slitnorth \pgf@xb=\pgf@x \pgf@yb=\pgf@y
    \pgfpathmoveto{\pgfpoint{\pgf@xa}{\pgf@ya}}
    \pgfpathlineto{\pgfpoint{\pgf@xb}{\pgf@yb}}

    \south \pgf@xa=\pgf@x \pgf@ya=\pgf@y
    \slitsouth \pgf@xb=\pgf@x \pgf@yb=\pgf@y
    \pgfpathmoveto{\pgfpoint{\pgf@xa}{\pgf@ya}}
    \pgfpathlineto{\pgfpoint{\pgf@xb}{\pgf@yb}}
  }
}




%%%%%%%%%%%%%%%%%%%%%%%%%%%%%%%%%%%%%%%%%%%%%%%%%%%%%%%%%%%%%%%%%%%%%%%%%%%%%%%%
% Shape [double slit]
%%%%%%%%%%%%%%%%%%%%%%%%%%%%%%%%%%%%%%%%%%%%%%%%%%%%%%%%%%%%%%%%%%%%%%%%%%%%%%%%
% keys :
% - slit height : (relative) height of the holes (each)
% - slit separation : (relative) distance between the centers of the holes
\pgfkeys{/tikz/optics/.cd,
  object height/.initial=2cm,
  slit height/.initial=0.075,
  slit separation/.initial=0.2
}

\pgfdeclareshape{double slit}
{
  \savedanchor{\center}{
    \pgfpointorigin
  }
  \anchor{center}{\center}
  
  \savedmacro\objectHeight{%
    \edef\objectHeight{\pgfkeysvalueof{/tikz/optics/object height}}%
  }

  \savedmacro\slitHeight{%
    \pgfmathparse{\pgfkeysvalueof{/tikz/optics/slit height}}%
    \ifpgfmathunitsdeclared%
      \pgfmathsetlengthmacro{\slitHeight}{\pgfkeysvalueof{/tikz/optics/slit height}}%
    \else%
      \pgfmathsetlengthmacro{\slitHeight}{\pgfkeysvalueof{/tikz/optics/slit height}*\pgfkeysvalueof{/tikz/optics/object height}}%
    \fi%
  }

  \savedmacro\slitSeparation{%
    \pgfmathparse{\pgfkeysvalueof{/tikz/optics/slit separation}}%
    \ifpgfmathunitsdeclared%
      \pgfmathsetlengthmacro{\slitSeparation}{\pgfkeysvalueof{/tikz/optics/slit separation}}%
    \else%
      \pgfmathsetlengthmacro{\slitSeparation}{\pgfkeysvalueof{/tikz/optics/slit separation}*\pgfkeysvalueof{/tikz/optics/object height}}%
    \fi%
  }

  \savedmacro\macro@slitOneCenter{
    \def\macro@slitOneCenter{
      \pgfpointorigin
      \pgf@ya=\slitSeparation
      \pgf@ya=0.5\pgf@ya
      \advance \pgf@y by \pgf@ya
    }
  }

  \savedmacro\macro@slitTwoCenter{
    \def\macro@slitTwoCenter{
      \pgfpointorigin
      \pgf@ya=\slitSeparation
      \pgf@ya=-0.5\pgf@ya
      \advance \pgf@y by \pgf@ya
    }
  }

  \savedanchor{\north}{
    \pgf@x=0cm%
    \pgf@y=\objectHeight%
    \pgf@y=0.5\pgf@y%
  }
  \anchor{north}{\north}

  \savedanchor{\south}{
    \pgf@x=0cm%
    \pgf@y=\objectHeight%
    \pgf@y=-0.5\pgf@y%
  }
  \anchor{south}{\south}

  % slit 1
  \savedanchor{\slitOneCenter}{
    \macro@slitOneCenter
  }
  \anchor{slit 1 center}{\slitOneCenter}

  \savedanchor{\slitOneNorth}{
    \macro@slitOneCenter
    \pgf@ya = \slitHeight
    \pgf@ya = 0.5\pgf@ya
    \advance \pgf@y by \pgf@ya
  }
  \anchor{slit 1 north}{\slitOneNorth}

  \savedanchor{\slitOneSouth}{
    \macro@slitOneCenter
    \pgf@ya = \slitHeight
    \pgf@ya = -0.5\pgf@ya
    \advance \pgf@y by \pgf@ya
  }
  \anchor{slit 1 south}{\slitOneSouth}

  % slit 2
  \savedanchor{\slitTwoCenter}{
    \macro@slitTwoCenter
  }
  \anchor{slit 2 center}{\slitTwoCenter}

  \savedanchor{\slitTwoNorth}{
    \macro@slitTwoCenter
    \pgf@ya = \slitHeight
    \pgf@ya = 0.5\pgf@ya
    \advance \pgf@y by \pgf@ya
  }
  \anchor{slit 2 north}{\slitTwoNorth}

  \savedanchor{\slitTwoSouth}{
    \macro@slitTwoCenter
    \pgf@ya = \slitHeight
    \pgf@ya = -0.5\pgf@ya
    \advance \pgf@y by \pgf@ya
  }
  \anchor{slit 2 south}{\slitTwoSouth}

  \anchor{east}{\center}
  \anchor{west}{\center}

  \inheritanchorborder[from=thin optics element]

  \backgroundpath
  {
    % erreurs possibles dans les spécifications du dessin
    \pgfmathparse{notgreater(\slitSeparation, \slitHeight)}
    \ifnum\pgfmathresult=1
      \opticserror{<slit separation> should be strictly lower than <slit height> (in double slit)}
    \fi
    \pgfmathparse{notless(\slitSeparation+\slitHeight, \objectHeight)}
    \ifnum\pgfmathresult=1
      \opticserror{<slit height> plus <slit separation> should be strictly lower than <object height> (in double slit)}
    \fi
    \north \pgf@xa=\pgf@x \pgf@ya=\pgf@y
    \slitOneNorth \pgf@xb=\pgf@x \pgf@yb=\pgf@y
    \pgfpathmoveto{\pgfpoint{\pgf@xa}{\pgf@ya}}
    \pgfpathlineto{\pgfpoint{\pgf@xb}{\pgf@yb}}

    \slitOneSouth \pgf@xa=\pgf@x \pgf@ya=\pgf@y
    \slitTwoNorth \pgf@xb=\pgf@x \pgf@yb=\pgf@y
    \pgfpathmoveto{\pgfpoint{\pgf@xa}{\pgf@ya}}
    \pgfpathlineto{\pgfpoint{\pgf@xb}{\pgf@yb}}

    \slitTwoSouth \pgf@xa=\pgf@x \pgf@ya=\pgf@y
    \south \pgf@xb=\pgf@x \pgf@yb=\pgf@y
    \pgfpathmoveto{\pgfpoint{\pgf@xa}{\pgf@ya}}
    \pgfpathlineto{\pgfpoint{\pgf@xb}{\pgf@yb}}
  }
}



%%%%%%%%%%%%%%%%%%%%%%%%%%%%%%%%%%%%%%%%%%%%%%%%%%%%%%%%%%%%%%%%%%%%%%%%%%%%%%%%
% Shape [mirror]
%%%%%%%%%%%%%%%%%%%%%%%%%%%%%%%%%%%%%%%%%%%%%%%%%%%%%%%%%%%%%%%%%%%%%%%%%%%%%%%%
% keys
% - mirror decoration separation
% - mirror decoration amplitude
\pgfkeys{/tikz/optics/.cd,
  object height/.initial=2cm,
  mirror decoration separation/.initial=0.15cm,
  mirror decoration amplitude/.initial=0.125cm,
}

\pgfdeclareshape{mirror}
{
  \savedanchor{\center}{
    \pgfpointorigin
  }
  \anchor{center}{\center}

  \savedmacro\objectHeight{%
    \edef\objectHeight{\pgfkeysvalueof{/tikz/optics/object height}}%
  }

  \savedanchor{\north}{
    \pgf@x=0cm%
    \pgf@y=\objectHeight%
    \pgf@y=0.5\pgf@y%
  }
  \anchor{north}{\north}

  \savedanchor{\south}{
    \pgf@x=0cm%
    \pgf@y=\objectHeight%
    \pgf@y=-0.5\pgf@y%
  }
  \anchor{south}{\south}

  \anchor{east}{\center}
  \anchor{west}{\center}

  % bof
  \inheritanchorborder[from=thin optics element]

  \backgroundpath
  {
    \north \pgf@xa=\pgf@x \pgf@ya=\pgf@y
    \south \pgf@xb=\pgf@x \pgf@yb=\pgf@y
    % We use a border decoration to make a mirror (hachures).
    % First set the decoration parameters :
    % - decoration angle
    % l'angle est 45-180 et on multiplie l'amplitude par -1 pour que la décoration commence et finisse par un trait (c'est un peu louche mais bon)
    \def\pgfdecorationsegmentangle{45-180}%
    % - decoration step length
    \pgfmathparse{\pgfkeysvalueof{/tikz/optics/mirror decoration separation}}
    \ifpgfmathunitsdeclared%
      \pgfmathsetlengthmacro{\pgfdecorationsegmentlength}{\pgfkeysvalueof{/tikz/optics/mirror decoration separation}}%
    \else%
      \pgfmathsetlengthmacro{\pgfdecorationsegmentlength}{\pgfkeysvalueof{/tikz/optics/mirror decoration separation}*\pgfkeysvalueof{/tikz/optics/object height}}%
    \fi%
    % magouillons pour que \pgfdecorationsegmentlength soit un multiple de la longueur ...
    \pgfmathsetmacro\initialstep{\pgfdecorationsegmentlength} 
    \pgfmathsetmacro\totallength{\pgfkeysvalueof{/tikz/optics/object height}} 
    \pgfmathsetmacro\newstep{\totallength/floor(\totallength/\initialstep)} 
    \pgfmathsetlengthmacro{\pgfdecorationsegmentlength}{\newstep}
    % fin magouille
    % - decoration amplitude
    % on multiplie par -1 pour que ça aille dans le bon sens (pour éviter des problèmes louches)
    \pgfmathparse{\pgfkeysvalueof{/tikz/optics/mirror decoration amplitude}}
    \ifpgfmathunitsdeclared%
      \pgfmathsetlengthmacro{\pgfdecorationsegmentamplitude}{-1*\pgfkeysvalueof{/tikz/optics/mirror decoration amplitude}}%
    \else%
      \pgfmathsetlengthmacro{\pgfdecorationsegmentamplitude}{-1*\pgfkeysvalueof{/tikz/optics/mirror decoration amplitude}*\pgfkeysvalueof{/tikz/optics/object height}}%
    \fi%
    % Use decoration.
    \pgfpathmoveto{\pgfpoint{\pgf@xa}{\pgf@ya}}
    \pgfpathlineto{\pgfpoint{\pgf@xb}{\pgf@yb}}
    \pgfdecoratecurrentpath{border} %
    % dessin du miroir (trait)
    \pgfpathmoveto{\pgfpoint{\pgf@xa}{\pgf@ya}}
    \pgfpathlineto{\pgfpoint{\pgf@xb}{\pgf@yb}}
  }
}

%%%%%%%%%%%%%%%%%%%%%%%%%%%%%%%%%%%%%%%%%%%%%%%%%%%%%%%%%%%%%%%%%%%%%%%%%%%%%%%%
% Shape [spherical mirror]
%%%%%%%%%%%%%%%%%%%%%%%%%%%%%%%%%%%%%%%%%%%%%%%%%%%%%%%%%%%%%%%%%%%%%%%%%%%%%%%%
% keys :
% - spherical mirror angle : the aperture angle of the mirror
% - spherical mirror type (concave or convex)
% - spherical mirror orientation (ltr or rtl)
% The properties of the decoration are controled by the same parameters as in [mirror].
% Height is controled by |object height|, as usual.
\pgfkeys{/tikz/optics/spherical mirror angle/.initial=150}

\newif\iftikz@optics@sphericalmirror@concave
\tikz@optics@sphericalmirror@concavetrue
\pgfkeys{/tikz/optics/.cd,
  spherical mirror type/.is choice,
  spherical mirror type/concave/.code={\tikz@optics@sphericalmirror@concavetrue},
  spherical mirror type/convex/.code={\tikz@optics@sphericalmirror@concavefalse}
}

\newif\iftikz@optics@sphericalmirror@ltr
\tikz@optics@sphericalmirror@ltrtrue
\pgfkeys{/tikz/optics/.cd,
  spherical mirror orientation/.is choice,
  spherical mirror orientation/ltr/.code={\tikz@optics@sphericalmirror@ltrtrue},
  spherical mirror orientation/rtl/.code={\tikz@optics@sphericalmirror@ltrfalse}
}

% shortcuts (styles) : [concave mirror] and [convex mirror]
\pgfkeys{/tikz/optics/.cd,
      concave mirror/.style={optics, spherical mirror, /tikz/optics/spherical mirror type=concave},
      convex mirror/.style={optics, spherical mirror, /tikz/optics/spherical mirror type=convex},
}

\pgfdeclareshape{spherical mirror}{%
  \savedmacro\installsphericalmirrorparameters{%
    %
    % Define a \centerpoint
    %
    \pgfextract@process\centerpoint{%
      \pgfpointorigin
    }%
    %
    % Define height
    %
    \pgfmathsetlengthmacro\height{\pgfkeysvalueof{/tikz/optics/object height}}%
    %
    % Function to define angle from radius
    % use e.g. [spherical mirror angle=from_radius(2cm)] instead of e.g. [spherical mirror angle=90]
    % This can seem somewhat ridiculous as we will undo this when calculating \radius, however it helps providing a simpler API.
    % This function has to be defined before parsing \pgfkeysvalueof{/tikz/optics/spherical mirror angle}.
    %
    \pgfmathdeclarefunction{from_radius}{1}{
      \begingroup
      \pgfmathparse{notless(2*#1,\height)}
      \ifnum\pgfmathresult=0
        \opticserror{(in /tikz/optics/spherical mirror angle=from_radius(R)) : for a spherical mirror, the radius R cannot be smaller than half the height </tikz/optics/object height>. Set a bigger radius of a smaller height.}
      \fi
      \newdimen\angle
      \pgfmathsetlength\angle{2*asin(\height/(2*#1))}
      \pgf@x=\angle
      \pgfmathreturn\pgf@x
      \endgroup
    }
    %
    % Define angle
    %
    \pgfmathsetmacro\angle{\pgfkeysvalueof{/tikz/optics/spherical mirror angle}}
    %
    % Compute radius from height and angle
    %
    \pgfmathsetlengthmacro\radius{\height/(2*sin(\angle/2))}
    %
    % Half of the sector angle is more useful.
    %
    \pgfmathmod{\angle}{360}%
    \ifdim\pgfmathresult pt<0pt\relax%
      \pgfmathadd@{\pgfmathresult}{360}%
    \fi%
    \let\angle\pgfmathresult%
    \pgfmathdivide@{\pgfmathresult}{2}%
    \let\halfangle\pgfmathresult%
    %
    % Get the start and end angles of the arc.
    %
    \iftikz@optics@sphericalmirror@concave
      \iftikz@optics@sphericalmirror@ltr
        \pgfmathsetmacro\startangle{-\halfangle}
        \pgfmathsetmacro\endangle{+\halfangle}
      \else
        \pgfmathsetmacro\startangle{180-\halfangle}
        \pgfmathsetmacro\endangle{180+\halfangle}
      \fi
      \else
      \iftikz@optics@sphericalmirror@ltr
        \pgfmathsetmacro\startangle{180-\halfangle}
        \pgfmathsetmacro\endangle{180+\halfangle}
      \else
        \pgfmathsetmacro\startangle{-\halfangle}
        \pgfmathsetmacro\endangle{+\halfangle}
      \fi
    \fi
    % 
    % Calculate R cos(angle/2) and R sin(angle/2)
    % 
    \pgfmathabs@{\halfangle}%
    \pgfmathcos@{\pgfmathresult}%
    \let\coshalfangle\pgfmathresult%
    \pgfmathabs@{\halfangle}%
    \pgfmathsin@{\pgfmathresult}%
    \let\sinhalfangle\pgfmathresult%
    \pgfmathsetlength\pgf@xa{\radius*\coshalfangle}
    \edef\rcoshalfangle{\the\pgf@xa}%
    \pgfmathsetlength\pgf@xa{\radius*\sinhalfangle}
    \edef\rsinhalfangle{\the\pgf@xa}%
    %
    % Calculate the arc coordinates
    %
    \pgfextract@process\arcstart{%
      \pgfqpointpolar{\startangle}{\radius}%
      \pgf@xa\pgf@x%
      \pgf@ya\pgf@y%
      \centerpoint%
      \advance\pgf@x\pgf@xa%
      \advance\pgf@y\pgf@ya%
    }%
    \pgfextract@process\arcend{%
      \pgfqpointpolar{\endangle}{\radius}%
      \pgf@xa\pgf@x%
      \pgf@ya\pgf@y%
      \centerpoint%
      \advance\pgf@x\pgf@xa%
      \advance\pgf@y\pgf@ya%
    }%
    \def\convexrtlsetx#1#2{
      \iftikz@optics@sphericalmirror@concave%
        \iftikz@optics@sphericalmirror@ltr%
          \advance\pgf@x by #1
        \else% % => rtl
          \advance\pgf@x by #2
        \fi%
      \else% % => convex
        \iftikz@optics@sphericalmirror@ltr%
          \advance\pgf@x by #2
        \else% % => rtl
          \advance\pgf@x by #1
        \fi%
      \fi%
    }
    \def\convexrtlinvert{%
      \iftikz@optics@sphericalmirror@concave%
        \iftikz@optics@sphericalmirror@ltr%
          %nothing
        \else% % => rtl
          \pgf@x=-\pgf@x%
        \fi%
      \else% % => convex
        \iftikz@optics@sphericalmirror@ltr%
          \pgf@x=-\pgf@x%
        \else% % => rtl
          %nothing
        \fi%
      \fi%
    }%
    %
    % Save everything. 
    % "NB \addtosavedmacro is currently experimental. May get changed." d'après le code où je l'ai piqué
    %
    \addtosavedmacro{\radius}%
    %
    \addtosavedmacro{\rcoshalfangle}%
    \addtosavedmacro{\rsinhalfangle}%
    %
    \addtosavedmacro{\endangle}%
    \addtosavedmacro{\startangle}%
    %
    \addtosavedmacro{\centerpoint}%
    \addtosavedmacro{\arcstart}%
    \addtosavedmacro{\arcend}%
     \addtosavedmacro{\convexrtlinvert}%
    %
  }%
  %
  % Define anchors
  %
  \savedanchor\mirrorcenterpoint{%
    \pgfpointorigin
  }%
  \savedanchor\centerpoint{%
    \pgfpointorigin%
    \advance\pgf@x by \radius%
    \advance\pgf@x by \rcoshalfangle%
    \iftikz@optics@sphericalmirror@concave%
      \iftikz@optics@sphericalmirror@ltr%
        %nothing
      \else% % => rtl
        \pgf@x=-\pgf@x%
      \fi%
    \else% % => convex
      \iftikz@optics@sphericalmirror@ltr%
        \pgf@x=-\pgf@x%
      \else% % => rtl
        %nothing
      \fi%
    \fi%
    \divide\pgf@x by 2%
  }%
  \savedanchor\focalpoint{%
    \pgfpointorigin%
    \pgf@xa=\radius%
    \advance\pgf@x by .5\pgf@xa%
    \iftikz@optics@sphericalmirror@concave%
      \iftikz@optics@sphericalmirror@ltr%
        %nothing
      \else% % => rtl
        \pgf@x=-\pgf@x%
      \fi%
    \else% % => convex
      \iftikz@optics@sphericalmirror@ltr%
        \pgf@x=-\pgf@x%
      \else% % => rtl
        %nothing
      \fi%
    \fi%
  }%
  \savedanchor\north{%
    \pgfpointorigin
    \pgf@xa=0pt
    \advance\pgf@xa by \rcoshalfangle
    \advance\pgf@xa by \radius
    \divide\pgf@xa by 2
    \advance\pgf@x by \pgf@xa
    \advance\pgf@y by \rsinhalfangle
    \convexrtlinvert
  }%
  \savedanchor\arccenter{%
    \centerpoint
    \advance\pgf@x by \radius
    \convexrtlinvert
  }
  \savedanchor\south{%
    \pgfpointorigin
    \pgf@xa=0pt
    \advance\pgf@xa by \rcoshalfangle
    \advance\pgf@xa by \radius
    \divide\pgf@xa by 2
    \advance\pgf@x by \pgf@xa
    \advance\pgf@y by -\rsinhalfangle
    \convexrtlinvert
  }
  \savedanchor\east{%
    \pgfpointorigin
    \convexrtlsetx{\radius}{\rcoshalfangle}
    \convexrtlinvert
  }
  \savedanchor\west{%
    \pgfpointorigin
    \convexrtlsetx{\rcoshalfangle}{\radius}
    \convexrtlinvert
  }
  \savedanchor\northwest{%
    \pgfpointorigin
    \convexrtlsetx{\rcoshalfangle}{\radius}
    \advance\pgf@y by \rsinhalfangle
    \convexrtlinvert
  }
  \savedanchor\southwest{%
    \pgfpointorigin
    \convexrtlsetx{\rcoshalfangle}{\radius}
    \advance\pgf@y by -\rsinhalfangle
    \convexrtlinvert
  }
  \savedanchor\northeast{%
    \pgfpointorigin
    \convexrtlsetx{\radius}{\rcoshalfangle}
    \advance\pgf@y by \rsinhalfangle
    \convexrtlinvert
  }
  \savedanchor\southeast{%
    \pgfpointorigin
    \convexrtlsetx{\radius}{\rcoshalfangle}
    \advance\pgf@y by -\rsinhalfangle
    \convexrtlinvert
  }
  \anchor{arc start}{%
    \installsphericalmirrorparameters%
    \arcstart%
  }
  \anchor{arc end}{%
    \installsphericalmirrorparameters%
    \arcend%
  }
  \anchor{focal point}{%
    \installsphericalmirrorparameters%
    \focalpoint
  }
  \anchor{focus}{%
    \installsphericalmirrorparameters%
    \focalpoint
  }
  \anchor{mirror center}{\mirrorcenterpoint}  
  \anchor{center}{\centerpoint} 
  \anchor{arc center}{\arccenter}
  \anchor{north}{\north}%
  \anchor{south}{\south}%
  \anchor{east}{\east}%
  \anchor{west}{\west}%
  \anchor{north west}{\northwest}%
  \anchor{south west}{\southwest}%
  \anchor{north east}{\northeast}%
  \anchor{south east}{\southeast}%
  %
  % Draw backgroundpath
  %
  \backgroundpath{%
    \installsphericalmirrorparameters%
    % We use a border decoration to make a mirror.
    % First set the decoration parameters :
    % - decoration angle
    \iftikz@optics@sphericalmirror@concave
      \def\pgfdecorationsegmentangle{45}%
    \else
      \def\pgfdecorationsegmentangle{-90-45}%
    \fi
    % - decoration step length
    \pgfmathparse{\pgfkeysvalueof{/tikz/optics/mirror decoration separation}}
    \ifpgfmathunitsdeclared%
      \pgfmathsetlengthmacro{\pgfdecorationsegmentlength}{\pgfkeysvalueof{/tikz/optics/mirror decoration separation}}%
    \else%
      \pgfmathsetlengthmacro{\pgfdecorationsegmentlength}{\pgfkeysvalueof{/tikz/optics/mirror decoration separation}*\pgfkeysvalueof{/tikz/optics/object height}}%
    \fi%
    % magouillons pour que \pgfdecorationsegmentlength soit un multiple de la longueur ...
    \pgfmathsetmacro\initialstep{\pgfdecorationsegmentlength} 
    \pgfmathsetmacro\totallength{(2*pi/360)*\angle*\radius} 
    \pgfmathsetmacro\newstep{\totallength/floor(\totallength/\initialstep)} 
    \pgfmathsetlengthmacro{\pgfdecorationsegmentlength}{\newstep}
    % fin magouille
    % - decoration amplitude
    % on multiplie par -1 pour que ça aille dans le bon sens sans devoir modifier l'angle (pour éviter des problèmes louches)
    \pgfmathparse{\pgfkeysvalueof{/tikz/optics/mirror decoration amplitude}}
    \ifpgfmathunitsdeclared%
      \pgfmathsetlengthmacro{\pgfdecorationsegmentamplitude}{-1*\pgfkeysvalueof{/tikz/optics/mirror decoration amplitude}}%
    \else%
      \pgfmathsetlengthmacro{\pgfdecorationsegmentamplitude}{-1*\pgfkeysvalueof{/tikz/optics/mirror decoration amplitude}*\pgfkeysvalueof{/tikz/optics/object height}}%
    \fi%
    % Now use decoration.
    % Draw decoration of path : an arc of radius \radius from \arcstart to \arcend
    \pgfpathmoveto{\arcstart}%
    \pgfpatharc{\startangle}{\endangle}{\radius}%
    \pgfdecoratecurrentpath{border} %
    % Now draw the path.
    \pgfpathmoveto{\arcstart}%
    \pgfpatharc{\startangle}{\endangle}{\radius}%
  }
  %
  % Anchor border
  % This is needed for anchors .<angle> (like mirror.0, mirror.90, etc.) to work.
  %
  \anchorborder{%
    % Save x and y
    \edef\externalx{\the\pgf@x}%
    \edef\externaly{\the\pgf@y}%
    \installsphericalmirrorparameters%
    % Use circular border
    \pgfpointborderellipse{ \pgfpoint{\externalx}{\externaly} }{ \pgfpoint{\radius}{\radius} }%
  }%
}



%%%%%%%%%%%%%%%%%%%%%%%%%%%%%%%%%%%%%%%%%%%%%%%%%%%%%%%%%%%%%%%%%%%%%%%%%%%%%%%%
% Shape [thick optics element]
%%%%%%%%%%%%%%%%%%%%%%%%%%%%%%%%%%%%%%%%%%%%%%%%%%%%%%%%%%%%%%%%%%%%%%%%%%%%%%%%
% keys :
% - object aspect ratio (alias : object width)
% ar = largeur/hauteur => largeur = ar * hauteur
\pgfkeys{/tikz/optics/.cd,
  object aspect ratio/.initial=0.2,
  object width/.style={/tikz/optics/object aspect ratio=#1}
}

\pgfdeclareshape{thick optics element}
{
  \savedanchor{\center}{
    \pgfpointorigin
  }
  \anchor{center}{\center}

  \savedmacro\objectHeight{%
    \edef\objectHeight{\pgfkeysvalueof{/tikz/optics/object height}}%
  }

  \savedmacro\objectWidth{%
    \pgfmathparse{\pgfkeysvalueof{/tikz/optics/object aspect ratio}}
    \ifpgfmathunitsdeclared%
      \pgfmathsetlengthmacro{\objectWidth}{\pgfkeysvalueof{/tikz/optics/object aspect ratio}}%
    \else%
      \pgfmathsetlengthmacro{\objectWidth}{\pgfkeysvalueof{/tikz/optics/object aspect ratio}*\pgfkeysvalueof{/tikz/optics/object height}}%
    \fi%
  }

  \savedanchor{\northeast}{
    \pgf@x=\objectWidth%
    \pgf@y=\objectHeight%
    \pgf@y=0.5\pgf@y%
    \pgf@x=0.5\pgf@x%
  }
  \anchor{north east}{\northeast}

  \savedanchor{\southwest}{
    \pgf@x=\objectWidth%
    \pgf@y=\objectHeight%
    \pgf@y=-0.5\pgf@y%
    \pgf@x=-0.5\pgf@x%
  }
  \anchor{south west}{\southwest}

  \anchor{north}{
    \center \pgf@xa=\pgf@x \pgf@ya=\pgf@y
    \northeast \pgf@xb=\pgf@x \pgf@yb=\pgf@y
    \pgf@x=\pgf@xa
    \pgf@y=\pgf@yb
  }

  \anchor{south}{
    \center \pgf@xa=\pgf@x \pgf@ya=\pgf@y
    \southwest \pgf@xb=\pgf@x \pgf@yb=\pgf@y
    \pgf@x=\pgf@xa
    \pgf@y=\pgf@yb
  }

  \anchor{east}{
    \center \pgf@xa=\pgf@x \pgf@ya=\pgf@y
    \northeast \pgf@xb=\pgf@x \pgf@yb=\pgf@y
    \pgf@x=\pgf@xb
    \pgf@y=\pgf@ya
  }

  \anchor{west}{
    \center \pgf@xa=\pgf@x \pgf@ya=\pgf@y
    \southwest \pgf@xb=\pgf@x \pgf@yb=\pgf@y
    \pgf@x=\pgf@xb
    \pgf@y=\pgf@ya
  }

  \anchor{north west}{
    \northeast \pgf@xa=\pgf@x \pgf@ya=\pgf@y
    \southwest \pgf@xb=\pgf@x \pgf@yb=\pgf@y
    \pgf@x=\pgf@xb
    \pgf@y=\pgf@ya
  }

  \anchor{south east}{
    \northeast \pgf@xa=\pgf@x \pgf@ya=\pgf@y
    \southwest \pgf@xb=\pgf@x \pgf@yb=\pgf@y
    \pgf@x=\pgf@xa
    \pgf@y=\pgf@yb
  }

  \inheritanchorborder[from=rectangle]

  \backgroundpath
  {
    % rectangle
    \pgfpathrectanglecorners{\northeast}{\southwest}
  }
}



%%%%%%%%%%%%%%%%%%%%%%%%%%%%%%%%%%%%%%%%%%%%%%%%%%%%%%%%%%%%%%%%%%%%%%%%%%%%%%%%
% Shape [polarizer]
%%%%%%%%%%%%%%%%%%%%%%%%%%%%%%%%%%%%%%%%%%%%%%%%%%%%%%%%%%%%%%%%%%%%%%%%%%%%%%%%
\pgfdeclareshape{polarizer}
{
  \savedanchor{\center}{
    \pgfpointorigin
  }
  \anchor{center}{\center}

  \savedmacro\objectHeight{%
    \edef\objectHeight{\pgfkeysvalueof{/tikz/optics/object height}}%
  }

  \savedmacro\objectWidth{%
    \pgfmathparse{\pgfkeysvalueof{/tikz/optics/object aspect ratio}}
    \ifpgfmathunitsdeclared%
      \pgfmathsetlengthmacro{\objectWidth}{\pgfkeysvalueof{/tikz/optics/object aspect ratio}}%
    \else%
      \pgfmathsetlengthmacro{\objectWidth}{\pgfkeysvalueof{/tikz/optics/object aspect ratio}*\pgfkeysvalueof{/tikz/optics/object height}}%
    \fi%
  }

  \savedanchor{\northeast}{
    \pgf@x=\objectWidth%
    \pgf@y=\objectHeight%
    \pgf@y=0.5\pgf@y%
    \pgf@x=0.5\pgf@x%
  }
  \anchor{north east}{\northeast}

  \savedanchor{\southwest}{
    \pgf@x=\objectWidth%
    \pgf@y=\objectHeight%
    \pgf@y=-0.5\pgf@y%
    \pgf@x=-0.5\pgf@x%
  }
  \anchor{south west}{\southwest}

  \anchor{north}{
    \center \pgf@xa=\pgf@x \pgf@ya=\pgf@y
    \northeast \pgf@xb=\pgf@x \pgf@yb=\pgf@y
    \pgf@x=\pgf@xa
    \pgf@y=\pgf@yb
  }

  \anchor{south}{
    \center \pgf@xa=\pgf@x \pgf@ya=\pgf@y
    \southwest \pgf@xb=\pgf@x \pgf@yb=\pgf@y
    \pgf@x=\pgf@xa
    \pgf@y=\pgf@yb
  }

  \anchor{east}{
    \center \pgf@xa=\pgf@x \pgf@ya=\pgf@y
    \northeast \pgf@xb=\pgf@x \pgf@yb=\pgf@y
    \pgf@x=\pgf@xb
    \pgf@y=\pgf@ya
  }

  \anchor{west}{
    \center \pgf@xa=\pgf@x \pgf@ya=\pgf@y
    \southwest \pgf@xb=\pgf@x \pgf@yb=\pgf@y
    \pgf@x=\pgf@xb
    \pgf@y=\pgf@ya
  }

  \anchor{north west}{
    \northeast \pgf@xa=\pgf@x \pgf@ya=\pgf@y
    \southwest \pgf@xb=\pgf@x \pgf@yb=\pgf@y
    \pgf@x=\pgf@xb
    \pgf@y=\pgf@ya
  }

  \anchor{south east}{
    \northeast \pgf@xa=\pgf@x \pgf@ya=\pgf@y
    \southwest \pgf@xb=\pgf@x \pgf@yb=\pgf@y
    \pgf@x=\pgf@xa
    \pgf@y=\pgf@yb
  }

  \inheritanchorborder[from=rectangle]

  \backgroundpath
  {
    % rectangle
    \pgfpathrectanglecorners{\northeast}{\southwest}

    % diagonale du polariseur
    \northeast \pgf@xa=\pgf@x \pgf@ya=\pgf@y
    \southwest \pgf@xb=\pgf@x \pgf@yb=\pgf@y
    \pgfpathmoveto{\pgfpoint{\pgf@xa}{\pgf@ya}}
    \pgfpathlineto{\pgfpoint{\pgf@xb}{\pgf@yb}}
  }
}

%%%%%%%%%%%%%%%%%%%%%%%%%%%%%%%%%%%%%%%%%%%%%%%%%%%%%%%%%%%%%%%%%%%%%%%%%%%%%%%%
% Shape [double amici prism]
%%%%%%%%%%%%%%%%%%%%%%%%%%%%%%%%%%%%%%%%%%%%%%%%%%%%%%%%%%%%%%%%%%%%%%%%%%%%%%%%
% keys :
% - prism height
% - prism apex angle
\pgfkeys{/tikz/optics/.cd,
  prism height/.initial=1.5cm,
  prism apex angle/.initial=60,
}
% Idea : we get 
% apexAngle
% prismHeight
% demiPrismWidth = tan(apexAngle/2)*prismHeight
%
% so we compute
%
% C = (0,0)
% NE = (demiPrismWidth,0.5*prismHeight)
% SW = (-2*demiPrismWidth,-0.5*prismHeight)
% NW = (-demiPrismWidth,0.5*prismHeight)
% SE = (2*demiPrismWidth,-0.5*prismHeight)
%
% N = (0,0.5*prismHeight)
% S = (0,-0.5*prismHeight)
% E = (3*0.5*demiPrismWidth,0)
% W = (-3.0.5*demiPrismWidth,0)
\pgfdeclareshape{double amici prism}
{
  \savedanchor{\center}{
    \pgfpointorigin
  }
  \anchor{center}{\center}

  \savedmacro\prismHeight{%
    \edef\prismHeight{\pgfkeysvalueof{/tikz/optics/prism height}}%
  }

  \savedmacro\apexAngle{%
      \pgfmathsetlengthmacro{\apexAngle}{\pgfkeysvalueof{/tikz/optics/prism apex angle}}
  }

  \savedmacro\demiPrismWidth{%
      \pgfmathsetlengthmacro{\demiPrismWidth}{tan(0.5*\pgfkeysvalueof{/tikz/optics/prism apex angle})*\pgfkeysvalueof{/tikz/optics/prism height}}
  }

  \savedanchor{\northeast}{
    \pgf@x=\demiPrismWidth%
    \pgf@y=\prismHeight%
    \pgf@y=0.5\pgf@y%
  }
  \anchor{north east}{\northeast}

  \savedanchor{\southwest}{
    \pgf@x=\demiPrismWidth%
    \pgf@y=\prismHeight%
    \pgf@x=-2\pgf@x%
    \pgf@y=-0.5\pgf@y%
  }
  \anchor{south west}{\southwest}

  \savedanchor{\northwest}{
    \pgf@x=\demiPrismWidth%
    \pgf@y=\prismHeight%
    \pgf@x=-\pgf@x%
    \pgf@y=0.5\pgf@y%
  }
  \anchor{north west}{\northwest}

  \savedanchor{\southeast}{
    \pgf@x=\demiPrismWidth%
    \pgf@y=\prismHeight%
    \pgf@x=2\pgf@x%
    \pgf@y=-0.5\pgf@y%
  }
  \anchor{south east}{\southeast}

  \anchor{north}{
    \center \pgf@xa=\pgf@x \pgf@ya=\pgf@y
    \northeast \pgf@xb=\pgf@x \pgf@yb=\pgf@y
    \pgf@x=\pgf@xa
    \pgf@y=\pgf@yb
  }

  \savedanchor{\south}{
    \pgfpointorigin
    \pgf@y=\prismHeight%
    \pgf@y=-0.5\pgf@y%
  }
  \anchor{south}{\south}

  \anchor{east}{
    \southeast \pgf@xa=\pgf@x \pgf@ya=\pgf@y
    \northeast \pgf@xb=\pgf@x \pgf@yb=\pgf@y
    \pgf@x=\pgf@xa
    \advance\pgf@x by\pgf@xb
    \pgf@x=0.5\pgf@x
    \pgf@y=\pgf@ya
    \advance\pgf@y by\pgf@yb
    \pgf@y=0.5\pgf@y
  }

  \anchor{west}{
    \southwest \pgf@xa=\pgf@x \pgf@ya=\pgf@y
    \northwest \pgf@xb=\pgf@x \pgf@yb=\pgf@y
    \pgf@x=\pgf@xa
    \advance\pgf@x by\pgf@xb
    \pgf@x=0.5\pgf@x
    \pgf@y=\pgf@ya
    \advance\pgf@y by\pgf@yb
    \pgf@y=0.5\pgf@y
  }

   \inheritanchorborder[from=rectangle]

  \backgroundpath
  {
    \northwest
    \pgfpathmoveto{\pgfpoint{\pgf@x}{\pgf@y}}
    \southwest
    \pgfpathlineto{\pgfpoint{\pgf@x}{\pgf@y}}
    \southeast
    \pgfpathlineto{\pgfpoint{\pgf@x}{\pgf@y}}
    \northeast
    \pgfpathlineto{\pgfpoint{\pgf@x}{\pgf@y}}
    \pgfpathclose
    %
    % FIXME : ceci devrait s'appliquer seulement au triangle intérieur -> un fgpath ou assimilé
    \pgfsetbeveljoin
    \northwest
    \pgfpathmoveto{\pgfpoint{\pgf@x}{\pgf@y}}
    \south
    \pgfpathlineto{\pgfpoint{\pgf@x}{\pgf@y}}
    \northeast
    \pgfpathlineto{\pgfpoint{\pgf@x}{\pgf@y}}
  }
}






%%%%%%%%%%%%%%%%%%%%%%%%%%%%%%%%%%%%%%%%%%%%%%%%%%%%%%%%%%%%%%%%%%%%%%%%%%%%%%%%
% Shape [generic optics io]
%%%%%%%%%%%%%%%%%%%%%%%%%%%%%%%%%%%%%%%%%%%%%%%%%%%%%%%%%%%%%%%%%%%%%%%%%%%%%%%%
% keys :
% - io body height : height of the body of the io body
% - io body aspect ratio (alias io body width) : width/height of the body of the io body
% - io aperture width : width of the output device (condenser, etc.) [in units of io body height]
% - io aperture height : height of the output device (condenser, etc.) [in units of io body height]
% - io aperture shift : vertical shift of the output device [in units of io body height]
% - io orientation (ltr or rtl)
\pgfkeys{/tikz/optics/.cd,
  io body height/.initial=0.75cm,
  io body aspect ratio/.initial=2,
  io aperture width/.initial=0.33,
  io aperture height/.initial=0.66,
  io aperture shift/.initial=0,
}
\pgfkeys{/tikz/optics/io body width/.style={/tikz/optics/io body aspect ratio=#1}}
%
\newif\if@tikz@optics@io@ltr
\@tikz@optics@io@ltrtrue
%
\pgfkeys{/tikz/optics/io orientation/.is choice}
\pgfkeys{/tikz/optics/io orientation/ltr/.code={\@tikz@optics@io@ltrtrue}}
\pgfkeys{/tikz/optics/io orientation/rtl/.code={\@tikz@optics@io@ltrfalse}}
%
\pgfdeclareshape{generic optics io}
{
  \savedanchor{\center}{
    \pgfpointorigin
  }
  \anchor{center}{\center}

  \savedmacro\objectHeight{%
    \edef\objectHeight{\pgfkeysvalueof{/tikz/optics/io body height}}%
  }

  \savedmacro\objectWidth{%
    \pgfmathparse{\pgfkeysvalueof{/tikz/optics/io body aspect ratio}}
    \ifpgfmathunitsdeclared%
      \pgfmathsetlengthmacro{\objectWidth}{\pgfkeysvalueof{/tikz/optics/io body aspect ratio}}%
    \else%
      \pgfmathsetlengthmacro{\objectWidth}{\pgfkeysvalueof{/tikz/optics/io body aspect ratio}*\pgfkeysvalueof{/tikz/optics/io body height}}%
    \fi%
  }

  \savedmacro\outHeight{%
    \pgfmathparse{\pgfkeysvalueof{/tikz/optics/io aperture height}}
    \ifpgfmathunitsdeclared%
      \pgfmathsetlengthmacro{\outHeight}{\pgfkeysvalueof{/tikz/optics/io aperture height}}%
    \else%
      \pgfmathsetlengthmacro{\outHeight}{\pgfkeysvalueof{/tikz/optics/io aperture height}*\pgfkeysvalueof{/tikz/optics/io body height}}%
    \fi%
  }

  \savedmacro\outWidth{%
    \pgfmathparse{\pgfkeysvalueof{/tikz/optics/io aperture width}}
    \ifpgfmathunitsdeclared%
      \pgfmathsetlengthmacro{\outWidth}{\pgfkeysvalueof{/tikz/optics/io aperture width}}%
    \else%
      \pgfmathsetlengthmacro{\outWidth}{\pgfkeysvalueof{/tikz/optics/io aperture width}*\pgfkeysvalueof{/tikz/optics/io body height}}%
    \fi%
  }

  \savedmacro\outShift{%
    \pgfmathparse{\pgfkeysvalueof{/tikz/optics/io aperture shift}}
    \ifpgfmathunitsdeclared%
      \pgfmathsetlengthmacro{\outShift}{\pgfkeysvalueof{/tikz/optics/io aperture shift}}%
    \else%
      \pgfmathsetlengthmacro{\outShift}{\pgfkeysvalueof{/tikz/optics/io aperture shift}*\pgfkeysvalueof{/tikz/optics/io body height}}%
    \fi%
  }

  \savedanchor{\bodynortheast}{
    \pgf@x=\objectWidth%
    \pgf@y=\objectHeight%
    \pgf@y=0.5\pgf@y%
    \pgf@x=0.5\pgf@x%
  }
  \anchor{body north east}{\bodynortheast}

  \savedanchor{\bodysouthwest}{
    \pgf@x=\objectWidth%
    \pgf@y=\objectHeight%
    \pgf@y=-0.5\pgf@y%
    \pgf@x=-0.5\pgf@x%
  }
  \anchor{body south west}{\bodysouthwest}

  \savedanchor{\outnortheast}{   
    \if@tikz@optics@io@ltr
      % Left To Right (LTR)
      \pgf@x=\objectWidth%
      \pgf@x=0.5\pgf@x%
      \advance\pgf@x by\outWidth
    \else
      % Right To Left (RTL)
      \pgf@x=\objectWidth%
      \pgf@x=0.5\pgf@x%
      \pgf@x=-\pgf@x%
    \fi
    \pgf@y=0pt%
    \advance\pgf@y by\outHeight
    \pgf@y=0.5\pgf@y%
    \advance\pgf@y by\outShift
  }
  \anchor{aperture north east}{\outnortheast}

  \savedanchor{\outsouthwest}{
    \if@tikz@optics@io@ltr
      % Left To Right (LTR)
      \pgf@x=\objectWidth%
      \pgf@x=0.5\pgf@x%
    \else
      % Right To Left (RTL)
      \pgf@x=\objectWidth%
      \pgf@x=0.5\pgf@x%
      \advance\pgf@x by\outWidth
      \pgf@x=-\pgf@x%
    \fi
    \pgf@y=0pt%
    \advance\pgf@y by\outHeight
    \pgf@y=-0.5\pgf@y%
    \advance\pgf@y by\outShift
  }
  \anchor{aperture south west}{\outsouthwest}

  \savedanchor{\outcenter}{
    \if@tikz@optics@io@ltr
      % Left To Right (LTR)
      \pgf@x=\objectWidth%
      \advance\pgf@x by\outWidth
    \else
      % Right To Left (RTL)
      \pgf@x=\objectWidth%
      \advance\pgf@x by\outWidth
      \pgf@x=-\pgf@x%
    \fi
    \pgf@x=0.5\pgf@x%
    \pgf@y=0pt%
    \advance\pgf@y by\outShift
  }
  \anchor{aperture center}{\outcenter}
  

  \anchor{body center}{\center}

  \anchor{text}{
    \pgfpointorigin
    \advance\pgf@x by -.5\wd\pgfnodeparttextbox%
    \advance\pgf@y by -.5\ht\pgfnodeparttextbox%
    \advance\pgf@y by +.5\dp\pgfnodeparttextbox%
  }

  \anchor{body north}{
    \center \pgf@xa=\pgf@x \pgf@ya=\pgf@y
    \bodynortheast \pgf@xb=\pgf@x \pgf@yb=\pgf@y
    \pgf@x=\pgf@xa
    \pgf@y=\pgf@yb
  }
  % north = body north
  \anchor{north}{
    \center \pgf@xa=\pgf@x \pgf@ya=\pgf@y
    \bodynortheast \pgf@xb=\pgf@x \pgf@yb=\pgf@y
    \pgf@x=\pgf@xa
    \pgf@y=\pgf@yb
  }

  \anchor{body south}{
    \center \pgf@xa=\pgf@x \pgf@ya=\pgf@y
    \bodysouthwest \pgf@xb=\pgf@x \pgf@yb=\pgf@y
    \pgf@x=\pgf@xa
    \pgf@y=\pgf@yb
  }
  % south = body south
  \anchor{south}{
    \center \pgf@xa=\pgf@x \pgf@ya=\pgf@y
    \bodysouthwest \pgf@xb=\pgf@x \pgf@yb=\pgf@y
    \pgf@x=\pgf@xa
    \pgf@y=\pgf@yb
  }

  \anchor{body east}{
    \center \pgf@xa=\pgf@x \pgf@ya=\pgf@y
    \bodynortheast \pgf@xb=\pgf@x \pgf@yb=\pgf@y
    \pgf@x=\pgf@xb
    \pgf@y=\pgf@ya
  }

  \anchor{body west}{
    \center \pgf@xa=\pgf@x \pgf@ya=\pgf@y
    \bodysouthwest \pgf@xb=\pgf@x \pgf@yb=\pgf@y
    \pgf@x=\pgf@xb
    \pgf@y=\pgf@ya
  }

  \anchor{body north west}{
    \bodynortheast \pgf@xa=\pgf@x \pgf@ya=\pgf@y
    \bodysouthwest \pgf@xb=\pgf@x \pgf@yb=\pgf@y
    \pgf@x=\pgf@xb
    \pgf@y=\pgf@ya
  }

  \anchor{body south east}{
    \bodynortheast \pgf@xa=\pgf@x \pgf@ya=\pgf@y
    \bodysouthwest \pgf@xb=\pgf@x \pgf@yb=\pgf@y
    \pgf@x=\pgf@xa
    \pgf@y=\pgf@yb
  }


  \anchor{aperture north}{
    \outcenter \pgf@xa=\pgf@x \pgf@ya=\pgf@y
    \outnortheast \pgf@xb=\pgf@x \pgf@yb=\pgf@y
    \pgf@x=\pgf@xa
    \pgf@y=\pgf@yb
  }

  \anchor{aperture south}{
    \outcenter \pgf@xa=\pgf@x \pgf@ya=\pgf@y
    \outsouthwest \pgf@xb=\pgf@x \pgf@yb=\pgf@y
    \pgf@x=\pgf@xa
    \pgf@y=\pgf@yb
  }

  \anchor{aperture east}{
    \outcenter \pgf@xa=\pgf@x \pgf@ya=\pgf@y
    \outnortheast \pgf@xb=\pgf@x \pgf@yb=\pgf@y
    \pgf@x=\pgf@xb
    \pgf@y=\pgf@ya
  }

  \anchor{aperture west}{
    \outcenter \pgf@xa=\pgf@x \pgf@ya=\pgf@y
    \outsouthwest \pgf@xb=\pgf@x \pgf@yb=\pgf@y
    \pgf@x=\pgf@xb
    \pgf@y=\pgf@ya
  }

  \anchor{aperture north west}{
    \outnortheast \pgf@xa=\pgf@x \pgf@ya=\pgf@y
    \outsouthwest \pgf@xb=\pgf@x \pgf@yb=\pgf@y
    \pgf@x=\pgf@xb
    \pgf@y=\pgf@ya
  }

  \anchor{aperture south east}{
    \outnortheast \pgf@xa=\pgf@x \pgf@ya=\pgf@y
    \outsouthwest \pgf@xb=\pgf@x \pgf@yb=\pgf@y
    \pgf@x=\pgf@xa
    \pgf@y=\pgf@yb
  }

  \anchor{center}{\center}

  \savedanchor{\realeast}{
    \pgfpointorigin
    \if@tikz@optics@io@ltr
      % Left To Right (LTR)
      %ltr : aperture east (<- out north east)
      \pgf@x=\objectWidth%
      \pgf@x=0.5\pgf@x%
      \advance\pgf@x by\outWidth
      %\pgf@y=0pt%
    \else
      % Right To Left (rtl)
      \pgf@x=\objectWidth%
      \pgf@x=0.5\pgf@x%
    \fi
  }
  \anchor{east}{\realeast}

  \savedanchor{\realwest}{
    \pgfpointorigin
    \if@tikz@optics@io@ltr
      % Left To Right (LTR)
      \pgf@x=\objectWidth%
      \pgf@x=-0.5\pgf@x%
    \else
      % Right To Left (rtl)
      \pgf@x=\objectWidth%
      \pgf@x=0.5\pgf@x%
      \advance\pgf@x by\outWidth
      \pgf@x=-\pgf@x%
    \fi
  }
  \anchor{west}{\realwest}

  % this is used only for the anchorborder
  \savedanchor{\anchorbordersouthwest}{
    \if@tikz@optics@io@ltr
      % Left To Right (LTR)
      \pgf@x=\objectWidth%
      \pgf@x=-0.5\pgf@x%
    \else
      % Right To Left (rtl)
      \pgf@x=\objectWidth%
      \pgf@x=0.5\pgf@x%
      \advance\pgf@x by\outWidth
      \pgf@x=-\pgf@x%
    \fi
    \pgf@y=\objectHeight%
    \pgf@y=-0.5\pgf@y%
  }

  % this is used only for the anchorborder
  \savedanchor{\anchorbordernortheast}{
    \if@tikz@optics@io@ltr
      % Left To Right (LTR)
      \pgf@x=\objectWidth%
      \pgf@x=0.5\pgf@x%
      \advance\pgf@x by\outWidth
    \else
      % Right To Left (RTL)
      \pgf@x=\objectWidth%
      \pgf@x=0.5\pgf@x%
      %\pgf@x=-\pgf@x%
    \fi
    \pgf@y=\objectHeight%
    \pgf@y=0.5\pgf@y%
  }

  % anchorborder
  % c'est celui de rectangle, mais un peu ajusté
  \anchorborder{%
    \pgf@xb=\pgf@x% xb/yb is target
    \pgf@yb=\pgf@y%
    \anchorbordersouthwest%
    \pgf@xa=\pgf@x% xa/ya is se
    \pgf@ya=\pgf@y%
    \anchorbordernortheast%
    \advance\pgf@x by-\pgf@xa%
    \advance\pgf@y by-\pgf@ya%
    \pgf@xc=.5\pgf@x% x/y is half width/height
    \pgf@yc=.5\pgf@y%
    \advance\pgf@xa by\pgf@xc% xa/ya becomes center
    \advance\pgf@ya by\pgf@yc%
    \edef\pgf@marshal{%
      \noexpand\pgfpointborderrectangle
      {\noexpand\pgfqpoint{\the\pgf@xb}{\the\pgf@yb}}
      {\noexpand\pgfqpoint{\the\pgf@xc}{\the\pgf@yc}}%
    }%
    \pgf@process{\pgf@marshal}%
    % \advance\pgf@x by\pgf@xa%
    % \advance\pgf@y by\pgf@ya%
  }

  \backgroundpath
  {
    % corps
    \pgfpathrectanglecorners{\bodynortheast}{\bodysouthwest}
    % out
    \pgfpathrectanglecorners{\outnortheast}{\outsouthwest}

    %\pgfusepath{draw}
    % il ne faut PAS mettre ça pour pouvoir utiliser des styles correctement après (genre double)
  }
}


%%%%%%%%%%%%%%%%%%%%%%%%%%%%%%%%%%%%%%%%%%%%%%%%%%%%%%%%%%%%%%%%%%%%%%%%%%%%%%%%
%%%%%%%%%%%%%%%%%%%%%%%%%%%%%%%%%%%%%%%%%%%%%%%%%%%%%%%%%%%%%%%%%%%%%%%%%%%%%%%%
% Sensors
%%%%%%%%%%%%%%%%%%%%%%%%%%%%%%%%%%%%%%%%%%%%%%%%%%%%%%%%%%%%%%%%%%%%%%%%%%%%%%%%
%%%%%%%%%%%%%%%%%%%%%%%%%%%%%%%%%%%%%%%%%%%%%%%%%%%%%%%%%%%%%%%%%%%%%%%%%%%%%%%%


%%%%%%%%%%%%%%%%%%%%%%%%%%%%%%%%%%%%%%%%%%%%%%%%%%%%%%%%%%%%%%%%%%%%%%%%%%%%%%%%
% Shape [sensor line]
%%%%%%%%%%%%%%%%%%%%%%%%%%%%%%%%%%%%%%%%%%%%%%%%%%%%%%%%%%%%%%%%%%%%%%%%%%%%%%%%
% Sensor line.
%
% This defines anchors center, north, south, east, west, north east
% north west, south east, south west, as well as
% pixel <i> <subanchor> with <subanchor> is any of the former.
% ça va être horrible à faire
%
% keys :
% - sensor line height
% - sensor line aspect ratio (largeur en unités de la hauteur du capteur)
% - sensor line pixel number
% - sensor line pixel width (en unités de la largeur du capteur)
% - sensor line inner ysep (en unités de la hauteur du capteur)
\pgfkeys{/tikz/optics/.cd,
  sensor line height/.initial=2cm,
  sensor line aspect ratio/.initial=0.2,
  sensor line pixel number/.initial=5,
  sensor line pixel width/.initial=0.4,
  sensor line inner ysep/.initial=0.05, % entre le bord et les capteurs
}

\pgfdeclareshape{sensor line}
{
  \savedanchor{\center}{
    \pgfpointorigin
  }
  \anchor{center}{\center}

  \savedmacro\objectHeight{%
    \edef\objectHeight{\pgfkeysvalueof{/tikz/optics/sensor line height}}%
  }

  \savedmacro\objectWidth{%
    \pgfmathsetlengthmacro{\objectWidth}{\pgfkeysvalueof{/tikz/optics/sensor line aspect ratio}*\pgfkeysvalueof{/tikz/optics/sensor line height}}
  }

  \savedmacro\pixelNumber{%
    \pgfmathtruncatemacro\pixelNumber{\pgfkeysvalueof{/tikz/optics/sensor line pixel number}}%
  }

  \savedmacro\innerysep{%
    \pgfmathparse{\pgfkeysvalueof{/tikz/optics/sensor line inner ysep}}
    \ifpgfmathunitsdeclared%
      \pgfmathsetlengthmacro{\innerysep}{\pgfkeysvalueof{/tikz/optics/sensor line inner ysep}}%
    \else%
      \pgfmathsetlengthmacro{\innerysep}{\pgfkeysvalueof{/tikz/optics/sensor line inner ysep}*\pgfkeysvalueof{/tikz/optics/sensor line height}}%
    \fi%
  }
  
  \savedmacro\pixelWidth{%
    \pgfmathparse{\pgfkeysvalueof{/tikz/optics/sensor line pixel width}}
    \ifpgfmathunitsdeclared%
      \pgfmathsetlengthmacro{\pixelWidth}{\pgfkeysvalueof{/tikz/optics/sensor line pixel width}}%
    \else%
      \pgfmathsetlengthmacro{\pixelWidth}{\pgfkeysvalueof{/tikz/optics/sensor line pixel width}*\objectWidth}%
    \fi%
  }

  \savedmacro\pixelHeight{%
    \pgfmathparse{(\objectHeight-2*\innerysep)/\pixelNumber}
    \edef\pixelHeight{\pgfmathresult pt}%
  }

  \savedanchor{\northeast}{
    \pgf@x=\objectWidth%
    \pgf@y=\objectHeight%
    \pgf@y=0.5\pgf@y%
    \pgf@x=0.5\pgf@x%
  }
  \anchor{north east}{\northeast}

  \savedanchor{\southwest}{
    \pgf@x=\objectWidth%
    \pgf@y=\objectHeight%
    \pgf@y=-0.5\pgf@y%
    \pgf@x=-0.5\pgf@x%
  }
  \anchor{south west}{\southwest}

  \anchor{north}{
    \center \pgf@xa=\pgf@x \pgf@ya=\pgf@y
    \northeast \pgf@xb=\pgf@x \pgf@yb=\pgf@y
    \pgf@x=\pgf@xa
    \pgf@y=\pgf@yb
  }

  \anchor{south}{
    \center \pgf@xa=\pgf@x \pgf@ya=\pgf@y
    \southwest \pgf@xb=\pgf@x \pgf@yb=\pgf@y
    \pgf@x=\pgf@xa
    \pgf@y=\pgf@yb
  }

  \anchor{east}{
    \center \pgf@xa=\pgf@x \pgf@ya=\pgf@y
    \northeast \pgf@xb=\pgf@x \pgf@yb=\pgf@y
    \pgf@x=\pgf@xb
    \pgf@y=\pgf@ya
  }

  \anchor{west}{
    \center \pgf@xa=\pgf@x \pgf@ya=\pgf@y
    \southwest \pgf@xb=\pgf@x \pgf@yb=\pgf@y
    \pgf@x=\pgf@xb
    \pgf@y=\pgf@ya
  }

  \anchor{north west}{
    \northeast \pgf@xa=\pgf@x \pgf@ya=\pgf@y
    \southwest \pgf@xb=\pgf@x \pgf@yb=\pgf@y
    \pgf@x=\pgf@xb
    \pgf@y=\pgf@ya
  }

  \anchor{south east}{
    \northeast \pgf@xa=\pgf@x \pgf@ya=\pgf@y
    \southwest \pgf@xb=\pgf@x \pgf@yb=\pgf@y
    \pgf@x=\pgf@xa
    \pgf@y=\pgf@yb
  }

  % ok, idée :
  % HAUT (north) du pixel n (allant de 1 à N)
  % \objectHeight/2 - \innerysep - (n-1) * \pixelHeight
  % BAS (south) du pixel n
  % \objectHeight/2 - \innerysep - n * \pixelHeight
  % west = le même que la boite, donc utiliser \southwest
  % east = west + \pixelWidth   

  % Le but de ce code est de définir des ancres
  % pixel <i> <pos>
  % pour <i> = 4, 2, ..., \pixelNumber
  % et <pos> = north, south, east, west, north west, south west, north east south east, center
  % Comme \pixelNumber est défini dynamiquement, il faut passer par cette horreur.
  % Le code est inspiré de pfglibraryshapes.geometric.code.tex (shape regular polygon).
  \expandafter\pgfutil@g@addto@macro\csname pgf@sh@s@sensor line\endcsname{%
    \c@pgf@counta\pixelNumber\relax%
    \pgfmathloop%
      \ifnum\c@pgf@counta>0\relax%
        \pgfutil@ifundefined{pgf@anchor@sensor line@pixel\space\the\c@pgf@counta\space north west}{%
        %
        % ...(manually \xdef as \gdef is normally used by \anchor)...
        %
        %
        % pixel surface north
        \expandafter\xdef\csname pgf@anchor@sensor line@pixel\space\the\c@pgf@counta\space north\endcsname{%
          \noexpand\northeast \noexpand\pgf@xa=\noexpand\pgf@x \noexpand\pgf@ya=\noexpand\pgf@y
          \noexpand\southwest \noexpand\pgf@xb=\noexpand\pgf@x \noexpand\pgf@yb=\noexpand\pgf@y
          \noexpand\pgf@x=\noexpand\pixelWidth
          \noexpand\pgf@x=.5\noexpand\pgf@x
          \noexpand\advance\noexpand\pgf@x by\noexpand\pgf@xb
          \noexpand\pgf@y=\noexpand\pgf@ya
          \noexpand\newdimen\noexpand\temp@y
          \noexpand\pgfmathsetlength\noexpand\temp@y{(\the\c@pgf@counta-\noexpand\pixelNumber)*\noexpand\pixelHeight-\noexpand\innerysep}
          \noexpand\advance\noexpand\pgf@y by\noexpand\temp@y
        }%
        %
        %
        % pixel surface north east
        \expandafter\xdef\csname pgf@anchor@sensor line@pixel\space\the\c@pgf@counta\space north east\endcsname{%
          \noexpand\northeast \noexpand\pgf@xa=\noexpand\pgf@x \noexpand\pgf@ya=\noexpand\pgf@y
          \noexpand\southwest \noexpand\pgf@xb=\noexpand\pgf@x \noexpand\pgf@yb=\noexpand\pgf@y
          \noexpand\pgf@x=\noexpand\pixelWidth
          \noexpand\advance\noexpand\pgf@x by\noexpand\pgf@xb
          \noexpand\pgf@y=\noexpand\pgf@ya
          \noexpand\newdimen\noexpand\temp@y
          \noexpand\pgfmathsetlength\noexpand\temp@y{(\the\c@pgf@counta-\noexpand\pixelNumber)*\noexpand\pixelHeight-\noexpand\innerysep}
          \noexpand\advance\noexpand\pgf@y by\noexpand\temp@y
        }%
        %
        %
        % pixel surface north west
        \expandafter\xdef\csname pgf@anchor@sensor line@pixel\space\the\c@pgf@counta\space north west\endcsname{%
          \noexpand\northeast \noexpand\pgf@xa=\noexpand\pgf@x \noexpand\pgf@ya=\noexpand\pgf@y
          \noexpand\southwest \noexpand\pgf@xb=\noexpand\pgf@x \noexpand\pgf@yb=\noexpand\pgf@y
          \noexpand\pgf@x=\noexpand\pgf@xb
          \noexpand\pgf@y=\noexpand\pgf@ya
          \noexpand\newdimen\noexpand\temp@y
          \noexpand\pgfmathsetlength\noexpand\temp@y{(\the\c@pgf@counta-\noexpand\pixelNumber)*\noexpand\pixelHeight-\noexpand\innerysep}
          \noexpand\advance\noexpand\pgf@y by\noexpand\temp@y
        }%
        %
        %
        % pixel surface south
        \expandafter\xdef\csname pgf@anchor@sensor line@pixel\space\the\c@pgf@counta\space south\endcsname{%
          \noexpand\northeast \noexpand\pgf@xa=\noexpand\pgf@x \noexpand\pgf@ya=\noexpand\pgf@y
          \noexpand\southwest \noexpand\pgf@xb=\noexpand\pgf@x \noexpand\pgf@yb=\noexpand\pgf@y
          \noexpand\pgf@x=\noexpand\pixelWidth
          \noexpand\pgf@x=.5\noexpand\pgf@x
          \noexpand\advance\noexpand\pgf@x by\noexpand\pgf@xb
          \noexpand\pgf@y=\noexpand\pgf@ya
          \noexpand\newdimen\noexpand\temp@y
          \noexpand\pgfmathsetlength\noexpand\temp@y{(\the\c@pgf@counta-\noexpand\pixelNumber-1)*\noexpand\pixelHeight-\noexpand\innerysep}
          \noexpand\advance\noexpand\pgf@y by\noexpand\temp@y
        }%
        %
        %
        % pixel surface south east
        \expandafter\xdef\csname pgf@anchor@sensor line@pixel\space\the\c@pgf@counta\space south east\endcsname{%
          \noexpand\northeast \noexpand\pgf@xa=\noexpand\pgf@x \noexpand\pgf@ya=\noexpand\pgf@y
          \noexpand\southwest \noexpand\pgf@xb=\noexpand\pgf@x \noexpand\pgf@yb=\noexpand\pgf@y
          \noexpand\pgf@x=\noexpand\pixelWidth
          \noexpand\advance\noexpand\pgf@x by\noexpand\pgf@xb
          \noexpand\pgf@y=\noexpand\pgf@ya
          \noexpand\newdimen\noexpand\temp@y
          \noexpand\pgfmathsetlength\noexpand\temp@y{(\the\c@pgf@counta-\noexpand\pixelNumber-1)*\noexpand\pixelHeight-\noexpand\innerysep}
          \noexpand\advance\noexpand\pgf@y by\noexpand\temp@y
        }%
        %
        %
        % pixel surface south west
        \expandafter\xdef\csname pgf@anchor@sensor line@pixel\space\the\c@pgf@counta\space south west\endcsname{%
          \noexpand\northeast \noexpand\pgf@xa=\noexpand\pgf@x \noexpand\pgf@ya=\noexpand\pgf@y
          \noexpand\southwest \noexpand\pgf@xb=\noexpand\pgf@x \noexpand\pgf@yb=\noexpand\pgf@y
          \noexpand\pgf@x=\noexpand\pgf@xb
          \noexpand\pgf@y=\noexpand\pgf@ya
          \noexpand\newdimen\noexpand\temp@y
          \noexpand\pgfmathsetlength\noexpand\temp@y{(\the\c@pgf@counta-\noexpand\pixelNumber-1)*\noexpand\pixelHeight-\noexpand\innerysep}
          \noexpand\advance\noexpand\pgf@y by\noexpand\temp@y
        }%
        %
        %
        % pixel surface east
        \expandafter\xdef\csname pgf@anchor@sensor line@pixel\space\the\c@pgf@counta\space east\endcsname{%
          \noexpand\northeast \noexpand\pgf@xa=\noexpand\pgf@x \noexpand\pgf@ya=\noexpand\pgf@y
          \noexpand\southwest \noexpand\pgf@xb=\noexpand\pgf@x \noexpand\pgf@yb=\noexpand\pgf@y
          \noexpand\pgf@x=\noexpand\pixelWidth
          \noexpand\advance\noexpand\pgf@x by\noexpand\pgf@xb
          \noexpand\pgf@y=\noexpand\pgf@ya
          \noexpand\newdimen\noexpand\temp@y
          \noexpand\pgfmathsetlength\noexpand\temp@y{(\the\c@pgf@counta-\noexpand\pixelNumber-0.5)*\noexpand\pixelHeight-\noexpand\innerysep}
          \noexpand\advance\noexpand\pgf@y by\noexpand\temp@y
        }%
        %
        %
        % pixel surface center
        \expandafter\xdef\csname pgf@anchor@sensor line@pixel\space\the\c@pgf@counta\space center\endcsname{%
          \noexpand\northeast \noexpand\pgf@xa=\noexpand\pgf@x \noexpand\pgf@ya=\noexpand\pgf@y
          \noexpand\southwest \noexpand\pgf@xb=\noexpand\pgf@x \noexpand\pgf@yb=\noexpand\pgf@y
          \noexpand\pgf@x=\noexpand\pixelWidth
          \noexpand\pgf@x=.5\noexpand\pgf@x
          \noexpand\advance\noexpand\pgf@x by\noexpand\pgf@xb
          \noexpand\pgf@y=\noexpand\pgf@ya
          \noexpand\newdimen\noexpand\temp@y
          \noexpand\pgfmathsetlength\noexpand\temp@y{(\the\c@pgf@counta-\noexpand\pixelNumber-0.5)*\noexpand\pixelHeight-\noexpand\innerysep}
          \noexpand\advance\noexpand\pgf@y by\noexpand\temp@y
        }%
        %
        %
        % pixel surface west
        \expandafter\xdef\csname pgf@anchor@sensor line@pixel\space\the\c@pgf@counta\space west\endcsname{%
          \noexpand\northeast \noexpand\pgf@xa=\noexpand\pgf@x \noexpand\pgf@ya=\noexpand\pgf@y
          \noexpand\southwest \noexpand\pgf@xb=\noexpand\pgf@x \noexpand\pgf@yb=\noexpand\pgf@y
          \noexpand\pgf@x=\noexpand\pgf@xb
          \noexpand\pgf@y=\noexpand\pgf@ya
          \noexpand\newdimen\noexpand\temp@y
          \noexpand\pgfmathsetlength\noexpand\temp@y{(\the\c@pgf@counta-\noexpand\pixelNumber-0.5)*\noexpand\pixelHeight-\noexpand\innerysep}
          \noexpand\advance\noexpand\pgf@y by\noexpand\temp@y
        }%
      }{\c@pgf@counta0\relax}% 
      \advance\c@pgf@counta-1\relax%
    \repeatpgfmathloop% 
  }%

  \inheritanchorborder[from=rectangle]

  \backgroundpath
  {
    % rectangle contour
    \pgfpathrectanglecorners{\northeast}{\southwest}

    % dessin des pixels
    \c@pgf@counta\pixelNumber\relax%
    \pgfmathloop%
      \ifnum\c@pgf@counta>0\relax%
        \newdimen\pixel@northeast@x
        \newdimen\pixel@northeast@y
        \newdimen\pixel@southwest@x
        \newdimen\pixel@southwest@y
        \northeast \pgf@xa=\pgf@x \pgf@ya=\pgf@y
        \southwest \pgf@xb=\pgf@x \pgf@yb=\pgf@y
        % calcul pixel@northeast
        %% calcul x
        \pixel@northeast@x=\pgf@xb
        \advance\pixel@northeast@x by\pixelWidth
        %% calcul y
        \pixel@northeast@y=\pgf@ya
        \newdimen\temp@y
        \pgfmathsetlength\temp@y{(\the\c@pgf@counta-\pixelNumber)*\pixelHeight-\innerysep}
        \advance\pixel@northeast@y by\noexpand\temp@y
        % calcul pixel@southwest
        \pixel@southwest@x=\pgf@xb
        \pixel@southwest@y=\pgf@ya
        \newdimen\temp@y
        \pgfmathsetlength\temp@y{(\the\c@pgf@counta-\pixelNumber-1)*\pixelHeight-\innerysep}
        \advance\pixel@southwest@y by\temp@y
        % dessin
        \pgfpathrectanglecorners{\pgfpoint{\pixel@northeast@x}{\pixel@northeast@y}}{\pgfpoint{\pixel@southwest@x}{\pixel@southwest@y}}
      \advance\c@pgf@counta-1\relax%
    \repeatpgfmathloop% 
  }
}


%%%%%%%%%%%%%%%%%%%%%%%%%%%%%%%%%%%%%%%%%%%%%%%%%%%%%%%%%%%%%%%%%%%%%%%%%%%%%%%%
%%%%%%%%%%%%%%%%%%%%%%%%%%%%%%%%%%%%%%%%%%%%%%%%%%%%%%%%%%%%%%%%%%%%%%%%%%%%%%%%
% Styles defining optics elements in terms of the existing shapes.
%%%%%%%%%%%%%%%%%%%%%%%%%%%%%%%%%%%%%%%%%%%%%%%%%%%%%%%%%%%%%%%%%%%%%%%%%%%%%%%%
%%%%%%%%%%%%%%%%%%%%%%%%%%%%%%%%%%%%%%%%%%%%%%%%%%%%%%%%%%%%%%%%%%%%%%%%%%%%%%%%

% Style [screen]
\pgfkeys{/tikz/optics/screen/.style={thin optics element, very thick}}
% Style [diffraction grating]
\pgfkeys{/tikz/optics/diffraction grating/.style={thin optics element, /tikz/optics/cheating dash={on 4pt off 2pt}}}
% Style [grid]
\pgfkeys{/tikz/optics/grid/.style={thin optics element, /tikz/optics/cheating dash={on 4pt off 2pt}, ultra thick}}
% Style [semi-transparent mirror]
\pgfkeys{/tikz/optics/semi-transparent mirror/.style={thin optics element, densely dotted, thick}}
% Style [diaphragm]
\pgfkeys{/tikz/optics/diaphragm/.style={slit,/tikz/optics/slit height=0.4}}


% Style [generic lamp]
\pgfkeys{/tikz/optics/generic lamp/.style={shape=generic optics io,optics,draw}}
% Style [generic sensor]
\pgfkeys{/tikz/optics/generic sensor/.style={shape=generic optics io,optics,io orientation=rtl,draw, io body height=1cm, io body aspect ratio=0.5, io aperture width=0.15}}
% Style [halogen lamp]
\pgfkeys{/tikz/optics/halogen lamp/.style={/tikz/optics/generic lamp, io body height=0.75cm, io body aspect ratio=2,io aperture width=0.33,io aperture height=0.66,io aperture shift=0}}
% Style [spectral lamp]
\pgfkeys{/tikz/optics/spectral lamp/.style={/tikz/optics/generic lamp, io body height=2.25cm, io body aspect ratio=2/3,io aperture width=0.11,io aperture height=0.22,io aperture shift=0.25,/tikz/optics/io multiline}}
% Style [laser]
\pgfkeys{/tikz/optics/laser/.style={/tikz/optics/generic lamp, io body height=0.5cm, io body aspect ratio=3,io aperture width=0.22,io aperture height=0.5,io aperture shift=0}}
% Style [laser']
\pgfkeys{/tikz/optics/laser'/.style={/tikz/optics/generic lamp, io body height=0.5cm, io body aspect ratio=3,io aperture width=0,io aperture height=0.5,io aperture shift=0}}
% Style [beam splitter]
\pgfkeys{/tikz/optics/beam splitter/.style={shape=polarizer, optics, draw, object height=1cm, object aspect ratio=1}}



%%%%%%%%%%%%%%%%%%%%%%%%%%%%%%%%%%%%%%%%%%%%%%%%%%%%%%%%%%%%%%%%%%%%%%%%%%%%%%%%
% Helper style to draw correctly dashed paths.
%%%%%%%%%%%%%%%%%%%%%%%%%%%%%%%%%%%%%%%%%%%%%%%%%%%%%%%%%%%%%%%%%%%%%%%%%%%%%%%%
% source : http://tex.stackexchange.com/questions/133271/can-tikz-dashed-lines-emulate-pstricks-dashed-lines
% le but est d'avoir des dash symétriques par rapport au milieu et surtout avec un trait entier de chaque côté
\tikzset{
    /tikz/optics/cheating dash/.code args={on #1 off #2}{
        % Use csname so catcode of @ doesn't have do be changed.
        \csname tikz@addoption\endcsname{%
            \pgfgetpath\currentpath%
            \pgfprocessround{\currentpath}{\currentpath}%
            \csname pgf@decorate@parsesoftpath\endcsname{\currentpath}{\currentpath}%
            \pgfmathparse{\csname pgf@decorate@totalpathlength\endcsname-#1}\let\rest=\pgfmathresult%
            \pgfmathparse{#1+#2}\let\onoff=\pgfmathresult%
            \pgfmathparse{max(floor(\rest/\onoff), 1)}\let\nfullonoff=\pgfmathresult%
            \pgfmathparse{max((\rest-\onoff*\nfullonoff)/\nfullonoff+#2, #2)}\let\offexpand=\pgfmathresult%
            \pgfsetdash{{#1}{\offexpand}}{0pt}}%
    }
}


%%%%%%%%%%%%%%%%%%%%%%%%%%%%%%%%%%%%%%%%%%%%%%%%%%%%%%%%%%%%%%%%%%%%%%%%%%%%%%%%
% Helper style for multiline io elements.
%%%%%%%%%%%%%%%%%%%%%%%%%%%%%%%%%%%%%%%%%%%%%%%%%%%%%%%%%%%%%%%%%%%%%%%%%%%%%%%%
\pgfkeys{/tikz/optics/io multiline/.code={\newdimen\tmplen\pgfmathsetlength{\tmplen}{\pgfkeysvalueof{/tikz/optics/io body aspect ratio}*\pgfkeysvalueof{/tikz/optics/io body height}}\tikzset{text width=\the\tmplen,align=center}}}


%%%%%%%%%%%%%%%%%%%%%%%%%%%%%%%%%%%%%%%%%%%%%%%%%%%%%%%%%%%%%%%%%%%%%%%%%%%%%%%%
%%%%%%%%%%%%%%%%%%%%%%%%%%%%%%%%%%%%%%%%%%%%%%%%%%%%%%%%%%%%%%%%%%%%%%%%%%%%%%%%
% Helper styles to mark interesing points
%%%%%%%%%%%%%%%%%%%%%%%%%%%%%%%%%%%%%%%%%%%%%%%%%%%%%%%%%%%%%%%%%%%%%%%%%%%%%%%%
%%%%%%%%%%%%%%%%%%%%%%%%%%%%%%%%%%%%%%%%%%%%%%%%%%%%%%%%%%%%%%%%%%%%%%%%%%%%%%%%


% Style [mark point]
% Describes how interesting points should be drawn (e.g. by [draw focal points], [draw mirror focus], [draw mirror center])
\pgfkeys{/tikz/optics/mark point/.style={optics/mark a cross}}

% Style [draw focal points]
% Should be applied to a [shape=lens] node to draw its focal points according to the style [mark point].
\pgfkeys{/tikz/optics/draw focal points/.style={append after command={
    \pgfextra{
      \begin{pgfinterruptpath}
          \node[/tikz/optics/mark point,#1] at (\tikzlastnode.west focal point) {};
          \node[/tikz/optics/mark point,#1] at (\tikzlastnode.east focal point) {};
      \end{pgfinterruptpath}
    }
  }
}}

% Style [draw mirror focus]
% Should be applied to a [shape=spherical mirror] node to draw its focal point according to the style [mark point].
\pgfkeys{/tikz/optics/draw mirror focus/.style={append after command={
    \pgfextra{
      \begin{pgfinterruptpath}
          \node[/tikz/optics/mark point,#1] at (\tikzlastnode.focus) {};
      \end{pgfinterruptpath}
    }
  }
}}

% Style [draw mirror center]
% Should be applied to a [shape=spherical mirror] node to draw its center according to the style [mark point].
\pgfkeys{/tikz/optics/draw mirror center/.style={append after command={
    \pgfextra{
      \begin{pgfinterruptpath}
          \node[/tikz/optics/mark point,#1] at (\tikzlastnode.mirror center) {};
      \end{pgfinterruptpath}
    }
  }
}}

% Style [mark a cross]
% Draws a cross at the node
\pgfkeys{/tikz/optics/mark a cross/.style={cross out,draw,inner sep=0pt,minimum width=2pt,minimum height=2pt}}

% idée : utiliser pgfextra pour les flèches, si on peut avoir les deux dernières nodes ?

%%%%%%%%%%%%%%%%%%%%%%%%%%%%%%%%%%%%%%%%%%%%%%%%%%%%%%%%%%%%%%%%%%%%%%%%%%%%%%%%
%%%%%%%%%%%%%%%%%%%%%%%%%%%%%%%%%%%%%%%%%%%%%%%%%%%%%%%%%%%%%%%%%%%%%%%%%%%%%%%%
% Helper styles to easily put things on paths.
% It is often needed in optics to have e.g. arrows in the middle of a path.
% It is often needed everywhere to be able to put a coordinate somewhere on 
% a path. 
% This is the aim of these helpers.
%%%%%%%%%%%%%%%%%%%%%%%%%%%%%%%%%%%%%%%%%%%%%%%%%%%%%%%%%%%%%%%%%%%%%%%%%%%%%%%%
%%%%%%%%%%%%%%%%%%%%%%%%%%%%%%%%%%%%%%%%%%%%%%%%%%%%%%%%%%%%%%%%%%%%%%%%%%%%%%%%


%%%%%%%%%%%%%%%%%%%%%%%%%%%%%%%%%%%%%%%%%%%%%%%%%%%%%%%%%%%%%%%%%%%%%%%%%%%%%%%%
% Style [put coordinate]
%%%%%%%%%%%%%%%%%%%%%%%%%%%%%%%%%%%%%%%%%%%%%%%%%%%%%%%%%%%%%%%%%%%%%%%%%%%%%%%%
% Le style [put coordinate=<coord> at <pos>] crée une node[coordinate]
% à l'abscisse curviligne <pos> sur le chemin auquel est appliqué le style.
%
% Exemple : 
% \draw[put coordinate=P at 0.3] (0,0) to[bend left] (2cm,0);
% \draw[red] (P) -- (0,0);
%
\tikzset{
  put coordinate/.style args={#1 at #2}{decoration={markings, mark=at position #2 with {\node[coordinate] (#1) {};}},postaction={decorate}}
}

%%%%%%%%%%%%%%%%%%%%%%%%%%%%%%%%%%%%%%%%%%%%%%%%%%%%%%%%%%%%%%%%%%%%%%%%%%%%%%%%
% Style [put arrow]
%%%%%%%%%%%%%%%%%%%%%%%%%%%%%%%%%%%%%%%%%%%%%%%%%%%%%%%%%%%%%%%%%%%%%%%%%%%%%%%%
%
% The aim of this horrible mess is that I want expansion to take place at the right time, so I can ask for something like
% \draw[/tikz/optics/->-={at=0.25}, /tikz/optics/->-={at=0.75}] (0,0) -- (2cm,2cm) -- (4cm,0);
% and have it work. Obviously, I want more complicated things (several arrows with different directions, colors, number of >, etc. on the same path).
% With a naive implementation, this does not work and the last setting always wins. 
% A possible solution would be positional arguments, however keywords arguments are Better (tm).
% Hence the need to take care of expansion order, so the specifications apply only to the currently drawn arrow tip.
% The main idea behing this code is to use Magic (tm) so that Things Work (tm) and never touch it again. 
% However, it is obvious from a trivial application of Murphy Law that this will backfire sometime.
% Indeed, a key with positional arguments are used internally (|ordered draw key|), which is called from the parsed arguments.
\tikzset{/tikz/put arrow/.cd,
  pos/.initial=0.5,
  at/.style={pos=#1},
  pos var/.initial={},
  style/.initial={},
  style var/.initial={},
  postaction style/.initial={}, % more expansion magic needed for ->n- and friends (\arrow[thing=stuff] does not seem to work, probably because of the =)
  arrow macro/.initial={arrow},
  arrow macro var/.initial={},
  arrow type/.initial={>},
  arrow type var/.initial={},
  reversed/.style={arrow macro=arrowreversed},
  arrow/.style={arrow type=#1},
  arrow'/.style={arrow type=#1, reversed},
  every arrow/.style={},
}

\tikzset{put arrow/.code={
  % parse arguments correctly
  \tikzset{/tikz/put arrow/.cd, #1}
  \tikzset{/tikz/put arrow/pos var/.expand once={\pgfkeysvalueof{/tikz/put arrow/pos}}}
  \tikzset{/tikz/put arrow/style var/.expanded={\pgfkeysvalueof{/tikz/put arrow/style}}}
  \tikzset{/tikz/put arrow/arrow macro var/.expand once={\pgfkeysvalueof{/tikz/put arrow/arrow macro}}}
  \tikzset{/tikz/put arrow/arrow type var/.expand once={\pgfkeysvalueof{/tikz/put arrow/arrow type}}}
  % call |ordered draw key| which does the real job
  \tikzset{/tikz/put arrow/ordered draw key/.expanded=%
      {\pgfkeysvalueof{/tikz/put arrow/pos var}}%
      {\pgfkeysvalueof{/tikz/put arrow/style var}}%
      {\pgfkeysvalueof{/tikz/put arrow/arrow macro var}}%
      {\pgfkeysvalueof{/tikz/put arrow/arrow type var}}%
      {\pgfkeysvalueof{/tikz/put arrow/postaction style}}%
  }
  % restore initial values
  \tikzset{/tikz/put arrow/pos=0.5} 
  \tikzset{/tikz/put arrow/style={}}
  \tikzset{/tikz/put arrow/arrow macro={arrow}}
  \tikzset{/tikz/put arrow/arrow type={>}}
}}

\tikzset{/tikz/put arrow/ordered draw key/.code n args={5}{
  \tikzset{postaction={#5,decorate, decoration={markings, mark=at position #1 with {\csname #3\endcsname[/tikz/put arrow/every arrow,#2]{#4}};}}}
}}

%%%%%%%%%%%%%%%%%%%%%%%%%%%%%%%%%%%%%%%%%%%%%%%%%%%%%%%%%%%%%%%%%%%%%%%%%%%%%%%%
% Styles to mark light rays
%%%%%%%%%%%%%%%%%%%%%%%%%%%%%%%%%%%%%%%%%%%%%%%%%%%%%%%%%%%%%%%%%%%%%%%%%%%%%%%%
\tikzset{/tikz/optics/multiple ray arrow/.cd,
  n/.initial=1,
  n var/.initial=1,
  set n/.code={\pgfsetarrowoptions{multiple ray arrow}{#1}},
}


\pgfkeys{
  /tikz/optics/.cd, 
  use ray arrow >/.code 2 args={
    \pgfsetarrowoptions{ray arrow@length}{4pt}
    \pgfsetarrowoptions{ray arrow@angle}{50}
    \tikzset{/tikz/put arrow/postaction style/.expanded={/tikz/optics/multiple ray arrow/set n=#1}}
    \tikzset{/tikz/put arrow/.expanded={arrow={multiple ray arrow}, #2}}
  },
  use ray arrow </.code 2 args={
    \pgfsetarrowoptions{ray arrow@length}{4pt}
    \pgfsetarrowoptions{ray arrow@angle}{50}
    \tikzset{/tikz/put arrow/postaction style/.expanded={/tikz/optics/multiple ray arrow/set n=#1}}
    \tikzset{/tikz/put arrow/.expanded={arrow'={multiple ray arrow}, #2}}
  },
  ->n-/.code={
    \tikzset{/tikz/optics/multiple ray arrow/.cd, .collect unknowns,%
      #1,
      unknown options/.get = \arrowkeys}
    \tikzset{/tikz/optics/multiple ray arrow/n var/.expand once={\pgfkeysvalueof{/tikz/optics/multiple ray arrow/n}}}
    \tikzset{/tikz/optics/use ray arrow >={\pgfkeysvalueof{/tikz/optics/multiple ray arrow/n var}}{\arrowkeys}}
  },
  -<n-/.code={
    \tikzset{/tikz/optics/multiple ray arrow/.cd, .collect unknowns,%
      #1,
      unknown options/.get = \arrowkeys}
    \tikzset{/tikz/optics/multiple ray arrow/n var/.expand once={\pgfkeysvalueof{/tikz/optics/multiple ray arrow/n}}}
    \tikzset{/tikz/optics/use ray arrow <={\pgfkeysvalueof{/tikz/optics/multiple ray arrow/n var}}{\arrowkeys}}
  },
  %
  ->-/.style={/tikz/optics/->n-={n=1, #1}},
  -<-/.style={/tikz/optics/-<n-={n=1, #1}},
  ->>-/.style={/tikz/optics/->n-={n=2, #1}},
  -<<-/.style={/tikz/optics/-<n-={n=2, #1}},
  ->>>-/.style={/tikz/optics/->n-={n=3, #1}},
  -<<<-/.style={/tikz/optics/-<n-={n=3, #1}},
  ->>>>-/.style={/tikz/optics/->n-={n=4, #1}},
  -<<<<-/.style={/tikz/optics/-<n-={n=4, #1}},
}




%%%%%%%%%%%%%%%%%%%%%%%%%%%%%%%%%%%%%%%%%%%%%%%%%%%%%%%%%%%%%%%%%%%%%%%%%%%%%%%%
%%%%%%%%%%%%%%%%%%%%%%%%%%%%%%%%%%%%%%%%%%%%%%%%%%%%%%%%%%%%%%%%%%%%%%%%%%%%%%%%
% Styles [dim arrow] and [short dim arrow]
%%%%%%%%%%%%%%%%%%%%%%%%%%%%%%%%%%%%%%%%%%%%%%%%%%%%%%%%%%%%%%%%%%%%%%%%%%%%%%%%
%%%%%%%%%%%%%%%%%%%%%%%%%%%%%%%%%%%%%%%%%%%%%%%%%%%%%%%%%%%%%%%%%%%%%%%%%%%%%%%%
\def\dimarrow@short@position{0}
\newif\ifdimarrow@nearstart
\dimarrow@nearstarttrue
\tikzset{%
  /tikz/dim arrow/.code={\tikzset{draw,/tikz/dim arrow/draw dim arrow}\pgfkeys{/tikz/dim arrow/.cd,#1}},
  /tikz/dim arrow'/.code={\pgfkeysgetvalue{/tikz/dim arrow/raise}{\tmp@tdar}\tikzset{draw,/tikz/dim arrow/draw dim arrow,/tikz/dim arrow/raise=-\tmp@tdar}\pgfkeys{/tikz/dim arrow/.cd,#1}},
  /tikz/short dim arrow/.code={\tikzset{draw,/tikz/dim arrow/draw short dim arrow}\pgfkeys{/tikz/dim arrow/.cd,#1}},
  /tikz/short dim arrow'/.code={\pgfkeysgetvalue{/tikz/dim arrow/raise}{\tmp@tdar}\tikzset{draw,/tikz/dim arrow/draw short dim arrow,/tikz/dim arrow/raise=-\tmp@tdar}\pgfkeys{/tikz/dim arrow/.cd,#1}},
  /tikz/dim arrow/.cd,
  raise/.initial={0.5cm},
  no raise/.style={raise=0},
  label/.code={\pgfkeys{/tikz/dim arrow/label text=#1}},
  label'/.code={\pgfkeys{/tikz/dim arrow/label text=#1,/tikz/dim arrow/label style/.append style={swap},}},
  label text/.initial={},
  label style/.style={},
  label near start/.code={\def\dimarrow@short@position{0}}, % only short
  label near middle/.code={\def\dimarrow@short@position{2}}, % only short
  label near end/.code={\def\dimarrow@short@position{1}}, % only short
  arrow length/.initial={5mm}, % only for short
  draw short dim arrow/.style={to path={\pgfextra{%
    \let\tikz@mode@save=\tikz@mode%
        \let\tikz@options@save=\tikz@options%
    \newdimen\labelTotalRaise
    \pgfmathsetlength\labelTotalRaise{\pgfkeysvalueof{/tikz/dim arrow/raise}}
        \pgfinterruptpath
        \draw[>=technical,->|] \pgfextra{\let\tikz@mode=\tikz@mode@save\let\tikz@options=\tikz@options@save}
    let
        \p1=($(\tikztostart)!\pgfkeysvalueof{/tikz/dim arrow/raise}!90:(\tikztotarget)$),
        \p2=($(\tikztotarget)!\pgfkeysvalueof{/tikz/dim arrow/raise}!-90:(\tikztostart)$)
        in ($(\p1)!-\pgfkeysvalueof{/tikz/dim arrow/arrow length}!(\p2)$) -- ($(\p1)!0!(\p2)$);
    \draw[>=technical,->|] \pgfextra{\let\tikz@mode=\tikz@mode@save\let\tikz@options=\tikz@options@save}
    let
        \p1=($(\tikztostart)!\pgfkeysvalueof{/tikz/dim arrow/raise}!90:(\tikztotarget)$),
        \p2=($(\tikztotarget)!\pgfkeysvalueof{/tikz/dim arrow/raise}!-90:(\tikztostart)$)
        in ($(\p2)!-\pgfkeysvalueof{/tikz/dim arrow/arrow length}!(\p1)$) -- ($(\p2)!0!(\p1)$);
    \ifnum\dimarrow@short@position=0
    \path let 
        \p1=($(\tikztostart)!\labelTotalRaise!90:(\tikztotarget)$),
        \p2=($(\tikztotarget)!\labelTotalRaise!-90:(\tikztostart)$)
        in let
    \p3=($(\p1)!-1*\pgfkeysvalueof{/tikz/dim arrow/arrow length}!(\p2)$),
    \p4=($(\p1)!-0*\pgfkeysvalueof{/tikz/dim arrow/arrow length}!(\p2)$)
    in
    (\p3) -- (\p4) node[pos=0.5,auto=left,/tikz/dim arrow/label style] {\pgfkeysvalueof{/tikz/dim arrow/label text}};
  \fi
    \ifnum\dimarrow@short@position=1
      \path let 
        \p1=($(\tikztostart)!\labelTotalRaise!90:(\tikztotarget)$),
        \p2=($(\tikztotarget)!\labelTotalRaise!-90:(\tikztostart)$)
        in let
    \p3=($(\p2)!-1*\pgfkeysvalueof{/tikz/dim arrow/arrow length}!(\p1)$),
    \p4=($(\p2)!-0*\pgfkeysvalueof{/tikz/dim arrow/arrow length}!(\p1)$)
    in
    (\p4) -- (\p3) node[pos=0.5,auto=left,/tikz/dim arrow/label style] {\pgfkeysvalueof{/tikz/dim arrow/label text}};
    \fi
  \ifnum\dimarrow@short@position=2
    \path let 
        \p1=($(\tikztostart)!\labelTotalRaise!90:(\tikztotarget)$),
        \p2=($(\tikztotarget)!\labelTotalRaise!-90:(\tikztostart)$)
    in
    (\p1) -- (\p2) node[pos=0.5,/tikz/dim arrow/label style] {\pgfkeysvalueof{/tikz/dim arrow/label text}};
  \fi
        \endpgfinterruptpath
      }(\tikztostart) (\tikztotarget) \tikztonodes
    }
  },
  draw dim arrow/.style={to path={\pgfextra{%
    \let\tikz@mode@save=\tikz@mode%
        \let\tikz@options@save=\tikz@options%
    \newdimen\labelTotalRaise
    \pgfmathsetlength\labelTotalRaise{\pgfkeysvalueof{/tikz/dim arrow/raise}}
        \pgfinterruptpath
        \draw[>=technical,|<->|] \pgfextra{\let\tikz@mode=\tikz@mode@save\let\tikz@options=\tikz@options@save}
    let 
        \p1=($(\tikztostart)!\pgfkeysvalueof{/tikz/dim arrow/raise}!90:(\tikztotarget)$),
        \p2=($(\tikztotarget)!\pgfkeysvalueof{/tikz/dim arrow/raise}!-90:(\tikztostart)$)
        in (\p1) -- (\p2);
    \path let 
        \p1=($(\tikztostart)!\labelTotalRaise!90:(\tikztotarget)$),
        \p2=($(\tikztotarget)!\labelTotalRaise!-90:(\tikztostart)$)
        in (\p1) -- (\p2) node[pos=0.5,auto=left,/tikz/dim arrow/label style] {\pgfkeysvalueof{/tikz/dim arrow/label text}};
    % rq : inner sep controle la distance chemin-node
        \endpgfinterruptpath
      }(\tikztostart) (\tikztotarget) \tikztonodes
    }
  },
}

%%%%%%%%%%%%%%%%%%%%%%%%%%%%%%%%%%%%%%%%%%%%%%%%%%%%%%%%%%%%%%%%%%%%%%%%%%%%%%%%
%%%%%%%%%%%%%%%%%%%%%%%%%%%%%%%%%%%%%%%%%%%%%%%%%%%%%%%%%%%%%%%%%%%%%%%%%%%%%%%%
% Arrows
%%%%%%%%%%%%%%%%%%%%%%%%%%%%%%%%%%%%%%%%%%%%%%%%%%%%%%%%%%%%%%%%%%%%%%%%%%%%%%%%
%%%%%%%%%%%%%%%%%%%%%%%%%%%%%%%%%%%%%%%%%%%%%%%%%%%%%%%%%%%%%%%%%%%%%%%%%%%%%%%%

%%%%%%%%%%%%%%%%%%%%%%%%%%%%%%%%%%%%%%%%%%%%%%%%%%%%%%%%%%%%%%%%%%%%%%%%%%%%%%%%
% Arrow lens arrow
% used to draw lenses (perhaps not the best idea).
%%%%%%%%%%%%%%%%%%%%%%%%%%%%%%%%%%%%%%%%%%%%%%%%%%%%%%%%%%%%%%%%%%%%%%%%%%%%%%%%
\pgfsetarrowoptions{lens arrow@length}{6pt}
\pgfsetarrowoptions{lens arrow@angle}{50}
\pgfarrowsdeclare{lens arrow}{lens arrow}
{
   \pgfarrowsleftextend{0pt}
   \pgfarrowsrightextend{0pt}
}
{
  \pgfsetroundcap
  \pgfsetmiterjoin
  \pgfmathsetlength{\pgfutil@tempdimb}{\pgfgetarrowoptions{lens arrow@length}*sin(\pgfgetarrowoptions{lens arrow@angle}/2)}    
  \def\arrow@origin{\pgfpoint{0pt}{0pt}}
  \pgfutil@tempdima=\pgfgetarrowoptions{lens arrow@length}%
  \pgfmathsetmacro{\tmp@lens@angle}{90+\pgfgetarrowoptions{lens arrow@angle}}
  \pgfmathsetmacro{\tmp@lens@anglediv}{\pgfgetarrowoptions{lens arrow@angle}/2}
  \advance\pgfutil@tempdima by -1.5\pgflinewidth%
  \pgfmathsetlength{\pgfutil@tempdima}{\pgfutil@tempdima/cos(\pgfgetarrowoptions{lens arrow@angle}/2)}    
  \pgfpathmoveto{\pgfpointadd{\arrow@origin}{\pgfqpointpolar{\tmp@lens@angle}{\pgfutil@tempdima}}}
  \pgfpathlineto{\arrow@origin}
  \pgfpathlineto{\pgfpointadd{\arrow@origin}{\pgfqpointpolar{-\tmp@lens@angle}{\pgfutil@tempdima}}}
  \pgfusepathqstroke
}
\pgfarrowsdeclarereversed{lens arrow reversed}{lens arrow reversed}{lens arrow}{lens arrow}


%%%%%%%%%%%%%%%%%%%%%%%%%%%%%%%%%%%%%%%%%%%%%%%%%%%%%%%%%%%%%%%%%%%%%%%%%%%%%%%%
% Ray arrow
% It is useful to have an arrow which goes on the exact middle of a path.
% This is used on ->-, etc.
%%%%%%%%%%%%%%%%%%%%%%%%%%%%%%%%%%%%%%%%%%%%%%%%%%%%%%%%%%%%%%%%%%%%%%%%%%%%%%%%
% flèche utilisée pour marquer les rayons lumineux (avec les styles ->-, etc.)
\pgfsetarrowoptions{ray arrow@length}{4pt}
\pgfsetarrowoptions{ray arrow@angle}{45}

\makeatletter

\pgfsetarrowoptions{multiple ray arrow}{0}
\pgfarrowsdeclare{multiple ray arrow}{multiple ray arrow}
{
    \pgfarrowsleftextend{0pt}
    \pgfarrowsrightextend{0pt}
}
{
  \pgfsetroundcap
  \pgfsetmiterjoin
  \pgfutil@tempdima=\pgfgetarrowoptions{ray arrow@length}%
  \pgfmathsetmacro{\tmp@ray@angle}{90+\pgfgetarrowoptions{ray arrow@angle}}
  \pgfmathsetmacro{\tmp@ray@anglediv}{\pgfgetarrowoptions{ray arrow@angle}/2}
  \advance\pgfutil@tempdima by -1.5\pgflinewidth%
  \pgfmathsetlength{\pgfutil@tempdima}{\pgfutil@tempdima/cos(\pgfgetarrowoptions{ray arrow@angle}/2)}
  %
  \foreach \i in {1,...,\pgfgetarrowoptions{multiple ray arrow}}
  {
    \pgfmathsetlength{\pgfutil@tempdimb}{(2*\i-\pgfgetarrowoptions{multiple ray arrow})*\pgfgetarrowoptions{ray arrow@length}*sin(\pgfgetarrowoptions{ray arrow@angle}/2)}
    %
    \def\arrow@origin{\pgfpoint{\pgfutil@tempdimb}{0pt}}
    %
    \pgfpathmoveto{\pgfpointadd{\arrow@origin}{\pgfqpointpolar{\tmp@ray@angle}{\pgfutil@tempdima}}}
    \pgfpathlineto{\arrow@origin}
    \pgfpathlineto{\pgfpointadd{\arrow@origin}{\pgfqpointpolar{-\tmp@ray@angle}{\pgfutil@tempdima}}}
    \pgfusepathqstroke
  }
}


%%%%%%%%%%%%%%%%%%%%%%%%%%%%%%%%%%%%%%%%%%%%%%%%%%%%%%%%%%%%%%%%%%%%%%%%%%%%%%%%
% Arrow technical
%%%%%%%%%%%%%%%%%%%%%%%%%%%%%%%%%%%%%%%%%%%%%%%%%%%%%%%%%%%%%%%%%%%%%%%%%%%%%%%%
\makeatletter
\pgfarrowsdeclare{technical}{technical}
{%
  \pgfutil@tempdima=0.48pt%
  \pgfutil@tempdimb=\pgflinewidth%
  \ifdim\pgfinnerlinewidth>0pt%
    \pgfmathsetlength\pgfutil@tempdimb{.6\pgflinewidth-.4*\pgfinnerlinewidth}%
  \fi%
  \advance\pgfutil@tempdima by.3\pgfutil@tempdimb%
  \pgfarrowsleftextend{+-3\pgfutil@tempdima}%
  \pgfarrowsrightextend{+8\pgfutil@tempdima}%
}
{%
  \pgfutil@tempdima=0.48pt%
  \pgfutil@tempdimb=\pgflinewidth%
  \ifdim\pgfinnerlinewidth>0pt%
    \pgfmathsetlength\pgfutil@tempdimb{.6\pgflinewidth-.4*\pgfinnerlinewidth}%
  \fi%
  \advance\pgfutil@tempdima by.3\pgfutil@tempdimb%
  \pgfpathmoveto{\pgfqpoint{8\pgfutil@tempdima}{0pt}}%
  \pgfpathlineto{\pgfqpoint{-3\pgfutil@tempdima}{3\pgfutil@tempdima}}%
  \pgfpathlineto{\pgfpointorigin}%
  \pgfpathlineto{\pgfqpoint{-3\pgfutil@tempdima}{-3\pgfutil@tempdima}}%
  \pgfusepathqfill%
}

\pgfarrowsdeclare{technical reversed}{technical reversed}
{%
  \pgfutil@tempdima=0.48pt%
  \pgfutil@tempdimb=\pgflinewidth%
  \ifdim\pgfinnerlinewidth>0pt%
    \pgfmathsetlength\pgfutil@tempdimb{.6\pgflinewidth-.4*\pgfinnerlinewidth}%
  \fi%
  \advance\pgfutil@tempdima by.3\pgfutil@tempdimb%
  \pgfarrowsleftextend{-8\pgfutil@tempdima}
  \pgfarrowsrightextend{-8\pgfutil@tempdima}
}
{%
  \pgfutil@tempdima=0.48pt%
  \pgfutil@tempdimb=\pgflinewidth%
  \ifdim\pgfinnerlinewidth>0pt%
    \pgfmathsetlength\pgfutil@tempdimb{.6\pgflinewidth-.4*\pgfinnerlinewidth}%
  \fi%
  \advance\pgfutil@tempdima by.3\pgfutil@tempdimb%
  \pgfpathmoveto{\pgfqpoint{-8\pgfutil@tempdima}{0pt}}%
  \pgfpathlineto{\pgfqpoint{3\pgfutil@tempdima}{3\pgfutil@tempdima}}%
  \pgfpathlineto{\pgfpointorigin}%
  \pgfpathlineto{\pgfqpoint{3\pgfutil@tempdima}{-3\pgfutil@tempdima}}%
  \pgfusepathqfill%
}


% Changelog:
% 2013-10-21 : ajout du style |distance arrow| et de la décoration |line| correspondante.
% 2013-11-19 : suppression de |distance arrow| et ajout à la place de |dim arrow| et assimilés
% 2013-11-22 : choix entre distances relatives et absolues (http://www.texample.net/tikz/examples/supersonic-nozzle/)
% 2013-11-24 : styles de flèches |->-|, |-<-|, |->>-|, |-<<-| (et flèches pgf |ray arrow|, etc. correspondantes)
% 2013-11-24 : flèches pgf |lens arrow| et |lens arrow reversed|
% 2013-11-24 : |generic optics element| -> |thin optics element| et |thick optics element| ; conséquences. |beam splitter|
% 2014-01-01 : ajout de |double amici prism|,  |optics| -> |use optics| et |one arrow| -> |put arrow| ; |mark a *| supprimés
% 2014-03-19 : anchorborder pour |generic optics io| (les labels devraient donc être placés correctement)
%              |io body aspect ratio| accepte désormais aussi des longueurs absolues, ajout d'un alias |io body width| pour |io body aspect ratio|
% 2014-06-26 : modification du code des flèches |->-|, |->>-|, etc. et ajout de |->>>-|, |->>>>-|,|->n-=<nombre>| (idem dans l'autre sens)
% 2014-07-08 : ajout de |arrow style| à |put arrow|
% 2014-09-20 : ajout de |spherical mirror| et quelques modifications à |mirror| (ajustement de la décoration et de ses réglages par défaut)
% 2014-09-22 : ajustements de |spherical mirror| (concave et convexe), et ajout des styles correspondants |convex mirror| et |concave mirror|
% 2014-09-24 : ajustements de |spherical mirror| (ltr/rtl) ; correction des ancres de generic optics io (aperture north, aperture center, aperture south étaient incorrectes)
% 2014-09-25 : corrections à |spherical mirror| (ltr/rtl vs concave/convex) ; 
% 2014-10-02 : ajout d'une fonction |from_radius| pour définir l'angle d'ouverture de |spherical mirror|, encore des corrections à |spherical mirror| (ltr/rtl vs concave/convex) ; 
%               macro pour les messages d'erreur
% 2014-10-03 : vérifications de cohérence des grandeurs pour |slit| et |double slit| ; messages d'erreur au besoin
% 2014-12-07 : modifications substantielles à |put arrow| et |optics/->n-|, etc. pour pouvoir avoir plusieurs flèches sur le même chemin ; la compatibilité arrière est brisée.
% 2014-12-11 : nettoyage
% 2015-03-10 : ajout d'un alias |object width| pour |object aspect ratio|, qui accepte désormais aussi des longueurs absolues
% 2015-06-13 : mise en cohérence des noms des points focaux pour le miroir et la lentille (désormais, "focus" et "focal point")
% 2016-11-21 : appel aux biblothèques tikz |decorations| et |decorations.pathreplacing| qui sont nécessaires
\makeatother


\usepackage{hyperref}
\usepackage{bookmark}
\hypersetup{%
  colorlinks=true,
  linkcolor=blue,
  filecolor=blue,
  urlcolor=blue,
  citecolor=blue,
  pdfborder=0 0 0,
}


%%%%%%%%%%%%%%%%%%%%%%%%%%%%%%%%%%%%%%%%%%%%%%%%%%%%%%%%%%%%%%%%%%%%%%%%%%%%%%%%
% Macros du manuel pgf/tikz (version : texlive 2012)
% cf. http://tex.stackexchange.com/questions/72999/is-there-a-listings-configuration-for-tikz-as-used-in-pgf-tikz-manual
%%%%%%%%%%%%%%%%%%%%%%%%%%%%%%%%%%%%%%%%%%%%%%%%%%%%%%%%%%%%%%%%%%%%%%%%%%%%%%%%
\def\pgfautoxrefs{1}
\def\pgfmanual@warning#1{\immediate\write16{! Package pgfmanual Warning: #1}}%

\input{macros/pgfmanual-en-macros}
\input{macros/pgfmanual.prettyprinter.code.tex}
\input{macros/pgfmanual.pdflinks.code.tex}


\pgfkeys{
  % set this to 'false' to disable auto reference generation.
  % However, a little bit runtime overhead will still remain
  % (and the \pgfmanualpdflabel commands will still be used)
  /pdflinks/codeexample links=true,
  %
  /codeexample/prettyprint/cs arguments/tikzset/.initial=1,
  /codeexample/prettyprint/cs/tikzset/.code 2 args={\pgfmanualpdfref{#1}{#1}\{\pgfmanualprettyprintpgfkeys{#2}\pgfmanualclosebrace},
  /codeexample/prettyprint/cs arguments/usetikzlibrary/.initial=1,
  /codeexample/prettyprint/cs/usetikzlibrary/.code 2 args={\pgfmanualpdfref{#1}{#1}\{\pgfmanualpdfref{#2}{#2}\pgfmanualclosebrace},
  /codeexample/prettyprint/cs arguments/usepgflibrary/.initial=1,
  /codeexample/prettyprint/cs/usepgflibrary/.code 2 args={\pgfmanualpdfref{#1}{#1}\{\pgfmanualpdfref{#2}{#2}\pgfmanualclosebrace},
  %
  %
  %
  % whenever an unqualified key is found, the following key prefix
  % list is tried to find a match.
  /pdflinks/search key prefixes in={/tikz/,/pgf/,/tikz/optics/,/tikz/dim/arrow/,},
  %
  % the link prefix written to the pdf file:
  /pdflinks/internal link prefix=tikzopt,
  %
  /pdflinks/warnings=false,
  % for debugging:
  /pdflinks/show labels=false,
}%

%%%%%%%%%%%%%%%%%%%%%%%%%%%%%%%%%%%%%%%%%%%%%%%%%%%%%%%%%%%%%%%%%%%%%%%%%%%%%%%%
% Global styles from pgf/tikz manual (version : texlive 2012)
%%%%%%%%%%%%%%%%%%%%%%%%%%%%%%%%%%%%%%%%%%%%%%%%%%%%%%%%%%%%%%%%%%%%%%%%%%%%%%%%
\tikzset{
  every plot/.style={prefix=plots/pgf-},
  shape example/.style={
    color=black!30,
    draw,
    fill=yellow!30,
    line width=.5cm,
    inner xsep=2.5cm,
    inner ysep=0.5cm}, 
  line shape example/.style={
    color=black!30,
    draw,
%    fill=black!30, % utile à quoi ?
    line width=.5cm,
    inner xsep=2.5cm,
    inner ysep=0.5cm}
}

%%%%%%%%%%%%%%%%%%%%%%%%%%%%%%%%%%%%%%%%%%%%%%%%%%%%%%%%%%%%%%%%%%%%%%%%%%%%%%%%
% Specific environnements
%%%%%%%%%%%%%%%%%%%%%%%%%%%%%%%%%%%%%%%%%%%%%%%%%%%%%%%%%%%%%%%%%%%%%%%%%%%%%%%%

% Environnement |example only| : aspect similaire à codeexample, mais avec seulement l'exemple (quand le code est trop long)
\newmdenv[
  backgroundcolor=graphicbackground,
  linewidth=0pt,
  innerleftmargin=0.09cm,innerrightmargin=0.09cm,innertopmargin=0.09cm,
  leftmargin=0pt,rightmargin=0pt,
  skipabove=5pt, skipbelow=-2pt,
  ]{example only}

% Environnement |warning|
\colorlet{warningbackground}{red!30}
\makeatletter
\newenvironment{warning}{%
\mdfsetup{%
  backgroundcolor=warningbackground,
  skipabove=5pt,
  topline=false,bottomline=false,rightline=false,
  innerbottommargin=4pt,innertopmargin=0.18cm,
  leftmargin=0cm, innerleftmargin=0.3cm
    rightmargin=0cm,innerrightmargin=0.3cm,
    linewidth=1pt,linecolor=none, topline=false, bottomline=false,
    }%
\begin{mdframed}%
\setlength{\parindent}{0pt}%
\danger\,%
}%
{%
\end{mdframed}%
}%

% Environnement |update|
\colorlet{updatebackground}{blue!30}
\newenvironment{update}{%
\mdfsetup{%
  backgroundcolor=updatebackground,
  skipabove=5pt,
  topline=false,bottomline=false,rightline=false,
  innerbottommargin=4pt,innertopmargin=0.18cm,
  leftmargin=0cm, innerleftmargin=0.3cm
    rightmargin=0cm,innerrightmargin=0.3cm,
    linewidth=1pt,linecolor=none, topline=false, bottomline=false,
    }%
\begin{mdframed}%
\setlength{\parindent}{0pt}%
\danger\,%
}%
{%
\end{mdframed}%
}%
\makeatother

\begin{document}

\VerbatimFootnotes

\title{optics --- a tikz library for optics drawings}
\author{Michel Fruchart \\
\href{mailto:michel (dot) fruchart [at] ens-lyon (dot) org}{\texttt{michel (dot) fruchart [at] ens-lyon (dot) org}}}

\date{\tikzopticsversiondate{} -- version \tikzopticsversion \\[0.15cm]
\large\href{https://github.com/fruchart/tikz-optics}{\texttt{https://github.com/fruchart/tikz-optics}}}

\maketitle

\section{Introduction}

The aim of this library is to ease the creation of optical schematics including lenses, mirrors, and so on.
The physically (in)accurate ray tracing is left to the user.


\subsection{Legal matters}

This library can be distributed and modified under the LaTeX Project Public License (LPPL), version 1.3c\footnote{\url{http://latex-project.org/lppl/lppl-1-3c.txt}}. It can also be distributed and modified under the GNU General Public License (GNU GPL), either version 2\footnote{\url{http://www.gnu.org/licenses/gpl-2.0.en.html}}, or any ulterior version published by the Free Software Foundation.
Its documentation (that you are currently reading) can be distributed and modified under the LaTeX Project Public License (LPPL), version 1.3c\footnote{\url{http://latex-project.org/lppl/lppl-1-3c.txt}}. It can also be distributed and modified under the GNU Free Documentation License (GNU FDL), either version 1.3\footnote{\url{https://www.gnu.org/licenses/fdl-1.3.en.html}}, or any ulterior version published by the Free Software Foundation.

\subsection{Installation of the library}

If you have an up-to-date Texlive installation, the |tikz-optics| library might already be installed on your system. Try using the command |\usetikzlibrary{optics}| in your \TeX{} file. If it does not work, you should either update your \TeX{} installation or install the library manually (see below).

This library is a \enquote{tikz library}. It can be used it two ways:
\begin{itemize}
  \item add the file |tikzlibraryoptics.code.tex| in a folder where \TeX{} can find it, for instance in your |TEXMFHOME| folder\footnote{It can be found using \verb|kpsewhich -var-value TEXMFHOME|. }, then use the command |\usetikzlibrary{optics}| in your \TeX{} code;
  \item directly include the file |tikzlibraryoptics.code.tex| with the |\input| command.
\end{itemize}

When the library is installed in a |TEXMF| folder, say |/home/agamemnon/texmf/|, the TDS structure must be respected for the library to be found\footnote{See e.g. \url{https://www.ctan.org/TDS-guidelines}}. As a consequence, the file |tikzlibraryoptics.code.tex| should be put in the folder |home/agamemnon/texmf/tex/latex/|, or in a subdirectory such as |home/agamemnon/texmf/tex/latex/tikzoptics|.

\subsection{Basic usage}

\begin{key}{/tikz/use optics}
Once the library is installed, it can be loaded with |\usetikzlibrary{optics}|. Then, it should be activated by applying the key |use optics| to a |tikzpicture|\footnote{The key \texttt{use optics} loads the relevant elements from \texttt{/tikz/optics/} to \texttt{/tikz/}, so they can be used directly. They are initially kept in a separate namespace to reduce the risk of name collision.}

\begin{codeexample}[width=5cm]
\begin{tikzpicture}[use optics]
  \node[lens] at (0,0) {};
  \node[mirror] at (1cm,0) {};
\end{tikzpicture}
\end{codeexample}

If |use optics| is not used, the code will fail miserably (in best case scenarios) or will display unexpected behaviors (in other scenarios).
\end{key}

\section{Examples}

\subsection{Image of a slit on a screen}

Let us draw a simple schematic: the image of a slit on a screen.

The first step consists in positioning all elements. We start with the light source, located in |(0,0)|. 
Each further element is added to the right of the previous ones.
For instance, the heat filter is located at |0.5cm| of the halogen lamp output (anchor |aperture east| of the node |node| |quartz iode|) with the code |right=0.5cm of (quartz iode.aperture east)|. I chose to position the screen and lens with respect to the slit, but other choices are equally possible.

\begin{codeexample}[width=5cm]
\begin{tikzpicture}[use optics]
  \node[halogen lamp] (quartz iode) at (0,0) {Q.I.};
  \node[heat filter,right=0.5cm of quartz iode.aperture east] (AC) {};
  \node[slit,right=0.75cm of AC] (fente) {};
  \node[lens,right=2cm of fente] (L) {};
  \node[screen,right=5cm of fente] (screen) {};
\end{tikzpicture}
\end{codeexample}

%
We then wish to draw the light rays. Here, we are in luck: all the needed positions are predefined anchors of the existing |node|s.
For instance, |(L.lens south)| is a bit before the very end of the lens |L| to make the picture nicer (the very end can still be accessed with the anchor |(L.south)|). Let us then connect the relevant anchors with |--| in a |\draw| command.
%
\begin{codeexample}[]
\begin{tikzpicture}[use optics]
  \node[halogen lamp] (quartz iode) at (0,0) {Q.I.};
  \node[heat filter,right=0.5cm of quartz iode.aperture east] (AC) {} ;
  \node[slit,right=0.75cm of AC] (fente) {};
  \node[lens,right=2cm of fente] (L) {};
  \node[screen,right=5cm of fente] (screen) {};

  \draw[red] (fente.slit north) -- (L.lens north) -- (screen.center)
  (fente.slit south) -- (L.lens south) -- (screen.center);
\end{tikzpicture}
\end{codeexample}
%
The next step consists in adding labels. Surprisingly, this is done with the \verb|label| key, with a code like \verb|label=|\meta{texte}, or more generally \verb|label=[|\meta{opts}\verb|]|\meta{pos}\verb|:|\meta{text}. As an example, |align=center| will be required for multiline labels.

\begin{codeexample}[]
\begin{tikzpicture}[use optics]
  \node[halogen lamp] (quartz iode) at (0,0) {Q.I.};
  \node[heat filter,right=0.5cm of quartz iode.aperture east,label={below:AC}] (AC) {} ;
  \node[slit,right=0.75cm of AC,label={below:fente}] (fente) {};
  \node[lens,right=2cm of fente,label={[align=center]below:achromat \\ $(L)$}] (L) {};
  \node[screen,right=5cm of fente,label={below:\'ecran}] (screen) {};

  \draw[red] (fente.slit north) -- (L.lens north) -- (screen.center)
  (fente.slit south) -- (L.lens south) -- (screen.center);
\end{tikzpicture}
\end{codeexample}

Finally, we add arrows to indicate the different lengths. 
To do so, we'll define a |coordinate| (a particular |node| that is not drawn, and only has one anchor). 
The vertical position of this |coordinate| (point 1) and the horizontal coordinates of the (centers of the) various objects (point 2) define new positions through the |tikz| syntax \texttt{(point 1 -\textbar{} point 2)}.


\begin{codeexample}[]
\begin{tikzpicture}[use optics]
  \node[halogen lamp] (quartz iode) at (0,0) {Q.I.};
  \node[heat filter,right=0.5cm of quartz iode.aperture east,label={below:AC}] (AC) {} ;
  \node[slit,right=0.75cm of AC,label={below:fente}] (fente) {};
  \node[lens,right=2cm of fente,label={[align=center]below:achromat \\ $(L)$}] (L) {};
  \node[screen,right=5cm of fente,label={below:\'ecran}] (screen) {};

  \draw[red] (fente.slit north) -- (L.lens north) -- (screen.center)
  (fente.slit south) -- (L.lens south) -- (screen.center);

  \coordinate (arrow origin) at (0,1.5cm);

  \draw[>=technical,<->] (arrow origin -| fente) -- (arrow origin -| L) node[midway,above] {$d$};
  \draw[>=technical,<->] (arrow origin -| L) -- (arrow origin -| screen) node[midway,above] {$D$};
\end{tikzpicture}
\end{codeexample}

It is also possible to use the style |dim arrow| instead.


\subsection{Interferences}

Here, we compute the nodes' position instead of using |right=of| and friends.


\begin{codeexample}[]
\begin{tikzpicture}[use optics]
  \node[laser] (L) at (0,0) {\ce{HeNe}};
  \node[semi-transparent mirror,rotate=45] (ST) at ($(L)+(3cm,0)$) {};
  \node[above] at (ST.north) {s\'eparatrice};
  \node[mirror,rotate=-135] (M1) at ($(ST)+(0,-3cm)$) {};
  \node[mirror,rotate=-45] (M2) at ($(M1)+(5cm,0)$) {};
  \node[sensor line,rotate=45,anchor=pixel 3 west,label={[label distance=0.5cm]above right:CCD}]
    (CCD) at ($(ST)+(5cm,0)$) {};
  \draw[red] (L.aperture east) -- (ST.center) -- (M1.center) -- (M2.center) -- (CCD.pixel 3 west);
  \draw[red] (L.aperture east) -- (ST.center) -- (CCD.pixel 3 west);
\end{tikzpicture}
\end{codeexample}

The code |anchor=pixel 3 west| is used so that the center of the receptor is located at |($(ST)+(5cm,0)$)|. 

We also showcase various ways of adding a label: creating a |node| at the right place with the relevant text (here \emph{séparatrice}), using the \verb|label| key, etc. (see the pgf/tikz manual).

\subsection{Dispersion}
\label{sec:exemple_dispersion}

The syntax |($(A)!0.6!(B)$)| (\emph{partway modifiers}, read the tikz manual for details) allows to create a point located at the position |0.6| between the points |(A)| et |(B)| (|0| correspond à |(A)| et |1| à |(B)|).

\begin{codeexample}[]
\begin{tikzpicture}[use optics]
  % [align=center] permet les labels multiligne
  % [font=\footnotesize] fait un texte plus petit
  \tikzset{every label/.style={align=center,font=\footnotesize}}
   
  \node[halogen lamp] (S) at (0,0) {QI};
  \node[heat filter, right=of S] (AC) {};
  \node[diaphragm, right=0.1cm of AC] (diaphragme) {};
  \node[slit, right=0.5cm of diaphragme,label={south:{trou \\ \SI{5}{\milli\meter}}}] (T) {};
  \node[lens,right=2cm of T] (L) {};
  \node[screen,right=2cm of L] (ecran) {};

  \draw[red]
  (T.slit north) -- (L.lens north) -- (ecran.center)
  (T.slit south) -- (L.lens south) -- (ecran.center);

  \draw[blue]
  (T.slit north) -- (L.lens north) -- ($(ecran.north)!0.6!(ecran.south)$)
  (T.slit south) -- (L.lens south) -- ($(ecran.north)!0.4!(ecran.south)$);
\end{tikzpicture}
\end{codeexample}


\subsection{A Cassegrain telescope}

In order to draw a mirror with a hole, we use the tikz command |\clip| (placed in a |scope|, so that only the content of the scope is affected by the |\clip|)
We then draw the light rays manually.

\begin{codeexample}[]
\begin{tikzpicture}[use optics]
% mirror with hole
\begin{scope}
  \clip (-0.75cm,-2.2cm) rectangle (1cm,0-0.33cm) (-0.75cm,2.2cm) rectangle (1cm,0+0.33cm);
  \node[spherical mirror, object height=4cm, spherical mirror angle=50] (M1) at (0cm,0) {};
\end{scope}

% small mirror
\node[convex mirror, spherical mirror orientation=rtl,
    object height=1cm, spherical mirror angle=90] (M2) at (-4cm,0) {};

% convergence point
\coordinate (F) at (1cm,0);

% red ray
\begin{scope}[red]
  \draw[-<-] (M1.22) coordinate (P1) -- +(-5cm,0);
  \draw[->-] (P1) -- (M2.30) coordinate (Q1);
  \draw[->-] (Q1) -- ($(Q1)!1.25!(F)$) coordinate (R1);
\end{scope}

% blue ray
\begin{scope}[blue]
  \draw[-<<-] (M1.-22) coordinate (P2) -- +(-5cm,0);
  \draw[->>-] (P2) -- (M2.-30) coordinate (Q2);
  \draw[->>-] (Q2) -- ($(Q2)!1.25!(F)$) coordinate (R2);
\end{scope}

% violet ray
\begin{scope}[violet]
  \draw[-<-] (M1.22) coordinate (P3) -- +(175:5cm);
  \draw (P3) -- (M2.22) coordinate (Q3);
  \draw (Q3) -- ($(Q3)!1.25!($(F)+(0,-0.15cm)$)$) coordinate (R3);
\end{scope}

% sensor
\node[generic sensor, anchor=aperture west] at ($(R1)!0.5!(R2)$) {};
\end{tikzpicture}
\end{codeexample}

\subsection{Ray optics and computations}

\begin{codeexample}[]
\begin{tikzpicture}[use optics]
  \node[lens,draw focal points,focal length=1.5cm,object height=2cm] (L) at (0,0) {};
  \coordinate (P) at (-2cm,0.5cm);
  \coordinate (Q) at (-2cm,-0.5cm);
  \draw[red,shorten >=-1cm] (P) -- ($(L.north)!(P)!(L.south)$) -- (L.east focus);
  \draw[red,shorten >=-1cm] (Q) -- ($(L.north)!(Q)!(L.south)$) -- (L.east focus);
  \node[screen] at (2.45cm,0) {};
\end{tikzpicture}
\end{codeexample}

  The preceding example is not very clean, because I had to manually enlarge the rays up to the screen.
  The next example where the intersection is first computed with a macro |\toVerticalProjection| is way nicer.

\begin{codeexample}[]
\begin{tikzpicture}[use optics]
  \node[lens,draw focal points,focal length=1.5cm,object height=2cm] (L) at (0,0) {} ;
  \coordinate (P) at (-2cm,0.5cm) ;
  \coordinate (Q) at (-2cm,-0.5cm) ;
  \node[screen] (S) at (2.5cm,0) {};

  \def\toVerticalProjection#1#2#3{let \p{1} = #1, \p{2} = #2, \p{3} = #3 in
   -- (\x{3},{\y{1}+(\y{2}-\y{1})/(\x{2}-\x{1})*(\x{3}-\x{1})})}

  \draw[red] (P) -- ($(L.north)!(P)!(L.south)$) coordinate (Plens)
  \toVerticalProjection{(Plens)}{(L.east focus)}{(S)};

  \draw[red] (Q) -- ($(L.north)!(Q)!(L.south)$) coordinate (Qlens)
  \toVerticalProjection{(Qlens)}{(L.east focus)}{(S)};

\end{tikzpicture}
\end{codeexample}

\subsection{Birefringence with a Michelson interferometer}

  \begin{tikzpicture}[use optics]

  \node[semi-transparent mirror,rotate=-45,label={above:$(S)$}] (CS) at (0,0) {};
  % quand on tourne les node, il faut parfois préciser les ancres à utiliser pour que le résultat soit celui attendu
  \node[mirror,rotate=90, above=3cm of CS.center,anchor=center,label={below:$(M_1)$}] (M1) {};
  \node[mirror,rotate=-20, right=3cm of CS.center,anchor=center,label={[xshift=-0.5cm]above:$(M_2)$}] (M2) {};

  \node[draw,fit=(CS) (M1) (M2),inner sep=0.5cm] (support Michelson) {};

  \node[semi-transparent mirror,rotate=45,left=3cm of CS] (CSa) {};

  \node[laser,rotate=-90,above=3cm of CSa.center,anchor=center] (laser) {laser};

  \node[polarizer,left=1.5cm of CSa,label=above:$P_1$] (P1) {};
  \node[thick optics element,left=0.5cm of P1,label=below:quartz,object aspect ratio=0.2] (quartz) {};
  \node[polarizer,left=0.5cm of quartz,label=above:$P_2$] (P2) {};
  \node[lens,left=1.5cm of P2,lens height=0.6] (condenseur) {};
  \node[heat filter,left=1cm of condenseur,label=below:AC] (AC) {};
  \node[halogen lamp,left=0.5cm of AC,anchor=aperture east] (QI) {QI};


  \node[semi-transparent mirror,rotate=45,below=2cm of CS] (CSb) {};
  \node[lens,rotate=90,below=1.5cm of CSb.center,anchor=center,lens height=0.6] (Lp) {};
  \node[screen,rotate=90,below=2cm of Lp.center,anchor=center] (S) {};
  \node[screen,right=3cm of CSb.center,anchor=center] (S') {};


  \begin{scope}[ocg={name=Rayons lumineux,ref=raylum,status=invisible}]
  \draw[red] (laser.aperture east) -- (CSa.center) -- (CS.center)
  (CS.center) -- (M1.center)
  (CS.center) -- (M2.center)
  (CS.center) -- (CSb.center) -- (S'.center);

  % la syntaxe n'est pas très dense quand on a besoin d'utiliser des intersections
  % attention, il faut permutter nord et sud si on tourne le mirroir dans l'autre sens
  \path[name path=cM2n]  (condenseur.lens north) -| (M2.north);
  \path[name path=cM2s]  (condenseur.lens south) -| (M2.north);
  \path[name path=M2ns] (M2.north) -- (M2.south);
  \path[name path=CSns] (CS.north) -- (CS.south);
  \path[name intersections={of=cM2n and M2ns,by=M2 north int}];
  \path[name intersections={of=cM2s and M2ns,by=M2 south int}];
  \path[name intersections={of=cM2n and CSns,by=CS north int}];
  \path[name intersections={of=cM2s and CSns,by=CS south int}];

  \draw[blue]
    (QI.aperture east) -- (condenseur.lens north) -- (M2 north int)
    (M2 south int) -- (condenseur.lens south) -- cycle
    %
    (QI.aperture east) -- (condenseur.lens north) -- 
    (CS north int) |- (M1.center)
    %
    (M1.center) -| (CS south int)
    -- (condenseur.lens south) -- (QI.aperture east)
    %
    (CS north int) -- (Lp.lens south) -- (S.center)
    (CS south int) -- (Lp.lens north) -- (S.center);
  \end{scope}


  \node[draw,circle,line width=1pt,fill=lime!50,switch ocg with mark on={raylum}{}] (rays button) at (-9cm,-5cm) {};
  \node[right=0pt of rays button]{afficher les rayons};

  \end{tikzpicture}

This example uses intersections to draw light rays: the syntax is not the most short and fun to use, but it works.
It is also a good occasion to show how to use the \texttt{ocgx} package with \texttt{tikz} (this only works with select PDF readers).
The code of the figure is attached in the file \textattachfile[]{birefringence_michelson.pgf}{birefringence\_michelson.pgf}.

\subsection{A mirage}

\begin{codeexample}[]
\begin{tikzpicture}[use optics]
  \pgfmathsetmacro\thetaA{110}
  \pgfmathsetmacro\thetaB{65}
  \pgfmathsetmacro\thetaP{100}
  \pgfmathsetmacro\thetaQ{70}
  \pgfmathsetmacro\deltaTheta{5}
  \newdimen\radius
  \pgfmathsetlength\radius{10cm}

  \newdimen\height
  \pgfmathsetlength\height{0.4cm}

  \draw[thick, pattern=north east lines, pattern color=gray] (0,0) 
    arc [start angle=\thetaA, end angle=\thetaB, radius=\radius];
  \path (0,0) arc [start angle=\thetaA, end angle=\thetaP, radius=\radius] coordinate (P);
  \path (0,0) arc [start angle=\thetaA, end angle=\thetaQ, radius=\radius] coordinate (Q);
  \coordinate (QE) at ($(Q)+(\thetaQ:\height)$);

  \node[circle, draw,fill,red, inner sep=0, minimum size=0.1cm] at (P) {};
  \draw[red,->-={at=0.65}] (P) to[out={\thetaP-90}, in={\thetaQ+\deltaTheta+90}] (QE);
  \pic[scale=0.75,rotate={\thetaQ+\deltaTheta-90}] (eye) at (QE) {optics eye};

  \pgfmathsetmacro\thetaTangential{86}
  \path (0,0) arc [start angle=\thetaA, end angle=\thetaTangential, radius=\radius] coordinate (H);
  \draw[densely dashed,shorten >=-3cm] (eye-in) -- (H);
\end{tikzpicture}
\end{codeexample}

\section{Reference}

\subsection{General considerations}

\subsubsection{Common options}

Some options are shared by a lot of |shape|s (with the important exception of light sources and receptors)

\begin{key}{/tikz/optics/object height=\meta{length} (initially \pgfkeysvalueof{/tikz/optics/object height})}
    The key |object height| controls the height of most objects (it is the case if nothing is specified).

\begin{codeexample}[width=5cm]
\begin{tikzpicture}[use optics,scale=.5]
  \node[lens, object height=1cm] (L1) at (0,0) {};
  \node[lens, object height=2cm] (L2) at (3cm,0) {};
\end{tikzpicture}
\end{codeexample}
\end{key}

\begin{key}{/tikz/optics/object aspect ratio=\meta{number or length} (initially \pgfkeysvalueof{/tikz/optics/object aspect ratio})}
    The key |object aspect ratio| controls the aspect ratio of most objects having with a width. When \meta{number} is |1|, the width of the object is equal to its height.
    When for instance \meta{number}|=1/2|, the width is half of the height.
    When \meta{number or length} is a dimensionless number (such as |0.5|), it is interpreted as an aspect ratio (relative to the height).
    When it is a dimensioned length (such as |1cm|), it is directly interpreted as the width of the object.

\begin{codeexample}[width=5cm]
\begin{tikzpicture}[use optics,scale=.5]
  \node[polarizer, object aspect ratio=0.2] (L1) at (0,0) {};
  \node[polarizer, object aspect ratio=0.5] (L2) at (4cm,0) {};
\end{tikzpicture}
\end{codeexample}
\end{key}

\begin{stylekey}{/tikz/optics/object width}
    The key |object width| is an alias for |object aspect ratio|.
\end{stylekey}

\subsubsection{Absolute lengths and relative lengths}

Several elements have more than one adjustable length. Most of the time, it is enough to specify only one length (usually the height of the object) in an absolute way, i.e. with a length unit (|cm|, |pt|, |em|, etc.). The other lengths can be specified as a multiple of this absolute lengths scale (to do so, they should be inputed as a dimensionless number). Sometimes, it is more convenient to specify them as absolute length (a unit must then be used). In the following example, |slit height| is first specified as a multiple of |object height|, then as an absolute length.

\begin{codeexample}[width=6cm]
\begin{tikzpicture}[use optics]
  \draw[style=help lines,gray!50]
    (-3cm,-2cm) grid[step=0.5cm] (2cm,2cm);
  \node[slit,object height=2cm,slit height=0.5,red,very thick]
    at (-2cm,0) {};
  \node[slit,object height=2cm,slit height=0.25,blue,very thick]
    at (-1cm,0) {};
  \node[slit,object height=2cm,slit height=0.5cm,violet,very thick]
    at (0cm,0) {};
  \node[slit,object height=2cm,slit height=1cm,orange,very thick]
    at (1cm,0) {};
\end{tikzpicture}
\end{codeexample}

Using relative lengths allows to rescale the object without modifying its global shape by changing the single absolute length only, instead of having to change all lengths.

Optical elements have several \enquote{heights}. The total height is always called |objet height|. Other heights depend on the element. For instance, the size of a slit is |slit height|, the size of a lens is |lens height| (it does not change the drawing, but moves the anchors), etc.


\subsection{Optical elements}

\subsubsection{Lens}

\begin{shape}{lens}
Draws a lens.

\begin{codeexample}[width=6cm]
\begin{tikzpicture}[use optics,scale=.5]
  \node[lens] (L) at (0,0) {};
\end{tikzpicture}  
\end{codeexample}


\begin{codeexample}[width=6cm]
\begin{tikzpicture}[use optics,scale=.5]
  \node[lens,draw focal points] (L) at (0,0) {};
\end{tikzpicture}
\end{codeexample}

  \begin{stylekey}{/tikz/optics/draw focal points=\meta{style} (default empty)}
  The focal points of the lens can be marked with the key |draw focal points|, and the argument \meta{style} of the key determines how they are draxs.
  For instance, \texttt{draw focal points={red}} produces red markers at the focal points, and \texttt{draw focal points={circle,draw=none,fill=blue}} blue circles at the focal points.
  \end{stylekey}

  \begin{codeexample}[width=6cm]
  \begin{tikzpicture}[use optics,scale=.5]
    \node[lens,draw focal points={red}]
    (L1) at (0,0) {};
    \node[lens,draw focal points={circle,draw=none,fill=blue}]
    (L2) at (3cm,0) {};
  \end{tikzpicture}
\end{codeexample}

\begin{key}{/tikz/optics/object height=\meta{length} (initially \pgfkeysvalueof{/tikz/optics/object height})}
    The key |object height| applies.
\end{key}

\begin{key}{/tikz/optics/focal length=\meta{length} (initially \pgfkeysvalueof{/tikz/optics/focal length})}
    The key |focal length| determines the focal length of the lens.

    \begin{codeexample}[width=6cm,pre={\tikzset{every lens node/.append style={optics,draw focal points}}}]
\begin{tikzpicture}[use optics,scale=.5]
  \node[lens, focal length=1cm] (L1) at (0,0) {};
  \node[lens, focal length=2cm] (L2) at (0,5cm) {};
\end{tikzpicture}
    \end{codeexample}
\end{key}


\begin{key}{/tikz/optics/lens height=\meta{number or length} (initially \pgfkeysvalueof{/tikz/optics/lens height})}
    The key |lens height| determines the height of the lens.
    It can either be an absolute length (with a unit) or a relative length measured in units of the total height of the lens.

    \begin{codeexample}[width=5cm]
\begin{tikzpicture}[use optics,scale=.5]
  \node[lens] (L1) at (1cm,0) {};
  \node[lens, lens height=0.5] (L2) at (-1cm,0) {};
  \draw[red] (0,0) -- (L1.lens north) (0,0) -- (L1.lens south);
  \draw[green] (0,0) -- (L2.lens north) (0,0) -- (L2.lens south);
\end{tikzpicture}
    \end{codeexample}

    % Serait-il plus judicieux d'avoir un système de coordonnées sur la lentille, genre |(L1.0)| pour le centre, |(L1.1)| pour le haut, |(L1.0.5)| pour un point en haut, à mi-lentille, |(L1.-1)| pour le bas, etc. ? Ou est-ce que |($(L1.north)!0.1!(L1.south)$)| suffit amplement ?
\end{key}


\begin{key}{/tikz/optics/lens type}
    The key |lens type| controls the lens type; namely, |lens type=converging| draws a converging lens (default) while |lens type=diverging| draws a diverging lens.

    \begin{codeexample}[width=5cm]
\begin{tikzpicture}[use optics,scale=.5]
  \node[lens,lens type=converging] (L1) at (-1cm,0) {};
  \node[lens,lens type=diverging] (L2) at (1cm,0) {};
\end{tikzpicture}
    \end{codeexample}
\end{key}


The following figure summarizes the anchors defined by |lens|.


\begin{codeexample}[]
\Huge
\begin{tikzpicture}[use optics]
\node[name=s,lens,object height=7cm,focal length=3cm,
lens height=0.5,line shape example] {};
\foreach \anchor/\placement in
{north/above,south/below,lens north/right,lens south/right,center/right,
east focus/above,west focus/above}
\draw[shift=(s.\anchor)] plot[mark=x] coordinates{(0,0)}
node[\placement] {\scriptsize\texttt{(s.\anchor)}};
\end{tikzpicture}
\end{codeexample}

The keys |east focus| and |west focus| are respectively alias to |east focal point| and |west focal point|.


The \texttt{tikz} key \texttt{anchor=} can be used to position lenses with respect to each other.

\begin{codeexample}[width=6cm]
\begin{tikzpicture}[use optics,scale=.5]
  \node[lens,draw focal points={red}]
  (L1) at (0,0) {};
  \node[lens,draw focal points={circle,draw=blue},
  focal length=0.5cm,anchor=west focus]
  (L2) at (L1.east focus) {};
\end{tikzpicture}
\end{codeexample}

\end{shape}

\subsubsection{Slit}

\begin{shape}{slit}
Draws a slit.

\begin{codeexample}[width=6cm]
\begin{tikzpicture}[use optics,scale=.5]
  \node[slit] (S) at (0,0) {};
\end{tikzpicture}  
\end{codeexample}

\begin{key}{/tikz/optics/object height=\meta{length} (initially \pgfkeysvalueof{/tikz/optics/object height})}
    The key |object height| applies.
\end{key}


\begin{key}{/tikz/optics/slit height=\meta{number or length} (initially \pgfkeysvalueof{/tikz/optics/slit height})}
The key |slit height| determines the height of the slit aperture (relative lengths are in units of the total object height).
For instance, \meta{number}|=0.5| produces a slit half as large as the support.
Similarly \meta{length}|=1cm| produces a slit with height |1cm|. 
The value of \meta{number} should be smaller than one, and the value of \meta{length} should be smaller than the value of |object height|. 
The results are unspecified (and probably terrible) when this is not the case.

\begin{codeexample}[width=6cm]
\begin{tikzpicture}[use optics,scale=.5]
  \node[slit, slit height=0.5] (S) at (0,0) {};
  \node[slit, slit height=0.3] (S) at (1cm,0) {};
  \node[slit, slit height=0.1] (S) at (2cm,0) {};
\end{tikzpicture}  
\end{codeexample}
\end{key}

The following figure summarizes the anchors defined by |slit|.

\begin{codeexample}[]
\Huge
\begin{tikzpicture}[use optics]
\node[name=s,slit,object height=8cm,
slit height=0.2,line shape example] {};
\foreach \anchor/\placement in
{north/above,south/below,slit north/right,slit south/right,center/left,
slit center/right}
\draw[shift=(s.\anchor)] plot[mark=x] coordinates{(0,0)}
node[\placement] {\scriptsize\texttt{(s.\anchor)}};
\end{tikzpicture}
\end{codeexample}

\end{shape}

\subsubsection{Double slit}


\begin{shape}{double slit}
Draws a double slit.

\begin{codeexample}[width=5cm]
\begin{tikzpicture}[use optics,scale=.5]
  \node[double slit] (S) at (0,0) {};
\end{tikzpicture}  
\end{codeexample}

\begin{key}{/tikz/optics/object height=\meta{length} (initially \pgfkeysvalueof{/tikz/optics/object height})}
    The key |object height| applies.
\end{key}


\begin{key}{/tikz/optics/slit height=\meta{number or length} (initially \pgfkeysvalueof{/tikz/optics/slit height})}
The key |slit height| determines with height of each slit (relative lengths are in units of the object height).
Each slit will have a height \meta{number or length}. 

\begin{codeexample}[width=5cm]
\begin{tikzpicture}[use optics,scale=.5,optics/slit separation=0.5]
  \node[double slit, slit height=0.075] (S) at (0,0) {};
  \node[double slit, slit height=0.1] (S) at (1cm,0) {};
  \node[double slit, slit height=0.2] (S) at (2cm,0) {};
\end{tikzpicture}  
\end{codeexample}
\end{key}

\begin{key}{/tikz/optics/slit separation=\meta{number or length} (initially \pgfkeysvalueof{/tikz/optics/slit separation})}
The key |slit separation| determines the distance between the two slits (relative lengths are in units of the object height).

\begin{codeexample}[width=5cm]
\begin{tikzpicture}[use optics,scale=.5]
  \node[double slit, slit separation=0.1] (S) at (0,0) {};
  \node[double slit, slit separation=0.2] (S) at (1cm,0) {};
  \node[double slit, slit separation=0.3] (S) at (2cm,0) {};
\end{tikzpicture}  
\end{codeexample}
\end{key}



The following figure summarizes the anchors defined by |double slit|.

\begin{codeexample}[]
\Huge
\begin{tikzpicture}[use optics]
\node[double slit,name=s,object height=8cm, slit height=0.15,
slit separation=0.5, line shape example] {};
\foreach \anchor/\placement in
{north/above,south/below,center/left,
slit 1 north/right,slit 1 south/right,slit 1 center/right,
slit 2 north/right,slit 2 south/right,slit 2 center/right}
\draw[shift=(s.\anchor)] plot[mark=x] coordinates{(0,0)}
node[\placement] {\scriptsize\texttt{(s.\anchor)}};
\end{tikzpicture}
\end{codeexample}

\end{shape}


\subsubsection{Mirror}


\begin{shape}{mirror}
Draws a plane mirror.

\begin{codeexample}[width=6cm]
\begin{tikzpicture}[use optics,scale=.5]
  \node[mirror] (S) at (0,0) {};
\end{tikzpicture}  
\end{codeexample}

\begin{key}{/tikz/optics/object height=\meta{length} (initially \pgfkeysvalueof{/tikz/optics/object height})}
    The key |object height| applies.
\end{key}

\begin{key}{/tikz/optics/mirror decoration separation=\meta{number or length} (initially \pgfkeysvalueof{/tikz/optics/mirror decoration separation})}
Sets the key |/pgf/decoration/segment length| of the |border| decoration used to draw the mirror (see the tikz manual for details).
The value us |/pgf/decoration/segment length| used when the argument is dimensionless is obtained by multiplying \meta{number} by the height of the mirror.
\end{key}

\begin{key}{/tikz/optics/mirror decoration amplitude=\meta{number or length} (initially \pgfkeysvalueof{/tikz/optics/mirror decoration amplitude})}
Sets the key |/pgf/decoration/amplitude| of the |border| decoration used to draw the mirror (see the tikz manual for details).
The value of |/pgf/decoration/amplitude| used when the argument is dimensionless is obtained by multiplying \meta{number} by the height of the mirror.
\end{key}

The following figure summarizes the anchors defined by |mirror|.

\begin{codeexample}[]
\Huge
\begin{tikzpicture}[use optics]
\node[mirror,name=s,object height=8cm,line shape example,
mirror decoration separation=0.141, mirror decoration amplitude=0.2] {};
\foreach \anchor/\placement in
{north/above,south/below,center/right}
\draw[shift=(s.\anchor)] plot[mark=x] coordinates{(0,0)}
node[\placement] {\scriptsize\texttt{(s.\anchor)}};
\end{tikzpicture}
\end{codeexample}

\end{shape}

\subsubsection{Spherical mirror}

\begin{shape}{spherical mirror}
Draws a spherical mirror (convex or concave)

\begin{warning}
  This part is not finished and can change without warning.
\end{warning}

\begin{codeexample}[width=6cm]
\begin{tikzpicture}[use optics,scale=.5]
  \node[spherical mirror] (M) at (0,0) {};
\end{tikzpicture}  
\end{codeexample}

\begin{key}{/tikz/optics/object height=\meta{length} (initially \pgfkeysvalueof{/tikz/optics/object height})}
    The key |object height| applies.
\end{key}

\begin{key}{/tikz/optics/spherical mirror angle=\meta{angle} (initially \pgfkeysvalueof{/tikz/optics/spherical mirror angle})}
   The key |spherical mirror angle| determines the aperture angle of the spherical mirror (an arc with height determined by |object height| and with angular aperture \meta{angle} is drawn).
    Do not use $\meta{angle}=0$! Instead, use a plane |mirror|.
\begin{codeexample}[width=5cm]
\begin{tikzpicture}[use optics]
  \node[spherical mirror, spherical mirror angle=60] at (0,0) {};
  \node[spherical mirror, spherical mirror angle=120] at (2cm,0) {};
  \node[spherical mirror, spherical mirror angle=180] at (4cm,0) {};
\end{tikzpicture}
\end{codeexample}

\begin{warning}
  The function |from_radius| is experimental.
\end{warning}

It can be useful to specify the radius of curvature of the mirror instead of the aperture angle. 
To do so, a function |from_radius(R)| computes the aperture radius corresponding to the radius |R|, with fixed |/tikz/optics/object height|.
It is indeed not possible to have a height greater than twice the radius.
\begin{codeexample}[width=5cm]
\begin{tikzpicture}[use optics]
  \node[spherical mirror, object height=2cm, 
  spherical mirror angle=from_radius(3cm),
  draw mirror focus, draw mirror center={red}] (M) {};
\end{tikzpicture}
\end{codeexample}
\end{key}

\begin{key}{/tikz/optics/spherical mirror type}
    The key |spherical mirror type| determines the type of mirror: |spherical mirror type=concave| produces a concave mirror (default), and |spherical mirror type=convex| produces a convex mirror
    The styles |convex mirror| and |concave mirror| are shortcuts for |spherical mirror| along with this key.

\begin{codeexample}[width=5cm]
\begin{tikzpicture}[use optics]
  \node[convex mirror, label={[label distance=0.25cm]south:convex mirror}] at (0cm,0) {};
  \node[concave mirror, label={[label distance=0.25cm]south:concave mirror}] at (4cm,0) {};
\end{tikzpicture}
\end{codeexample}
\end{key}

\begin{stylekey}{/tikz/optics/concave mirror}
    The style |concave mirror| is equivalent to |spherical mirror, spherical mirror type=concave| and draws a concave spherical mirror. 
    See |spherical mirror type| for an example.
\end{stylekey}

\begin{stylekey}{/tikz/optics/convex mirror}
    The style |convex mirror| is equivalent to |spherical mirror, spherical mirror type=convex| and draws a convex spherical mirror.
    See |spherical mirror type| for an example.
\end{stylekey}


\begin{key}{/tikz/optics/spherical mirror orientation=\meta{type} (initially |ltr|)}
    The key |spherical mirror orientation| determines the orientation of the mirror (namely, the assumed direction of propagation of light).
    The allowed values are |ltr| (\enquote{left to right}) and |rtl| (\enquote{right to left}).

\begin{codeexample}[width=5cm]
\begin{tikzpicture}[use optics]
  \node[spherical mirror, spherical mirror orientation=ltr] at (0cm,0) {};
  \node[spherical mirror, spherical mirror orientation=rtl] at (2cm,0) {};
\end{tikzpicture}
\end{codeexample}
\end{key}

The crosshatch mirror decoration is controlled with the same keys as for |mirror| : |mirror decoration separation| et |mirror decoration amplitude|.

The following figure summarizes some of the anchors defined by |spherical mirror|.

\begin{codeexample}[]
\Huge
\begin{tikzpicture}[use optics]
\node[spherical mirror,name=s,object height=8cm,line shape example,
mirror decoration separation=0.141, mirror decoration amplitude=0.2] {};
\foreach \anchor/\placement in
{north/above,south/below,center/below right,
east/below right,
west/below,
north east/above right,
north west/above left,
south east/below right,
south west/below left,
mirror center/left,
arc start/above,
arc end/below,
arc center/right,
27/right,
focus/above}
\draw[shift=(s.\anchor)] plot[mark=x] coordinates{(0,0)}
node[\placement] {\scriptsize\texttt{(s.\anchor)}};
\end{tikzpicture}
\end{codeexample}

With spherical mirrors, tikz \enquote{border anchor} are particularly convenient. 
Numerical anchors (such as |node.27|) allow to access the boundary of the |shape| in the corresponding angular direction (here at \ang{27}). Recall that in tikz, angles are expressed in degrees and increase counterclockwise from the $x$-axis.
In the following example, the underlying circle has been drawn for clarity.

\begin{codeexample}[width=5cm]
\begin{tikzpicture}[use optics,scale=.5]]
\coordinate (O) at (0,0);
\node[circle,draw, inner sep=0, outer sep=0,minimum height=2cm, densely dashed, gray!60] (C) at (O) {};
\node[spherical mirror,draw,object height=2cm,anchor=mirror center,
    spherical mirror angle=180] (M) at (O) {};
\draw[blue] (O) -- (M.0);
\draw[violet] (O) -- (M.45);
\draw[red] (O) -- (M.90);
\draw[orange] (O) -- (M.135);
\end{tikzpicture}
\end{codeexample}

\begin{warning}
  The positions of the anchors with angles greater than $\pm$|spherical mirror angle|$/2$ is not specified.
  Currently, they are treated as if the mirror was a full circle, but this might change in the future.
\end{warning}

\end{shape}


\begin{stylekey}{/tikz/optics/draw mirror center=\meta{style}}
    The style |draw mirror center| marks the center of the mirror (with the style \meta{style} if specified).

\begin{codeexample}[width=5cm]
\begin{tikzpicture}[use optics]
\node[concave mirror,draw mirror center, draw mirror focus={red}] {};
\end{tikzpicture}
\end{codeexample}
\end{stylekey}

\begin{stylekey}{/tikz/optics/draw mirror focus=\meta{style}}
    The style |draw mirror focus| marks the focal point of the mirror (with the style \meta{style} if specified).
    See |draw mirror center| for an example.
\end{stylekey}

\subsubsection{Polarizer}

\begin{shape}{polarizer}
Draws a polarizer

\begin{codeexample}[width=5cm]
\begin{tikzpicture}[use optics,scale=.5]
  \node[polarizer] (S) at (0,0) {};
\end{tikzpicture}  
\end{codeexample}


\begin{key}{/tikz/optics/object height=\meta{length} (initially \pgfkeysvalueof{/tikz/optics/object height})}
    The key |object height| applies.
\end{key}


\begin{key}{/tikz/optics/object aspect ratio=\meta{number} (initially \pgfkeysvalueof{/tikz/optics/object aspect ratio})}
    The key |object aspect ratio| determines the aspect ratio of the polarizer.
    When \meta{number} is |1|, the with of the polarizer is equal to its height.
    When \meta{number}|=1/2|, the with of the polarizer is equal to half its height.

    \begin{codeexample}[width=5cm]
\begin{tikzpicture}[use optics,scale=.5]
  \node[polarizer, object aspect ratio=0.2] (L1) at (0,0) {};
  \node[polarizer, object aspect ratio=0.5] (L2) at (4cm,0) {};
\end{tikzpicture}
    \end{codeexample}
\end{key}



The following figure summarizes the anchors defined by |polarizer|.

\begin{codeexample}[]
\Huge
\begin{tikzpicture}[use optics]
\node[polarizer,name=s,object height=8cm,object aspect ratio=0.4,shape example] {};
\foreach \anchor/\placement in
{north/above,south/below,east/right,west/left,center/below,
north east/right,north west/left,south east/right,south west/left}
\draw[shift=(s.\anchor)] plot[mark=x] coordinates{(0,0)}
node[\placement] {\scriptsize\texttt{(s.\anchor)}};
\end{tikzpicture}
\end{codeexample}

\end{shape}


\subsubsection{Beam splitter}

\begin{stylekey}{/tikz/optics/beam splitter}
Draws a beam splitter. Keys and anchors are the same as |polarizer|.

\begin{codeexample}[width=6cm]
\begin{tikzpicture}[use optics,scale=.5]
  \node[beam splitter] at (0,0) {};
\end{tikzpicture}  
\end{codeexample}

\end{stylekey}


\subsubsection{Double Amici prism}

\begin{shape}{double amici prism}
Draws a double Amici prism.

\begin{codeexample}[width=6cm]
\begin{tikzpicture}[use optics,scale=.5]
  \node[double amici prism] (PVD) at (0,0) {};
\end{tikzpicture}  
\end{codeexample}


\begin{key}{/tikz/optics/prism height=\meta{length} (initially \pgfkeysvalueof{/tikz/optics/prism height})}
    The key |prism height| determines the height of the three identical prisms that form the double Amici prism (the length of a side is therefore $2/\sqrt{3}$ times this height).

    \begin{codeexample}[width=5cm]
\begin{tikzpicture}[use optics,scale=.5]
  \node[double amici prism, prism height=1cm] (PVD1) at (0,0) {};
  \node[double amici prism, prism height=0.5cm] (PVD2) at (4cm,0) {};
\end{tikzpicture}
    \end{codeexample}
\end{key}


\begin{key}{/tikz/optics/prism apex angle=\meta{angle} (initially \pgfkeysvalueof{/tikz/optics/prism apex angle})}
    The key |prism apex angle| determines the apex angle of the three identical prisms.
    \meta{angle} is expressed in degrees. When \meta{angle} is |60|, the prisms are equilateral triangles.

    \begin{codeexample}[width=5cm]
\begin{tikzpicture}[use optics,scale=.5]
  \node[double amici prism, prism apex angle=40] (PVD1) at (0,0) {};
  \node[double amici prism, prism apex angle=60] (PVD2) at (4cm,0) {};
  \node[double amici prism, prism apex angle=80] (PVD3) at (10cm,0) {};
\end{tikzpicture}
    \end{codeexample}
\end{key}



The following figure summarizes the anchors defined by |double amici prism|.

\begin{codeexample}[]
\Huge
\begin{tikzpicture}[use optics]
\node[double amici prism,name=s,prism height=4.5cm,shape example] {};
\foreach \anchor/\placement in
{north/above,south/below,east/right,west/left,center/below,
north east/right,north west/left,south east/right,south west/left}
\draw[shift=(s.\anchor)] plot[mark=x] coordinates{(0,0)}
node[\placement] {\scriptsize\texttt{(s.\anchor)}};
\end{tikzpicture}
\end{codeexample}

\end{shape}


\subsubsection{Thin generic element}

\begin{shape}{thin optics element}
Draws a generic optical element (used to build more concrete ones).
This |shape| can be useful to draw an element not present in this library, by adding some relevant style.

\begin{codeexample}[width=6cm]
\begin{tikzpicture}[use optics,scale=.5]
  \node[thin optics element] (S) at (0,0) {};
\end{tikzpicture}  
\end{codeexample}

\begin{key}{/tikz/optics/object height=\meta{length} (initially \pgfkeysvalueof{/tikz/optics/object height})}
    The key |object height| applies.
\end{key}

The following figure summarizes the anchors defined by |thin optics element|.

\begin{codeexample}[]
\Huge
\begin{tikzpicture}[use optics]
\node[thin optics element,name=s,object height=8cm,line shape example] {};
\foreach \anchor/\placement in
{north/above,south/below,center/right}
\draw[shift=(s.\anchor)] plot[mark=x] coordinates{(0,0)}
node[\placement] {\scriptsize\texttt{(s.\anchor)}};
\end{tikzpicture}
\end{codeexample}

\end{shape}

\subsubsection{Thick generic element}

\begin{shape}{thick optics element}
Draws a thick generic optical element.

\begin{codeexample}[width=6cm]
\begin{tikzpicture}[use optics,scale=.5]
  \node[thick optics element] at (0,0) {};
\end{tikzpicture}  
\end{codeexample}


\begin{key}{/tikz/optics/object height=\meta{length} (initially \pgfkeysvalueof{/tikz/optics/object height})}
    The key |object height| applies.
\end{key}


\begin{key}{/tikz/optics/object aspect ratio=\meta{number or length} (initially 0.05)}
    The key |object aspect ratio| determines the aspect ratio of the object.
\end{key}




The following figure summarizes the anchors defined by |thick optics element|.

\begin{codeexample}[]
\Huge
\begin{tikzpicture}[use optics]
\node[thick optics element,name=s,object height=8cm,shape example,object aspect ratio=0.2] {};
\foreach \anchor/\placement in
{north/above,south/below,east/right,west/left,center/below,
north east/right,north west/left,south east/right,south west/left}
\draw[shift=(s.\anchor)] plot[mark=x] coordinates{(0,0)}
node[\placement] {\scriptsize\texttt{(s.\anchor)}};
\end{tikzpicture}
\end{codeexample}

\end{shape}

\subsubsection{Heat filter}

\begin{stylekey}{/tikz/optics/heat filter}
The style |heat filter| draws a heat filter.
The options and anchors of |thick optics element| apply.

\begin{codeexample}[width=6cm]
\begin{tikzpicture}[use optics,scale=.5]
  \node[heat filter] (S) at (0,0) {};
\end{tikzpicture}  
\end{codeexample}
\end{stylekey}


\subsubsection{Screen}


\begin{stylekey}{/tikz/optics/screen}
The style |screen| draws a screen.
The options and anchors of |thin optics element| apply.

\begin{codeexample}[width=6cm]
\begin{tikzpicture}[use optics,scale=.5]
  \node[screen] (S) at (0,0) {};
\end{tikzpicture}  
\end{codeexample}
\end{stylekey}

\subsubsection{Diffraction grating}

\begin{stylekey}{/tikz/optics/diffraction grating}
The style |diffraction grating| draws a diffraction grating.
The options and anchors of |thin optics element| apply.

\begin{codeexample}[width=6cm]
\begin{tikzpicture}[use optics,scale=.5]
  \node[diffraction grating] (S) at (0,0) {};
\end{tikzpicture}  
\end{codeexample}
\end{stylekey}

\subsubsection{Grid}

\begin{stylekey}{/tikz/optics/grid}
The style |grid| draws a grid.
The options and anchors of |thin optics element| apply.

\begin{codeexample}[width=6cm]
\begin{tikzpicture}[use optics,scale=.5]
  \node[grid] (S) at (0,0) {};
\end{tikzpicture}  
\end{codeexample}
\end{stylekey}

\subsubsection{Semi-transparent mirror}

\begin{stylekey}{/tikz/optics/semi-transparent mirror}
The style |semi-transparent mirror| draws a semi-transparent mirror.
The options and anchors of |thin optics element| apply.

\begin{codeexample}[width=6cm]
\begin{tikzpicture}[use optics,scale=.5]
  \node[semi-transparent mirror] (S) at (0,0) {};
\end{tikzpicture}  
\end{codeexample}
\end{stylekey}

\subsubsection{Diaphragm}

\begin{stylekey}{/tikz/optics/diaphragm}
The style |diaphragm| draws a diaphragm (a |slit| element with a large slit). 
The options and anchors of |slit| apply.

\begin{codeexample}[width=6cm]
\begin{tikzpicture}[use optics,scale=.5]
  \node[diaphragm] (S) at (0,0) {};
\end{tikzpicture}  
\end{codeexample}
\end{stylekey}


\subsection{Light sources and receptors}

\subsubsection{Generic optical input/output}

\begin{shape}{generic optics io}
Draws a generic input/output system.
We will use this to build more concrete elements.

\begin{codeexample}[width=6cm]
\begin{tikzpicture}[use optics,scale=.5]
  \node[generic optics io] (S) at (0,0) {};
\end{tikzpicture}  
\end{codeexample}


\begin{key}{/tikz/optics/io body height=\meta{length} (initially \pgfkeysvalueof{/tikz/optics/io body height})}
    The key |io body height| determines the size of the light source or receptor. 
    More precisely, \meta{length} determines the height of the element body, while the other lengths are specified relatively to this height.
    By changing \meta{length}, the light source/receptor is then resized without deformation.

    \begin{codeexample}[]
\begin{tikzpicture}[use optics,scale=.5]
  \node[generic optics io, io body height=1cm] (L1) at (0,0) {};
  \node[generic optics io, io body height=2cm] (L2) at (10cm,0) {};
\end{tikzpicture}
    \end{codeexample}
\end{key}

\begin{key}{/tikz/optics/io body aspect ratio=\meta{number or length} (initially \pgfkeysvalueof{/tikz/optics/io body aspect ratio})}
    The key |io body aspect ratio| determines the aspect ratio of the light source/receptor.
    When \meta{number} is |1|, the width of the light source/receptor is equal to its height.
    When \meta{number}|=1/2|, the width of the light source/receptor is equal to half its height.

    \begin{codeexample}[]
\begin{tikzpicture}[use optics,scale=.5]
  \node[generic optics io, io body aspect ratio=1] (L1) at (0,0) {};
  \node[generic optics io, io body aspect ratio=2] (L2) at (8cm,0) {};
\end{tikzpicture}
    \end{codeexample}
\end{key}

\begin{stylekey}{/tikz/optics/io body width=\meta{number or length}}
Alias for |io body aspect ratio|.
\end{stylekey}


\begin{key}{/tikz/optics/io aperture width=\meta{number or length} (initially \pgfkeysvalueof{/tikz/optics/io aperture width})}
    The key |io aperture width| determines the height of the aperture of the element (e.g. a condenser, a collimating lens, etc.), relative to the body height. 
    When \meta{number}|=0|, the aperture is not drawn.

    \begin{codeexample}[]
\begin{tikzpicture}[use optics,scale=.5]
  \node[generic optics io, io aperture width=0] at (0,0) {};
  \node[generic optics io, io aperture width=0.1] at (8cm,0) {};
  \node[generic optics io, io aperture width=0.5] at (16cm,0) {};
\end{tikzpicture}
    \end{codeexample}
\end{key}




\begin{key}{/tikz/optics/io aperture height=\meta{number or length} (initially \pgfkeysvalueof{/tikz/optics/io aperture height})}
    The key |io aperture width| determines the height of the aperture relative to the body height.

    \begin{codeexample}[]
\begin{tikzpicture}[use optics,scale=.5]
  \node[generic optics io, io aperture height=0.5] at (0,0) {};
  \node[generic optics io, io aperture height=0.8] at (8cm,0) {};
\end{tikzpicture}
    \end{codeexample}
\end{key}


\begin{key}{/tikz/optics/io aperture shift=\meta{number or length} (initially \pgfkeysvalueof{/tikz/optics/io aperture shift})}
    The key |io aperture shift| determines how much the aperture should be shifted from the center of the body (again, the length is relative to the height of the body).

    \begin{codeexample}[]
\begin{tikzpicture}[use optics,scale=.5,
optics,io body height=2cm,io body aspect ratio=0.5,io aperture height=0.3, io aperture width=0.1]
	\node[generic optics io,io aperture shift=-0.25] at (-5cm,0) {};
  \node[generic optics io,io aperture shift=0] at (0,0) {};
  \node[generic optics io,io aperture shift=0.25] at (5cm,0) {};
\end{tikzpicture}
    \end{codeexample}
\end{key}


Text can be drawn inside the body of a lamp or receptor through the |node| text.

\begin{codeexample}[]
\begin{tikzpicture}[use optics,scale=.5]
  \node[generic optics io] at (-5cm,0) {Q.I.};
\end{tikzpicture}
\end{codeexample}

\begin{key}{/tikz/optics/io orientation=\meta{type} (initially |ltr|)}
    The key |io orientation| determines the direction in which the object should be drawn (the aperture is on the right [at the |east| anchor] when it is set to |ltr|, and it is on the left [at the |west| anchor] when it is set to |rtl|). 
    The Les noms correspondent à  et .
    The allowed values of this key are |ltr| (for \enquote{left to right}) and |rtl| (for \enquote{right to left}).
    Unlike |rotate|, the key |io orientation| modifies the anchors of the shape.

    \begin{codeexample}[]
\begin{tikzpicture}[use optics,scale=.5]
  \node[generic optics io] at (0,0) {};
  \node[generic optics io,io orientation=ltr] at (5cm,0) {};
  \node[generic optics io,io orientation=rtl] at (10cm,0) {};
\end{tikzpicture}
    \end{codeexample}
\end{key}


The following figure summarizes the anchors defined by |generic optics io|.

\begin{codeexample}[]
\Huge
\begin{tikzpicture}[use optics]
\node[generic optics io,name=s,io body height=4cm,io aperture width=0.7,io body aspect ratio=1.8,shape example] {};
\foreach \anchor/\placement in
{body north/above,body south/below,body east/above right,body west/right,body center/below,
body north east/above,body north west/above,body south east/below,body south west/below,
aperture north/above,aperture south/below,aperture east/right,aperture west/left,aperture center/below,
aperture north east/right,aperture north west/left,aperture south east/right,aperture south west/left}
\draw[shift=(s.\anchor)] plot[mark=x] coordinates{(0,0)}
node[\placement] {\scriptsize\texttt{(s.\anchor)}};
\end{tikzpicture}
\end{codeexample}

The anchors |body east| and |aperture west| correspond to the same point. Both are defined for consistency.
The anchor |east| is defined either as |body east| or |aperture east| depending on the value of |io orientation| so that |east| is the easternmost (rightmost) anchor. The same thing is done for the anchor |west|.

The following figure summarizes the anchors defined by |generic optics io, io orientation=rtl|.

\begin{codeexample}[]
\Huge
\begin{tikzpicture}[use optics]
\node[generic optics io,name=s,io body height=4cm,io aperture width=0.7,io body aspect ratio=1.8,
io orientation=rtl,shape example] {};
\foreach \anchor/\placement in
{body north/above,body south/below,body east/above right,body west/above right,body center/below,
body north east/above,body north west/above,body south east/below,body south west/below,
aperture north/above,aperture south/below,aperture east/right,aperture west/left,aperture center/below,
aperture north east/right,aperture north west/left,aperture south east/right,aperture south west/left}
\draw[shift=(s.\anchor)] plot[mark=x] coordinates{(0,0)}
node[\placement] {\scriptsize\texttt{(s.\anchor)}};
\end{tikzpicture}
\end{codeexample}

\end{shape}


\subsubsection{Sensor line}


\begin{shape}{sensor line}
Draws a sensor line (such as a CCD or CMOS sensor).

\begin{codeexample}[width=6cm]
\begin{tikzpicture}[use optics,scale=.5]
  \node[sensor line] (S) at (0,0) {};
\end{tikzpicture}  
\end{codeexample}

\begin{key}{/tikz/optics/sensor line height=\meta{length} (initially \pgfkeysvalueof{/tikz/optics/sensor line height})}
    The key |sensor line height| determines the size of the sensor line.
    More precisely, \meta{length} determines the height of the sensor line body and the other lengths can be specified relative to this one.

    \begin{codeexample}[]
\begin{tikzpicture}[use optics,scale=.5]
  \node[sensor line, sensor line height=1cm] (L1) at (0,0) {};
  \node[sensor line, sensor line height=2cm] (L2) at (4cm,0) {};
\end{tikzpicture}
    \end{codeexample}
\end{key}

\begin{key}{/tikz/optics/sensor line aspect ratio=\meta{number} (initially \pgfkeysvalueof{/tikz/optics/sensor line aspect ratio})}
    The key |sensor line aspect ratio| determines the aspect ratio of the sensor line.

    \begin{codeexample}[]
\begin{tikzpicture}[use optics,scale=.5]
  \node[sensor line, sensor line aspect ratio=0.2] (L1) at (0,0) {};
  \node[sensor line, sensor line aspect ratio=0.4] (L2) at (4cm,0) {};
\end{tikzpicture}
    \end{codeexample}
\end{key}


\begin{key}{/tikz/optics/sensor line pixel number=\meta{number} (initially \pgfkeysvalueof{/tikz/optics/sensor line pixel number})}
    The key |sensor line pixel number| determines the number of pixels \meta{number} of the sensor line.

    \begin{codeexample}[]
\begin{tikzpicture}[use optics,scale=.5]
  \node[sensor line, sensor line pixel number=2] at (0,0) {};
  \node[sensor line, sensor line pixel number=4] at (4cm,0) {};
  \node[sensor line, sensor line pixel number=10] at (8cm,0) {};
\end{tikzpicture}
    \end{codeexample}
\end{key}


\begin{key}{/tikz/optics/sensor line pixel width=\meta{number or length} (initially \pgfkeysvalueof{/tikz/optics/sensor line pixel width})}
    The key |sensor line pixel width| determines with width of each pixel (relative to the sensor width if it is a \meta{number}).

    \begin{codeexample}[]
\begin{tikzpicture}[use optics,scale=.5]
  \node[sensor line, sensor line pixel width=0.3] at (0,0) {};
  \node[sensor line, sensor line pixel width=0.8] at (4cm,0) {};
\end{tikzpicture}
    \end{codeexample}
\end{key}


\begin{key}{/tikz/optics/sensor line inner ysep=\meta{number or length} (initially \pgfkeysvalueof{/tikz/optics/sensor line inner ysep})}
    The key |sensor line inner ysep| determines the separation between the pixels and the top and bottom of the sensor line.
    When \meta{number or length} is a relative lengths, it is relative to the height of thr sensor line.
    The height of each pixel is such that there are |sensor line pixel number| of them, with this separation taken into account.

    \begin{codeexample}[]
\begin{tikzpicture}[use optics,scale=.5]
  \node[sensor line,sensor line inner ysep=0] at (0,0) {};
  \node[sensor line,sensor line inner ysep=0.1] at (4cm,0) {};
  \node[sensor line,sensor line inner ysep=0.2] at (8cm,0) {};
\end{tikzpicture}
    \end{codeexample}
\end{key}


The anchors |north|, |south|, |east|, |west|, |center|, |north east|, |north west|, |south east|, |south west|, as well as
|pixel <i> <anchor>| where |<anchor>| is |north|, |south|, etc. and where |<i>| is the number of the pixel
(going from |1| to |sensor line pixel number|) are defined. For instance, one can use |pixel 3 west|.

The following figure summarizes the anchors defined by |sensor line| globally as well as the anchors for the 4th pixel (in rouge).

\begin{codeexample}[]
\Huge
\begin{tikzpicture}[use optics]
\node[sensor line,name=s,sensor line height=10cm,sensor line aspect ratio=0.8,
sensor line inner ysep=0.1,sensor line pixel number=5, shape example] {};
\foreach \anchor/\placement in
{north/above,south/below,east/right,west/left,center/below,
north east/right,north west/left,south east/right,south west/left}

\draw[shift=(s.\anchor)] plot[mark=x] coordinates{(0,0)}
node[\placement] {\scriptsize\texttt{(s.\anchor)}};

\foreach \anchor/\placement in
{pixel 4 north/above,pixel 4 south/below,pixel 4 east/right,pixel 4 west/left,pixel 4 center/below,
pixel 4 north east/right,pixel 4 north west/left,pixel 4 south east/right,pixel 4 south west/left}
% manque pixel <i> <subanchor>
\draw[red,shift=(s.\anchor)] plot[mark=x,red] coordinates{(0,0)}
node[\placement] {\scriptsize\texttt{(s.\anchor)}};

\end{tikzpicture}
\end{codeexample}

\end{shape}


Note: when |east| and |west| are not explicitly defined, they are aliases for |center|. This allows |right=of...| and similar keys to work as expected.



\subsubsection{Generic sensor}

\begin{stylekey}{/tikz/optics/generic sensor}
The style |generic sensor| draws a sensor. 
The options of |generic optics io| apply.

\begin{codeexample}[width=6cm]
\begin{tikzpicture}[use optics,scale=.5]
  \node[generic sensor] (S) at (0,0) {};
\end{tikzpicture}  
\end{codeexample}
\end{stylekey}

\subsubsection{Generic light source}

\begin{stylekey}{/tikz/optics/generic lamp}
The style |generic lamp| draws a light source.
The options of |generic optics io| apply.

\begin{codeexample}[width=6cm]
\begin{tikzpicture}[use optics,scale=.5]
  \node[generic lamp] (S) at (0,0) {};
\end{tikzpicture}  
\end{codeexample}
\end{stylekey}

\subsubsection{Halogen lamp}

\begin{stylekey}{/tikz/optics/halogen lamp}
The style |halogen lamp| draws a halogen lamp. 
The options of |generic optics io| apply.

\begin{codeexample}[width=6cm]
\begin{tikzpicture}[use optics,scale=.5]
  \node[halogen lamp] (S) at (0,0) {};
  \node[halogen lamp] (S) at (5cm,0) {QI};
\end{tikzpicture}  
\end{codeexample}
\end{stylekey}

\subsubsection{Spectral lamp}

\begin{stylekey}{/tikz/optics/spectral lamp}
The style |spectral lamp| draws a spectral lamp.
The options of |generic optics io| apply.
In addition, a style is automatically applied to allow multiline text.

\begin{codeexample}[width=6cm]
\begin{tikzpicture}[use optics,scale=.5]
  \node[spectral lamp] (S) at (0,0) {};
  \node[spectral lamp] (S) at (5cm,0) {\ce{Hg} \\ LP};
\end{tikzpicture}  
\end{codeexample}
\end{stylekey}

\subsubsection{Laser}

\begin{stylekey}{/tikz/optics/laser}
The style |laser| draws a laser.
The options of |generic optics io| apply.

\begin{codeexample}[width=6cm]
\begin{tikzpicture}[use optics,scale=.5]
  \node[laser] (S) at (0,0) {};
  \node[laser] (S) at (5cm,0) {\ce{HeNe}};
\end{tikzpicture}  
\end{codeexample}
\end{stylekey}

\begin{stylekey}{/tikz/optics/laser'}
The style |laser'| draws a laser without output aperture (i.e. with |io aperture width=0pt|). 
The options of |generic optics io| apply.

\begin{codeexample}[width=6cm]
\begin{tikzpicture}[use optics,scale=.5]
  \node[laser'] (S) at (0,0) {\ce{Nd}:YAG};
  \draw[green] (S.aperture east) -- +(2cm,0);
\end{tikzpicture}  
\end{codeexample}
\end{stylekey}

\subsection{Utilities}


\subsubsection{Drawing rays}

\begin{update}
  2014-12-07 : this part was substantially modified
\end{update}

\begin{warning}
  This part is not fully stable. Shortcuts such as |->-|, should in principle not change.
\end{warning}

In optics, it is often useful to mark the rays with arrows to distinguish them. The following styles use an arrow tip designed such that the arrow will be centered at the middle of the path (or at any specified position), instead of being a bit offset (in contrast with what would happen e.g. with |put arrow| and |\arrow{>>}|).

\begin{stylekey}{/tikz/optics/->-}
  This style adds an arrow in the middle of the path.
  \begin{codeexample}[width=6cm]
\begin{tikzpicture}[use optics]
  \draw[->-] (0,0) -- (1.5cm,1cm);
\end{tikzpicture}  
  \end{codeexample}
\end{stylekey}

\begin{stylekey}{/tikz/optics/-<-}
  This style adds an arrow in the middle of the path, with a direction opposite of |->-|.
  \begin{codeexample}[width=6cm]
\begin{tikzpicture}[use optics]
  \draw[-<-] (0,0) -- (1.5cm,1cm);
\end{tikzpicture}  
  \end{codeexample}
\end{stylekey}

\begin{stylekey}{/tikz/optics/->>-}
  This style adds a double arrow in the middle of the path.
  \begin{codeexample}[width=6cm]
\begin{tikzpicture}[use optics]
  \draw[->>-] (0,0) -- (1.5cm,1cm);
\end{tikzpicture}  
  \end{codeexample}
\end{stylekey}

\begin{stylekey}{/tikz/optics/-<<-}
  This style adds a double arrow in the middle of the path, with a direction opposite of |->>-|.
  \begin{codeexample}[width=6cm]
\begin{tikzpicture}[use optics]
  \draw[-<<-] (0,0) -- (1.5cm,1.1cm);
\end{tikzpicture}  
  \end{codeexample}
\end{stylekey}

\begin{stylekey}{/tikz/optics/->n-=\{n=\meta{num}, \meta{specs}\}}
  This style adds \meta{num} arrows in the middle of the path.
  The specifications \meta{specs} are applied to the arrows following |put arrow| syntax.
  \begin{codeexample}[width=6cm]
\begin{tikzpicture}[use optics]
  \draw[->n-={n=4}] (0,0) -- (1.5cm,1cm);
\end{tikzpicture}  
  \end{codeexample}
\end{stylekey}

\begin{stylekey}{/tikz/optics/-<n-={n=\meta{num}, \meta{specs}}}
  This style adds \meta{num} arrows in the middle of the path, with a direction opposite of |->n-|.
  \begin{codeexample}[width=6cm]
\begin{tikzpicture}[use optics]
  \draw[-<n-={n=6}] (0,0) -- (1.5cm,1cm);
\end{tikzpicture}  
  \end{codeexample}
\end{stylekey}

Shortcuts from |->-| to |->>>>-| (similar for |-<-|) are available. The styles |->-|, |-<-|, etc. use |put arrow| with the arrow |multiple ray arrow|. They can be configured with the syntax of |put arrow|, for instance

\begin{codeexample}[width=4cm]
\begin{tikzpicture}[use optics]
  \draw[->>-={at=0.25}, ->-={at=0.75}] (0,0) -- (1.5cm,1cm) -- (3cm, 0);
\end{tikzpicture}
\end{codeexample}

This also works with |->n-| and |-<n-|.

\begin{codeexample}[width=4cm]
\begin{tikzpicture}[use optics]
  \draw[->n-={n=5, at=0.2, style=red}] (0,0) -- (1.5cm,1cm);
\end{tikzpicture}
\end{codeexample}

\subsubsection{Put things on paths}

\begin{update}
  2014-12-07 : this part was substantially modified
\end{update}

\begin{warning}
  This part is not stable yet. Everything can entirely change without any warning.
\end{warning}

\begin{key}{/tikz/put arrow}
The key |put arrow| allows to easily add an arrow on a path.

\begin{codeexample}[width=6cm]
\begin{tikzpicture}[use optics]
  \draw[put arrow] (0,0) -- (1.5cm,1cm);
\end{tikzpicture}  
\end{codeexample}

By default, the arrow is |\arrow{>}| and it is added in the middle of the path.

\begin{key}{/tikz/put arrow/pos=\meta{pos} (initially \pgfkeysvalueof{/tikz/put arrow/pos})}
  Sets the position of the arrow on the path to \meta{pos}  (for instance \meta{pos}|=0.5| means that that arrow is in the middle of the path).
\end{key}

\begin{key}{/tikz/put arrow/at}
  Alias for |/tikz/put arrow/pos|.
\end{key}

\begin{key}{/tikz/put arrow/arrow=\meta{arrow specification}}
  Uses the arrow defined by \meta{arrow specification} (for instance |stealth| of |latex|).
\end{key}

\begin{key}{/tikz/put arrow/arrow'=\meta{arrow specification}}
  Uses the arrow defined by \meta{arrow specification} (for instance |stealth| or |latex|), but reversed.
\end{key}

\begin{codeexample}[width=6cm]
\begin{tikzpicture}[use optics]
  \draw[put arrow={arrow'=stealth}] (0,0) -- (1cm,1cm);
  \draw[put arrow={at=0.2}] (2cm,0) -- (3cm,1cm);
\end{tikzpicture}  
\end{codeexample}

\begin{key}{/tikz/put arrow/style=\meta{style}}
  The |\meta{style}| is passed to |\arrow[|\meta{style}|]| to draw the arrows.
\end{key}

\begin{codeexample}[width=5cm]
\begin{tikzpicture}[use optics]
  \draw[red,put arrow={arrow=latex}]
  (0,0) -- (1cm,1cm);
  \draw[red,put arrow={arrow=latex,style={blue}}]
  (1cm,0) -- (2cm,1cm);
\end{tikzpicture}
\end{codeexample}

\begin{stylekey}{/tikz/put arrow/every arrow}
  This style is passed to all arrows drawn by |put arrow| (in the scope of the style).
  Use |every arrow/.style={|\meta{style}|}| (or |append style|, etc.).
\end{stylekey}

\begin{codeexample}[width=3cm]
\begin{tikzpicture}[use optics]
  \draw[red,put arrow={arrow=latex}]
  (0,0) -- (1cm,1cm);
  \draw[red, put arrow/every arrow/.style={blue},
    put arrow={at=0.2}, put arrow={at=0.5}, put arrow={at=0.8}]
  (1cm,0) -- (2cm,1cm);
  \draw[red, >=latex, put arrow/every arrow/.style={blue}, 
    put arrow={at=0.2}, put arrow={at=0.5}, put arrow={at=0.8}]
  (2cm,0) -- (3cm,1cm);
\end{tikzpicture}
\end{codeexample}


The arrows that can be used are discussed in the tikz manual.

For a finer control of what happens, use the tikz library |markings| instead. 

Specific styles |->-|, |-<-|, |->>-| et |-<<-| allow to label light rays.
\end{key}

\subsubsection{Add coordinates on paths}

\begin{key}{/tikz/put coordinate=\meta{coordinate} at \meta{position}}

The style |put coordinate| creates a coordinate with name \meta{coordinate}
at the relative position \meta{position} on the path to which the style is applied.

\begin{codeexample}[width=6cm]
\begin{tikzpicture}[use optics]
  \draw[put coordinate=P at 0.3,put coordinate=Q at 0.7] (0,0)
    to[bend left] (2cm,0);
  \draw[red] (P) -- (1cm,-1cm);
  \draw[blue] (Q) -- (1cm,-1cm);
\end{tikzpicture}  
\end{codeexample}

\end{key}

\subsubsection{Label dimensions and distances on figures}

\begin{warning}
  This part is experimental and might change a lot without warning.
\end{warning}

Let us start with an example.

\begin{codeexample}[width=6cm]
\begin{tikzpicture}
\draw[fill=yellow!30]
   (-1cm,0) coordinate (A) -- (1cm,0) coordinate (B)
   -- (1cm,1.5cm) coordinate (B') -- (0.25cm,2.5cm) coordinate (b)
   -- (-0.25cm,2.5cm) coordinate (a) -- (-1cm,1.5cm) coordinate (A')
   -- cycle;

\draw (A) to[dim arrow'={label'=$D$}] (B);
\draw (A) to[dim arrow={label=$H$}] (A');
\draw (a) to[short dim arrow={label=$d$,label near middle}] (b);
\draw (B') to[dim arrow'={label'=$h$}] (b -| B');
\end{tikzpicture}  
\end{codeexample}



\begin{stylekey}{/tikz/dim arrow=\meta{sous-clés}}
  The style |dim arrow| allows to label distances. It applies to a |to path|. For instance,

  \begin{codeexample}[width=4cm]
\begin{tikzpicture}[use optics]
  \node[circle,fill=red,inner sep=2pt] (a) at (0,0) {};
  \node[circle,fill=red,inner sep=2pt] (b) at (2cm,0) {};
  \draw (a.center) to[dim arrow={label=$\ell$}] (b.center);
\end{tikzpicture}  
  \end{codeexample}

  The dimension arrow is shifted with respect to the initial and final position in order not to overlap with the labeled object.

  Several subkeys can be used through |dim arrow={|\meta{subkeys}|}|.

  \begin{key}{/tikz/dim arrow/label=\meta{text}}
  The key |label| determiens the \meta{text} to show on the arrow.
  It is drawn on the left of the arrow with respect to the path orientation (via |/tikz/auto=left|).
  \end{key}

  \begin{key}{/tikz/dim arrow/label'=\meta{text}}
  The key |label'| does the same thing as |label|, with \meta{text} drawn on the right.
  \end{key}


\begin{codeexample}[]
\begin{tikzpicture}
  \node[rectangle, draw=black, fill=yellow!30, minimum width=1cm, minimum height=1cm] (R) at (0,0) {};

  \draw (R.north east) to[dim arrow'={label'=$\ell'$}] (R.south east);
  \draw (R.north east) to[dim arrow'={label=$\ell$}] (R.south east);
\end{tikzpicture}
\end{codeexample}

  \begin{key}{/tikz/dim arrow/label text=\meta{text}}
  The key |label text| determines the \meta{text} to add on the arrow without modifying its position.
  \end{key}

  \begin{stylekey}{/tikz/dim arrow/label style}
  The key |label style| determines the style to use to draw the label.
  This style should not be replaced (expect on purpose) as it is used to position the label.
  Instead, styling should be appended with |/.append style|.

\begin{codeexample}[]
\begin{tikzpicture}
  \node[rectangle, draw=black, fill=yellow!30, minimum width=1cm, minimum height=1cm] (R) at (0,0) {};

  \draw (R.north east) 
    to[dim arrow={label=$L$, label style/.append style=red}] (R.south east);
  \draw (R.north west)
    to[dim arrow'={label'=$L$}, red] (R.south west);
\end{tikzpicture}
\end{codeexample}
  \end{stylekey}

  \begin{key}{/tikz/dim arrow/label pos=\meta{number} (initially \pgfkeysvalueof{/tikz/dim arrow/label pos})}
  The key |label pos| determines the position \meta{number} where the label should be drawn.
  This key does not apply to |short dim arrow|.

\begin{codeexample}[]
\begin{tikzpicture}
  \node[rectangle, draw=black, fill=yellow!30, minimum width=2cm, minimum height=0.5cm] (R) at (0,0) {};

  \draw (R.north west) 
    to[dim arrow={label=$\ell$, label pos=0.25}] (R.north east);
\end{tikzpicture}
\end{codeexample}
  \end{key}

  \begin{key}{/tikz/dim arrow/label near start}
  Equivalent to |label pos=0| (for |short dim arrow|, see the specific documentation below).
  \end{key}

  \begin{key}{/tikz/dim arrow/label near middle}
  Equivalent to |label pos=0.5| (for |short dim arrow|, see the specific documentation below).
  \end{key}

  \begin{key}{/tikz/dim arrow/label near end}
  Equivalent to |label pos=1| (for |short dim arrow|, see the specific documentation below).
  \end{key}

  \begin{key}{/tikz/dim arrow/raise=\meta{length} (initially \pgfkeysvalueof{/tikz/dim arrow/raise})}
  The key |raise| determines the distance \meta{length} between the dimension arrow and the initial path.
  \begin{codeexample}[]
\begin{tikzpicture}
  \node[rectangle, draw=black, fill=yellow!30, minimum width=1cm, minimum height=1cm] (R) at (0,0) {};

  \draw (R.north east) 
    to[dim arrow={label=$\ell$, raise=0.5cm}, black] (R.south east);
  \draw (R.north east) 
    to[dim arrow={label=$\ell$, raise=1cm}, red] (R.south east);
\end{tikzpicture}
\end{codeexample}
  \end{key}

  \begin{stylekey}{/tikz/dim arrow/no raise}
  Equivalent to |raise=0|.
  \end{stylekey}

  \begin{key}{/tikz/dim arrow/->}
  When |->| is used, an arrow is drawn on one side only (selected in the same way as |->| in tikz).
  This can be used to label an algebraic length.
  \begin{codeexample}[]
\begin{tikzpicture}
  \node[rectangle, draw=black, fill=yellow!30, minimum width=1cm, minimum height=1cm] (R) at (0,0) {};

  \draw (R.north east) 
    to[dim arrow={->, label=$\overline{L}$, raise=0.5cm}] (R.south east);
  \draw (R.north west) 
    to[dim arrow={<-, label'=$\overline{H}$, raise=-0.5cm}] (R.south west);
\end{tikzpicture}
\end{codeexample}
  \end{key}

    \begin{key}{/tikz/dim arrow/<-}
  Similar to |/tikz/dim arrow/->|, but in the other direction.
  \end{key}

\end{stylekey}

\begin{stylekey}{/tikz/dim arrow'=\meta{sous-clés}}
  The style |dim arrow'| does the same thing as |dim arrow| with the initial value |/tikz/dim arrow/raise| set to |-|\texttt{\pgfkeysvalueof{/tikz/dim arrow/raise}} instead of \texttt{\pgfkeysvalueof{/tikz/dim arrow/raise}}.

\begin{codeexample}[]
\begin{tikzpicture}
  \node[rectangle, draw=black, fill=yellow!30, minimum width=1cm, minimum height=1cm] (R) at (0,0) {};

  \draw (R.north east) to[dim arrow'={label=$\ell$}, black] (R.south east);
  \draw (R.north east) to[dim arrow={label=$\ell$}, red] (R.south east);
\end{tikzpicture}
\end{codeexample}
\end{stylekey}


\begin{stylekey}{/tikz/short dim arrow=\meta{sous-clés}}
  The style |short dim arrow| has the same aim as |dim arrow|. It is used when the dimension to label is too small, so that the arrows should be on the outside.

  \begin{codeexample}[width=4cm]
\begin{tikzpicture}[use optics]
  \node[circle,fill=red,inner sep=2pt] (a) at (0,0) {};
  \node[circle,fill=red,inner sep=2pt] (b) at (1cm,0) {};
  \draw (a.center)
  to[short dim arrow={label=$\ell$}]
  (b.center);
\end{tikzpicture}  
  \end{codeexample}

  The options of |/tikz/dim arrow| apply, with some differences and additional options discussed below.

  \begin{key}{/tikz/dim arrow/label near start}
  The key |label near start| positions the label near the beginning of the path.
  This is the default option.

\begin{codeexample}[]
\begin{tikzpicture}
  \node[rectangle, draw=black, fill=yellow!30, minimum width=1cm, minimum height=0.5cm] (R) at (0,0) {};

  \draw (R.north east) to[short dim arrow={label=$\ell$}] (R.south east);
\end{tikzpicture}
\end{codeexample}
  \end{key}

  \begin{key}{/tikz/dim arrow/label near end}
  The key |label near end| positions the label near the end of the path.

\begin{codeexample}[]
\begin{tikzpicture}
  \node[rectangle, draw=black, fill=yellow!30, minimum width=1cm, minimum height=0.5cm] (R) at (0,0) {};

  \draw (R.north east)
      to[short dim arrow={label=$\ell$, label near end}] (R.south east);
\end{tikzpicture}
\end{codeexample}
  \end{key}

  \begin{key}{/tikz/dim arrow/label near middle}
  The key |label near middle| positions the label near the middle of the path.
  In this case, the label is not shifted with respect to the arrow (as there is no overlap). 
  This can be restored using |/tikz/dim arrow/label style|.

\begin{codeexample}[]
\begin{tikzpicture}
  \node[rectangle, draw=black, fill=yellow!30, minimum width=1cm, minimum height=0.5cm] (R) at (0,0) {};

  \draw (R.north east)
      to[short dim arrow={label=$\ell$, label near middle}] (R.south east);
\end{tikzpicture}
\end{codeexample}
  \end{key}



  \begin{key}{/tikz/dim arrow/arrow length=\meta{length} (initially \pgfkeysvalueof{/tikz/dim arrow/arrow length})}
  The key |arrow length| determines the \meta{length} of the external arrows.
  \end{key}
\end{stylekey}

\begin{stylekey}{/tikz/short dim arrow'=\meta{sous-clés}}
  The style |short dim arrow'| does the same thing as |short dim arrow|, with the initial value of |/tikz/dim arrow/raise| set to |-0.5cm| instead of |0.5cm|.

  \begin{codeexample}[width=4cm]
\begin{tikzpicture}[use optics]
  \coordinate (a) at (0,0);
  \coordinate (b) at (1cm,0);
  \draw[red,mark=x, draw=none] plot coordinates {(a) (b)};
  \draw (a.center) to[short dim arrow, blue] (b.center);
  \draw (a.center) to[short dim arrow', red] (b.center);
\end{tikzpicture}  
  \end{codeexample}
\end{stylekey}


\subsubsection{Eyes}

\begin{stylekey}{/tikz/pics/optics eye}
The pic |optics eye| draws a stylized eye.
Note that |optics eye| is not a |shape|, but a |pic| (see the tikz manual).

\begin{codeexample}[]
\begin{tikzpicture}[use optics]
  \node[mirror,rotate=-90] (M) at (0,0) {};

  \pgfmathsetmacro\rayangle{70}
  \draw[red,->-] ($(M.center)+({90+\rayangle}:1.5cm)$) -- (M.center);
  \draw[red,->-] (M.center) -- ($(M.center)+({90-\rayangle}:1.5cm)$) 
  coordinate (out) {};
  \pic[rotate={90-\rayangle}] (eye) at (out) {optics eye};
\end{tikzpicture}
\end{codeexample}


How the eye is drawn is controlled through |optics eye={\meta{styles}}|.
In particular, the styles |pupil|, |cornea|, |iris| and |contour| can be modified (e.g. with |append style|) to change how the eye is drawn.

\begin{codeexample}[]
\begin{tikzpicture}[use optics]
\pic (colorful eye) {optics eye={
cornea/.append style={red, fill=red!10},
iris/.append style={blue, fill=blue!10},
pupil/.append style={orange},
contour/.append style={green!60!black},
}};
\end{tikzpicture}
\end{codeexample}

When a |optics eye| pic is added at position |(0,0)|, it is shifted by a quantity |optics eye/shift| that can be controlled through |optics eye={shift=\meta{length}}| so that the light does not collide with the eye.
Two coordinates |(\meta{name}-in)| and |(\meta{name}-center)| are created; for instance, |\pic (eye) at (0,0) {optics eye};| will produce the coordinates |(eye-in)| [at |(0,0)|] and |(eye-center)|. They are separated by the length |optics eye/shift| and their positions are explained in the picture below.

\begin{tikzpicture}[use optics]
\pic[scale=3, ultra thick] (eye) at (0,0) {optics eye={
shift=0.25cm
}};
\begin{scope}[every node/.append style={cross out,draw,inner sep=0pt,minimum width=4pt,minimum height=4pt}]
\node[red, thick] at (eye-in) {};
\node[orange, thick] at (eye-center) {};
\end{scope}
\draw[/tikz/dim arrow/label style/.append style={below=0.2cm}] (eye-in) to[dim arrow={raise=-1.25cm, label'={\texttt{optics eye/shift}}}] (eye-center);

\draw[shorten <=0.2cm,<-,>=latex] (eye-in) -- +(120:1cm) node[above] {\texttt{(0,0)}};
\end{tikzpicture}

\end{stylekey}


\section{Thanks}

I thank the colleagues and friends who had the unfortunate privilege of testing the first versions of this library, and in particular the authors of the book \emph{Physique expérimentale}\footnote{\url{http://www.physique-experimentale.com/}} for which the library was developed.


\end{document}